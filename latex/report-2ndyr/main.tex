\documentclass[10pt,a4paper]{article}
\usepackage[margin=1in]{geometry}
\usepackage[utf8]{inputenc}
\usepackage{xcolor}% http://ctan.org/pkg/xcolor
\usepackage{color,soul}
\usepackage{sectsty}% http://ctan.org/pkg/sectsty
\usepackage{amsmath}
\usepackage{amsfonts}
\usepackage{amssymb}
\usepackage{wrapfig}
\usepackage{pgfgantt}
\usepackage{natbib}
\bibliographystyle{bibstyle}
\setcitestyle{authoryear, open={(},close={)}}
\usepackage[hidelinks]{hyperref}
\usepackage{graphicx}
\usepackage{multirow}
\usepackage{longtable}
\usepackage{pdflscape}
\usepackage{titlesec}
\usepackage{caption}
\usepackage{subcaption}
\usepackage{array}
\usepackage{booktabs}
\usepackage{nicefrac}
\usepackage{makecell}
\usepackage{lineno}
\usepackage{fancyhdr}
\usepackage[title]{appendix}
\graphicspath{{../report-firstyr/figs/}}
\renewcommand{\arraystretch}{1.2}
\renewcommand{\d}{\mathrm{d}}
\newcommand*\mean[1]{\bar{#1}}

\definecolor{blue}{RGB}{0,0,204}
\definecolor{red1}{RGB}{102,0,14}
\definecolor{teal1}{RGB}{0,138,124}
\definecolor{teal2}{RGB}{0,212,184}
\definecolor{darkgrey}{RGB}{25,55,50}

\definecolor{frugalGreyishPurple}{RGB}{136,113,145}
\definecolor{polledElectricPurple}{RGB}{170,35,255}
\definecolor{balkingMud}{RGB}{115,92,18}
\definecolor{silver}{RGB}{197,201,199}
%\definecolor{palelilac}{RGB}{228,203,}

\captionsetup[figure]{font=footnotesize,labelfont={it,bf,footnotesize,color=teal1}}
\captionsetup[table]{font=footnotesize,labelfont={it,bf,footnotesize,color=teal1}}

\setulcolor{polledElectricPurple}

\sectionfont{\color{darkgrey}\itshape}
\subsectionfont{\color{frugalGreyishPurple}\itshape}
\subsubsectionfont{\color{teal1}} %teal1


\pagestyle{fancy}
%\thispagestyle{empty}

\author{C. M. Guimond \\ \normalsize O. Shorttle \& J. F. Rudge, supervisors \\ \textit{Second-Year Report to the Department of Earth Sciences}}
\title{Rocky exoplanet surfaces: \\ \large Theoretical limits on land area}

\fancyhead[L]{Exoplanet topography}
\fancyhead[R]{C. M. Guimond}

\begin{document}


\begin{figure}
    \centering
    \begin{minipage}{0.5\textwidth}
        \centering
        \includegraphics[width=0.25\textwidth]{cats} % first figure itself
    \end{minipage}\hfill
    \begin{minipage}{0.5\textwidth}
       r \centering
        \includegraphics[width=0.25\textwidth]{cam} % second figure itself
    \end{minipage}
\end{figure}

\maketitle
\tableofcontents
%\linenumbers


\section{Updates since the last report}

\begin{itemize}

\item Given virtual presentations on exoplanet topography work, both internal and external

\begin{enumerate}
\item Europlanet Science Congress, 21 September -- 9 October 2020 [poster]
\item Bullard first-year talk, 12 February 2021 [talk]
\item Grad talk (downtown), 5 March 2021 [talk]
\item UK Exoplanet Meeting, 22 April 2021 [talk]
\item Cambridge Exoplanets and Life Day, 8 June 2021 [talk]
\end{enumerate}

\item Decided to include a contemporaneous first-author manuscript in my thesis (Paper 1), which I've been leading with a group outside Cambridge.

\item Nearly (this week?) ready to re-submit third review of Paper 1


\item Completed draft of first work on exoplanet dynamic topography (Paper 2).

\begin{itemize}

\item No more new numerical convection simulations to perform (notwithstanding referee report), although some runs are still going

\item Integrated dynamic topography scaling relationships into 1D thermal history model

\item Arrived at estimates of a planetary "ocean basin capacity" (given only dynamic topography), the amount of liquid water that can be stored before flooding the surface

\end{itemize}

\end{itemize}


\subsection{Abstract of Paper 1 (under review at PEPI)}

Before the origin of life, Earth's atmosphere was built up by outgassing of the mantle. Here we model how C-O-H outgassing could have evolved through the late Hadean and early Archean, under the conditions that global plate tectonics had not yet initiated, all outgassing was subaerial, and graphite was the stable carbon phase in the melt source regions. The model couples numerical mantle convection, partitioning of volatiles into the melt, and chemical speciation in the gas phase. Of the parameters tested, the mantle oxidation state has the largest effect on individual species' gas fluxes because it controls both the volatile speciation and the carbon content of basaltic magmas. Hence virtually no CO$_2$ would be outgassed below the iron-w\"ustite mineral redox buffer: in conditions this reduced, (i) carbonate ions dissolve in magmas only in very limited amounts, and (ii) almost all degassed carbon takes the form of CO instead of CO$_2$. Even for oxidised scenarios near the quartz-fayalite-magnetite buffer, we predict CO$_2$ outgassing rates of less than 5 Tmol yr$^{-1}$. This result contrasts common assumptions in Archean climate studies. Relatively small fluxes are due in part to our model's strong dependence of CO$_2$ melt concentrations on the mantle oxidation state (which may not have reached late Archean values in the Hadean). Imposing a stagnant lid regime also seems to curb outgassing rates, which points to the importance of the presumed tectonic mode on this vital atmospheric input. Our results could suggest that certain chemical and geodynamic conditions may have been required if outgassing were to supply enough greenhouse gas for a clement climate under the Faint Young Sun.



\subsection{Abstract of Paper 2 (submitting to PSJ)}


Topography on a rocky exoplanet would raise land above its hypothetical sea level. Although land elevation is the product of many complex processes, the large-scale topographic features on any geodynamically-active planet are the expression of the convecting mantle beneath the surface. This so-called ``dynamic topography" exists regardless of a planet's tectonic regime or volcanism; its amplitude, with a few assumptions, can be estimated via numerical simulations of convection as a function of the mantle Rayleigh number. Here, we develop new scaling relationships for dynamic topography on stagnant lid planets using 2D convection models with temperature-dependent viscosity. These scalings are applied to 1D thermal history models to explore how topography varies with exoplanetary observables over a wide parameter space. Synthetic dynamic topography is converted to an ocean basin capacity, the minimum water volume required to flood the entire surface. A planet's basin capacity increases less steeply with its mass than does the amount of water itself, assuming a water inventory at a constant planetary mass fraction. These results suggest that dynamically-supported topography alone would be sufficient to maintain subaerial land on planets with surface water inventories of less than approximately $10^{-6}$ times their mass. Our work indicates that deterministic geophysical modelling might be used to inform land/ocean fractions, a key unknown in planetary climate studies.

\section{Thesis plan}

\subsection{Next steps}

\begin{itemize}
\item Paper 3 to focus on planetary limits to volcanic constructional topography
\begin{itemize}
\item Will implement pre-existing mantle melting model into 1D thermal history model
\item Will develop relatively simple physical model of the maximum height of these landforms, depending on bulk planetary properties (e.g., surface gravity)
\end{itemize}
\end{itemize}


\subsection{Calendar}

% gantt chart
\begin{ganttchart}[
    x unit=0.3cm,    
    y unit title=0.4cm,
    y unit chart=0.5cm,
    vgrid,
    time slot format=isodate-yearmonth,
    time slot unit=month,
    title/.append style={draw=none, fill=darkgrey},
    title label font=\footnotesize\bfseries\color{white},
    title label node/.append style={below=-1.7ex},
    title left shift=0,
    title right shift=-.05,
    title height=1,
    bar/.append style={draw=none, fill=teal1},
    bar height=.8,
    bar label font=\normalsize\color{black!50},
    group right shift=0,
    group left shift=0,
    group top shift=0.5,
    group height=.03,
    group peaks height=.05,
    bar incomplete/.append style={fill=teal2}
   ]{2021-01}{2023-10}
   \gantttitlecalendar{year}\\
   
   
\ganttset{progress label text={}, link/.style={black, -to}}
%\ganttgroup[group label font=\small\bfseries]{Controls on. }{2050-05}{2050-06} \\
%\ganttbar[
%    progress=0,
%    bar label font=\scriptsize,
%    name=pp, bar incomplete/.append style={fill=black}
%   ]{Paper manuscript}{2021-05}{2021-06} \\
%\ganttset{progress label text={}, link/.style={black, -to}}



\ganttgroup[group label font=\small\bfseries]{\textit{Paper 1:} Archean outgassing}{2021-01}{2021-06} \\
\ganttbar[progress=0, name=T1A, bar label font=\scriptsize, bar incomplete/.append style={fill=frugalGreyishPurple}]{Review}{2021-01}{2021-06} \\
\\


\ganttset{progress label text={}, link/.style={black, -to}}
\ganttgroup[group label font=\small\bfseries]{\textit{Paper 2:}Dynamic topography}{2021-01}{2021-08} \\
\ganttbar[progress=0, name=T1A, bar label font=\scriptsize, bar incomplete/.append style={fill=silver}]{Experiments}{2021-01}{2021-06} \\
\ganttbar[progress=0, name=T1A, bar label font=\scriptsize, bar incomplete/.append style={fill=frugalGreyishPurple}]{Initial manuscript submission}{2021-02}{2021-08} \\
%\ganttlinkedbar[progress=0, bar label font=\scriptsize, bar incomplete/.append style={fill=orange}]{Write up}{2022-04}{2022-09} 
\\

\ganttset{progress label text={}, link/.style={black, -to}}
\ganttgroup[group label font=\small\bfseries]{\textit{Paper 3:} Volcanic topography }{2021-07}{2022-08} \\
\ganttbar[progress=0, name=T1A, bar label font=\scriptsize, bar incomplete/.append style={fill=silver}]{Experiments}{2021-07}{2022-01} \\
\ganttbar[progress=0, name=T1A, bar label font=\scriptsize, bar incomplete/.append style={fill=frugalGreyishPurple}]{Initial manuscript submission}{2022-01}{2022-08} \\
\\


\ganttset{progress label text={}, link/.style={black, -to}}
\ganttgroup[group label font=\small\bfseries]{Bonus project }{2022-08}{2023-03} \\
\ganttbar[progress=0, name=T1A, bar label font=\scriptsize, bar incomplete/.append style={fill=frugalGreyishPurple}]{If it makes sense}{2022-08}{2023-03} \\
\\

\ganttset{progress label text={}, link/.style={black, -to}}
\ganttgroup[group label font=\small\bfseries]{Thesis}{2022-10}{2023-10} \\
\ganttbar[progress=0, name=T1A, bar label font=\scriptsize, bar incomplete/.append style={fill=frugalGreyishPurple}]{Writing}{2022-10}{2023-10} \\
\\


\end{ganttchart}









%\section*{Abstract}

Topography is a crucial component of the Earth system: having exposed fresh rock lets surface temperatures self-regulate via silicate weathering. However, there are limits to a lithosphere’s capacity to support mountains or valleys over geologic time. We see in our solar system that the range in a body’s elevations tends to decrease with increasing planet mass. Meanwhile, most currently-known exoplanets are in between the sizes of Earth and Neptune---a regime totally exotic to the solar system---and many have apparent bulk densities consistent with a rocky composition. Here we investigate how massive an exoplanet can be before it becomes hypsometrically featureless. We extrapolate solar system trends in topography to massive rocky exoplanets using well-tested models from geodynamics, starting with 1D parameterized convection models that predict dynamic topography.


%\listoffigures

%\section{Background}

The interior of a planet affects its surface character. Motions in the mantle are continuously modulating important aspects of the entire planetary system like the volume of its oceans, the composition and pressure of its atmosphere, its surface elasticity, its magnetic field, and its overall climate and habitability \citep{Noack2014, Foley2016, Wordsworth2016, Tosi2017, Wordsworth2018, Shahar2019}. This fact is not always explicit in studies of exoplanets \citep{Shahar2019}, as astrophysics cannot directly constrain geophysical models. Nevertheless, advances brought on by the characterization era of exoplanet science\footnote{i.e., atmospheric spectroscopy} now permit some description of planetary interiors as well, through measurements of a planet's bulk density, the host star chemistry, and the detection of atmospheric species \citep{Santos2017, Dorn2017, Dorn2017a, Dorn2018, Bower2019, Madhusudhan2020}. In this way, exoplanet science is an intrinsically interdisciplinary field, drawing on geophysics and geochemistry to study planets around other stars.

Not only can Earth teach us about exoplanets, but exoplanets can place Earth in an illuminating context. We now know exoplanets are as common as stars, so even basic measurements across a sample have statistical power to complement more detailed studies of the cosmically-unrepresentative solar system. Indeed, other planetary systems are found to be very unlike the one we know: the first detection of an exoplanet revealed this fact with a bafflingly close-in Jupiter-mass world \citep{Mayor1995}. As exoplanet occurrence rate studies reveal \citep{Petigura2013, Foreman-Mackey2014, Dressing2015, Kunimoto2020}, every Sun-like star\footnote{FGK spectral class} in the solar neighbourhood is statistically likely to host at least one planet of mysteriously intermediate mass, between Earth and Neptune---before the detection of exoplanets we had no reason to imagine these existed. 

In the interim, before more detailed observational constraints on massive potentially-rocky planets can be made, there grows a theoretical literature on how geophysical matters scale with mass: internal structure \citep{Valencia2006, Zeng2017}, tidal response \citep{Tobie2019}, outgassing \citep{Kite2009, Noack2017, Dorn2018a}, geodynamos \citep{Gaidos2010}, tectonics \citep{ONeill2007, Korenaga2010}, and so on. Here we are interested in the planet mass-scaling of a particular expression of interior dynamics not typically considered in the exoplanet context: the occurence of topography. Topography is tightly linked to temperatures and mantle flows inside the planet. Therefore the problem is one of understanding a planet's thermal history.

\subsection{Terrestrial planet interiors through time}

%The late stage of planetary accretion leaves a molten and unrecognizable world \citep{Elkins-Tanton2012}. The magma ocean planet crystallizes to rock \footnote{unless the planet's surface is kept too hot such as by proximity to the star, in which case crystallization can take 100 million years, not a few million \citep{Hamano2013}.}, and during cooling, the interior gravititationally differentiates, and volatile species partition between the mantle and the primary crust and atmosphere. These processes set the ``intitial" thermal and chemical state of the solid planet \citep{Tosi2019}. 

The transport of heat through a planet drives much of its dynamics. In the broadest sense, the thermal state of the planet can be described by the ratio of heat production to heat loss; that is, the Urey ratio:
\begin{align}\label{eq:Ur}
{\rm Ur} = \frac{Q_{\rm radiogenic}}{Q_{\rm surface}},
\end{align}
where $Q_{\rm radiogenic}$ and $Q_{\rm surface}$ are the whole planet-integrated fluxes in W of radiogenic heating and surface heat loss, respectively. A young planet can have so much radiogenic heating power that its Ur is much larger than 1. Three things happen with age: the hot core transfers its excess heat into the mantle; its interior abundances of potassium-40, uranium-235, uranium-238 and thorium-232 decline; and the surface loses heat to the atmosphere and eventually to space. This thermal evolution is described via the energy balances,
\begin{align}\label{eq:T_ODE}
\begin{split}
M_m c_{m} \frac{{\rm d}T_m}{{\rm d}t} &= -Q_{\rm u} + Q_{\rm rad} + Q_c, \\
M_c c_{c} \frac{{\rm d}T_c}{{\rm d}t} &= -Q_c,
\end{split}
\end{align}
where $M_m$ is the mantle mass in kg, $c_{m}$ is the mantle specific heat capacity in J kg$^{-1}$ K$^{-1}$, $Q_{\rm rad}$ is the mass-integrated radiogenic heat flux in W, $Q_{u}$ is the magnitude of the surface-integrated heat flux out of the top of the mantle in W (the subscript $u$ denotes the upper boundary layer). The analagous notation with subscript $c$ applies to the core. Although (\ref{eq:T_ODE}) oversimplifies the problem by omitting other heat fluxes like volcanism, it will suffice in capturing the basic behaviour \citep{Jaupart2015}.



\subsection{Parameterized convection models}

The temperature difference between the core-mantle boundary and the lithosphere drives convection, a process that has been modelled to varying degrees of complexity \citep[e.g.,][]{McKenzie1974, Nakagawa2015}. While advanced 3D models are more accurate than lower-dimensional models, running them is computationally expensive. Cheaper 1D parameterized convection models are more reasonable for certain reasearch questions \citep{Sharpe1979, Schubert1980, Davies1980}. When investigating distant planets' topography, where we have little-to-no constraints on the model parameters, computational inexpensiveness allows us to explore a much larger parameter space. Conversely, numerical convection models are better used to predict, for example, the exact shape and distribution of surface topography on Earth or Venus \citep[e.g.,][]{Moresi1995, Vezolainen2004}. 1D parameterized models have previously been applied to planets larger than Earth \citep{Valencia2009, Stamenkovic2012}.

Parameterized models work because most of the interesting behaviour of a convecting cell is controlled by relatively thin thermal boundary layers at the top and bottom of the cell---boundary layers are where the cell's thermal history is effectively determined. Parameterized models will be therefore be valid as long as dynamic processes in the thermal boundary layers occur faster than the planet loses heat \citep{Sharpe1979, Korenaga2008a}. 

As detailed in the appendix to this report, one can write scaling laws for the heat fluxes across the thermal boundary layers. The heat fluxes ultimately depend on the Rayleigh number, Ra. Ra is a nondimensional parameter which quantifies the ``vigour" of convection, and is equal to the ratio of time scales for heat transport by conduction to heat transport by convection:
\begin{equation}\label{eq:Ra}
\mathrm{Ra} = \frac{\tau_{\rm conduction}}{\tau_{\rm convection}}= \frac{\alpha \rho g \Delta T d^3 }{ \kappa \eta(T)},
\end{equation}
where $\alpha$ is thermal expansivity in K$^{-1}$, $\rho$ is density in kg m$^{-3}$, $g$ is surface gravity in m s$^{-2}$, $\Delta T$ is the temperature contrast across the layer in K, $d$ is the thickness of the layer in m, $\kappa$ is the thermal diffusivity in m$^{2}$ s$^{-1}$, and $\eta$ is the dynamic viscosity in Pa s. For a non-isoviscous convecting cell in 2D or 3D, there are many values of viscosity to choose from when defining Ra, so the location this viscosity corresponds to must be specified. Different types of Rayleigh numbers exist, which make different assumptions about what quantity is fixed in the model (the ``thermal" Ra in (\ref{eq:Ra}) fixes $\Delta T$, e.g.), so we must be cautious when comparing between them.

Below a critical Rayleigh number, Ra$_{\rm crit}$, convection is so weak it halts altogether. The value of Ra$_{\rm crit}$ depends on the wavelength of the disturbance that initiates convection, but often a constant value is assumed for a planetary mantle. Ra controls the thermal boundary layer in that the thickness of this layer, $\delta_u$, is approximately the value of $d$ at which convection is inefficient, Ra = Ra$_{\rm crit}$ locally. Therefore $\delta_u \propto {\rm Ra}^{-3}$.

The flux across the upper thermal boundary layer, $q_u$, scales with Ra via $\delta_u$:
\begin{equation}\label{eq:q_Ra}
q_{u} = -k_m \frac{a_{rh} \Delta T_{rh}}{\delta_{u}},
\end{equation}
where $k_m$ is the thermal conductivity in W m$^{-1}$ K$^{-1}$, $\Delta T_{rh}$ is the rheological temperature scale in K, set by the rate of change of viscosity with temperature (see Appendix \ref{sec:methods}), $a_{rh}$ is a constant of order unity, and $q_u$ is in W m$^{-2}$.




\subsection{Regimes of mantle convection}

Stagnant lids appear to be a natural consequence of temperature-dependent convection. Laboratory experiments observing cells of convecting corn syrup or golden syrup show that if the viscosity contrast between the centre and surface is large enough, the upper part of the thermal boundary layer will be dynamically-decoupled from the interior; i.e., not participate in convection at all. These so-called stagnant lids develop in convection cells with large viscosity contrasts \citep{Davaille1993, Giannandrea1993}. Silicate rock does have a strongly temperature-dependent rheology \citep{Karato1993}, and indeed Venus and Mars seem to behave as if they have a stagnant lid. Heat flows through the lid via conduction, while convection occurs only below the lid, where conditions are hotter and more viscous \citep{Morris1984, Christensen1984, Hansen1993, Solomatov1995}. %This fact inspires many questions about whether stagnant lid planets can form ``continents" and other topographic complexities. The character of certain tesserae on Venus imply tectonic origin, questioning whether this world always had a purely stagnant lid \citep{Bindschadler1991, Lenardic1991}. 
This scenario is illustrated in figure \ref{fig:stagnant_lid}.

Stagnant lid behaviour has been reproduced numerically \citep{Solomatov1995, Moresi1995a, Solomatov1996a}. Maps of convective regimes in viscosity contrast-Rayleigh number space can be found in \citet{Solomatov1996a} and more recently in \citet{Huttig2011} and \citet{Miyagoshi2015}, with stagnant lids favouring higher Rayleigh numbers \textgreater~$10^{6}$--10$^{7}$ and viscosity contrasts \textgreater~$10^4$ between the interior mantle and the surface.

We consider a stagnant lid regime under the premise that it represents a ``default" rocky planet \citep{ORourke2012}. In our own solar system, rocky planets are more frequently found in a stagnant lid regime than in a plate tectonic regime. With the assumption of a stagnant lid we avoid some unknown complexities that mobile plates introduce. Namely, plate margin-dominated heat loss on Earth works differently \citep[and more efficiently---stagnant lid planets will run hotter for the same rheology and surface heat flow;][]{Stevenson2003} than in a stagnant lid regime, where heat loss is simply determined by boundary layer scaling laws like those we have discussed.


\begin{figure}

  \centering
  \includegraphics[width=0.5\linewidth]{stagnantlid}

\caption{Structural model of a stagnant lid planet, not to scale. $R_p$ is the radius of the planet, $R_l$ is the radius at the base of the lid, $R_c$ is the radius of the core, $T_s$ is the surface temperature, $T_l$ is the temperature at $R_l$, $T_m$ is the isothermal convecting cell temperature, $T_c$ is the isothermal core temperature, and $\delta_u$ and $\delta_c$ are the upper and lower thermal boundary layer thicknesses respectively. The boundary layer heat fluxes $q_c$ and $q_u$ and the surface heat flux $q_s$ are defined in equations (\ref{eq:q_Ra}) and (\ref{eq:q_s}), respectively}. 
\label{fig:stagnant_lid}
\end{figure}







\subsection{Stress and topography}

Earth is hypsometrically bimodal: continents and oceans form separate peaks in the hypsometric curve. This is a complexity not seen on Mercury, Venus, the Moon, or Titan \citep{Keller2009, Lorenz2011}.\footnote{Mars has dichotomous crustal thickness split between its north and south hemispheres. This produces bimodality for different reasons than Earth, a further complexity.} Earth's topography is much more convoluted than anything a parameterized convection model could predict. Topography on stagnant lid planets, however, is more tenable to modelling.


\subsubsection{Mechanisms of topographic support}\label{sec:top_mechs}


To predict topography, we need to look at the forces balancing topographic loads. The first two mechanisms we discuss are not the focus of this document, but are nevertheless associated with the most dramatic topography on planets:
\begin{enumerate}
\item \emph{Elastic flexure} of a shell supports loads by way of elastic stresses that develop in the lithosphere;
\item \emph{Airy or Pratt isostasy} occurs when high mountains made of low-density crust are underlain by either deep roots of the same low density (Airy), or by roots of the same thickness as the surrounding plain but of lower density (Pratt).
\end{enumerate}
Whether a load is supported by (1) or (2) is set by the ratio of the width of the load to a flexural parameter, which depends on the lithosphere's elastic properties.\footnote{$\alpha_{\rm flex} = \left[\frac{1}{3(1 - \nu^2)}\frac{Ed_e^3}{\rho_m g}\right]^{1/4}$, where $d_e$ is the lithosphere thickness, $\rho_m$ is the underlying mantle density, $\nu$ is Poisson's ratio, and $E$ is Young's modulus.}
%\begin{equation}
%\alpha_{\rm flex} = \left[\frac{1}{3(1 - \nu^2)}\frac{Ed_e^3}{\rho_m g}\right]^{1/4},
%\end{equation}
%where $d_e$ is the lithosphere thickness, $\rho_m$ is the underlying mantle density, $\nu$ is Poisson's ratio, and $E$ is Young's modulus. 
If the width of a load is smaller than the flexural parameter, elastic stress in the lithosphere will support the load. If the width is much larger, buoyancy forces will support it. Hence flexure is associated with short-wavelength topography and isostasy is associated with long-wavelength topography. 

The last support mechanism is borne by convection in the interior. Historically the term has always not referred to the same phenomena, so it is helpful to break down \emph{dynamic topography} further \citep{Orth2011, Molnar2015} into:
\begin{enumerate}
\setcounter{enumi}{2}
\item \emph{Flow-induced tractions} on the base of the lithosphere. These tractions are exerted by the deformation of density boundaries within the viscously-flowing material below the thermal boundary layer.
\item \emph{Thermal isostasy,} variations in the thickness and thermal structure of the upper boundary layer \citep{Fowler1985}. On Earth this can be split into lithospheric and aesthenospheric components, but for planets without plates, this collapses to just variations in the viscous lid \citep[see][]{Orth2011}. This is unlike Airy and Pratt isostasy in that the density contrasts are thermal rather than compositional. Thermal isostasy provides the majority of ``dynamic topography."\footnote{Analytically we expect component 4 to dominate component 3 for an idealized sine-curve temperature perturbation at the surface, regardless of the wavelength of that perturbation \citep{McKenzie1977}. It can be shown that the gravity anomaly $\Delta g_1(x)$ considering no density contrast in the lithosphere, just the lithosphere's deflection upwards, is $\Delta g_1(x) = 2\pi G \Delta \rho \Delta h(x)$, where $G$ is the gravitational constant and $\Delta h$ is the height of topography. The gravity anomaly $\Delta g_2(x)$ associated with the flow-induced density contrasts alone is $\Delta g_2(x) = 2\pi/3 \; G\rho\Delta h_2(x)$. For a given gravity anomaly, $\Delta h_1 > \Delta h_2/3$.}
\end{enumerate}
Together, components 3 and 4 make the ``full dynamic topography." In a given model, teasing out the distinction between components 3 and 4 can present a problem when comparing between models, as we will see in section \ref{sec:dyn_top_ss}. 
%The distinction between components 3 and 4 matters, not necessarily because different parties disagree with their competing importance (on Earth), but because ``dynamic topography" does not consistently refer to either or both components. %In some circumstances, the two mechanisms can even produce identical signals \citep{Molnar2015}. 



%It can be shown that the topography associated with (3) is about a third of the topography associated with (4) \citep{McKenzie1968, McKenzie1977, Molnar2015}. 


\subsubsection{Dynamic topography forward models}\label{sec:dyn_top_forward}

%\citet{McKenzie1968, 1977} derives analytic half-space equations for $\Delta h$ as a function of a harmonic temperature perturbation $T(x) = T_0 \cos(2\pi x/\lambda)$, associated with both static and dynamic density contrasts, In both cases the solution for $\Delta h(x)$ is propotional to $\alpha T_0 \lambda / (2\pi) \cos(2\pi x/\lambda)$, with the static scenario a factor of \nicefrac{4}{3} higher because it does not have to overcome resistance to flow from an additional vertical shear stress term (Stokes equation). 

Numerical models can calculate the amplitude of the dynamic component of topography, $\Delta h$, by solving the equations of motion and the heat transport equation for a convecting cell, and obtaining the velocity and temperature fields. The total stress in the vertical direction, $\tau_{zz}$, can then be calculated at the surface of the cell. Total vertical stress is given by 
\begin{equation}\label{eq:tau_zz}
\tau_{zz} = 2\eta \; \frac{\partial u_z}{\partial z} - p_1,
\end{equation}
where $\partial u_z / \partial z$ is the vertical velocity in m s$^{-1}$ and $p_1$ is the pressure pressure perturbation from thermal convection in Pa \citep{Parsons1983}. The first term in (\ref{eq:tau_zz}) is the viscous stress. The hydrostatic pressure $p_0$ of a topographic load balances this stress, $p_0 = \rho g \Delta h = \tau_{zz}$, where $\Delta h$ is in m. Combining this and (\ref{eq:tau_zz}) with self-consistent pressure and temperature profiles inherently includes both dynamic topography components, producing the full dynamic topography,
\begin{equation}\label{eq:h_stress}
\Delta h = \frac{\tau_{zz}}{\rho g}.
\end{equation} 

To separate the thermal and flow-induced components of (\ref{eq:tau_zz}), expressions for the maximum topography due solely to thinning of the lithosphere can be derived: 
\begin{equation}\label{eq:h_th}
\Delta h \sim 0.5\alpha (T_i - T_s) z_{l, 0},
\end{equation}
where $T_s$ is the surface temperature in K, $T_i$ is the temperature below the lithosphere in K at the point of interest, and $z_{l, 0}$ is the average thickness of the lithosphere in m \citep{Kucinskas1994, Orth2011}. Although this thermal thinning component may explain most of the full dynamic topography on stagnant lid planets, using (\ref{eq:h_th}) to approximate dynamic topography \citep[the ``isostatic stagnant lid approximation";][]{Orth2011} requires knowledge of the lateral variations of temperatures under the lithosphere, which is not possible with 1D models.


In principle, one can obtain $\tau_{zz}$ in (\ref{eq:tau_zz}--\ref{eq:h_stress}) from parameterized convection, since the upper thermal boundary layer contributes most of the depth-integrated stress \citep{Parsons1983, Solomatov1995}. The parameterized stress is given by
\begin{equation}\label{eq:tau_param}
\tau_{zz} = C_1 \alpha_m \Delta T_{rh} \delta_u.
\end{equation}
\citet{Reese2005} give proportionality constant $C_1 = 2$ for the shear stress at the lid generated by sinking plumes, based on fits to a numerical convection model with spherical geometry and temperature-dependent viscosity. We note that it is the normal component of stress, not the shear stress, that balances $\rho g \Delta h$. However, no other scaling is available to our knowledge, so we adopt this value as a preliminary measure. We treat $\tau_{zz}$ as the planetary root-mean-square (RMS) value for convective stress, producing the RMS dynamic topography, $\Delta h_{\rm RMS}$. 

According to (\ref{eq:tau_param}), we expect $\Delta h_{\rm RMS}$ to scale with the RMS thermal boundary layer thickness, which scales with Ra as $\delta_u \propto {\rm Ra}^{-3}$. The strength of this scaling would be affected by other modelling decisions, such as the rheology law and the amount of internal heating compared to basal heating \citep{McKenzie1977}. Under particular model assumptions, scalings of $\Delta h$ with Ra can be found in the existing literature.

For the first scaling, we combine (\ref{eq:h_stress}) and (\ref{eq:tau_param}) in terms of Ra via (\ref{eq:d_u}):
\begin{equation}\label{eq:dyn_top_stress}
\Delta h_{\rm RMS} \sim \alpha_m d \Delta T_{rh} \left(\frac{\rm Ra}{\rm Ra_{\rm crit}}\right)^{-\frac{1}{3}}.
\end{equation}
(\ref{eq:dyn_top_stress}) is equivalent to equation (34) in \citet{Parsons1983}, which is a boundary-layer-based approximation of their equation (33),
\begin{equation}\label{eq:PD83_0}
\Delta h = C_2 \frac{\eta_m \kappa_m}{\rho_m g d^2} \left({\rm Ra}_F\right)^n,
\end{equation}
where Ra$_F$ is the Rayleigh number based on surface heat flux,\footnote{Ra$_F = (\rho g \alpha d^4 q_u)/(k \kappa \eta)$} $n = 0.5$ \citep{McKenzie1977}, and $C_2 = 5.4$ to match the predictions of RMS dynamic topography in \citet{Lees2019} using a 3D isoviscous basal-heating convection model. Although it can be misleading to convert between Ra and Ra$_F$ as the model assumptions are fundamentally different, we can approximate (\ref{eq:PD83_0}) as a function of Ra with $n = 0.5$, using Ra$^{4/3} = \Delta T_m / \Delta T_{rh} {\rm Ra}_{\rm crit}^{1/3} {\rm Ra}_F$. This gives us the second scaling,
\begin{equation}\label{eq:PD83}
\Delta h_{\rm RMS} \sim \alpha_m d \left[\frac{a_{rh} \Delta T_{rh} \Delta T_m}{{\rm Ra}_{\rm crit}^{\frac{1}{3}}}\right]^{\frac{1}{2}} {\rm Ra}^{-\frac{1}{3}},
\end{equation}
where $a_{rh}$ is the rheological temperature scale prefactor \citep[equal to 2.44;][]{Thiriet2019}, and $\Delta T_m$ is the temperature difference across the convecting region in K. As expected, the exponent on Ra in (\ref{eq:PD83}) is equal to the exponent on Ra in (\ref{eq:dyn_top_stress}). The positive value of $n$ in (\ref{eq:PD83_0}) implies that topography increases with Ra$_F$ in this scaling---this is misleading because $\eta$ is outside of the exponential term.

The third scaling is a direct log-log fit by \citep{Kiefer1992} to the \emph{peak} value of $\Delta h$ over a mantle plume versus Ra, using a 2D cylindrical isoviscous numerical convection model applied to Venus:
\begin{equation}\label{eq:KH92}
0.7 \Delta h = 66 {\rm Ra}^{-0.121},
\end{equation}
where the factor of 0.7 = $(\rho_m - \rho_{\rm ocean} / \rho_m$ scales their water-loaded model to subaerial topography. We further scale this $\Delta h$ by 0.707 (the RMS value of a sine wave) to approximate $\Delta h_{\rm RMS}$.

 



























\begin{landscape}
\thispagestyle{empty}
%\begin{table}

\footnotesize


\begin{longtable}{ @{} p{4cm} r r p{2cm} p{2cm} r p{1.5cm} p{3.2cm} p{3.1cm} @{} } 
\caption{Predictions from numerical convection of dynamic topgraphies on Venus, with important model parameters noted. Internal heating is calculated as $(q_s - q_b)/q_s$, where $q_s$ and $q_b$ are the surface and basal heat fluxes in W m$^{-2}$ respectively. Reported values of the Rayleigh number are distringuished between that defined with a fixed temperature contrast, Ra (\ref{eq:Ra}), and that defined with a fixed basal heat flux, Ra$_B$. \;\;  *Calculated from the spherical harmonic power spectrum using $\Sigma_l [S(l)/(2l + 1)]^{1/2}$.} \label{tab:dyn_topo_obvs}\\



\toprule
\; & \multicolumn{2}{c}{\textsc{Dynamic topography} (km)} \\
\cline{2-3} \\
\textsc{Reference} & \textsc{Peak} & \textsc{RMS} & \textsc{Location} & \textsc{Viscosity} & Ra & \textsc{Internal heating} & \textsc{Model type} & \textsc{Dominant component}\\
\midrule 

\citet{Kiefer1991} & \makecell[tr]{7.5 \\ 5.2 \\ 3.6} & n/a  & multiple Regiones &  $f(z)$ & \makecell[tr]{Ra = $10^5$ \\ Ra = $10^6$ \\ Ra = $10^7$} & 0\% & Cylindrical plume & Viscous stress  \\
% after  \citep{Hager1985}


%\citet{Kiefer1992} &  7.5  & n/a & Global & Ra = $10^6$ & $\eta_m$ constant with high-$\eta$ stagnant lid & Numerical cylindrical plume & (3) & for $D_{\rm lid}$ = 130 km; gives scaling laws with Ra, assumes no internal heating (bottom of page 203). h from figure 9 constant visc \\$f(T)$


\citet{Moresi1995} & \makecell[tr]{5.8 \\ 3.8 \\ 5.1} & n/a & Atla Regio  &  \makecell[tl]{$f(T)$ \\ $f(T,z)$ \\ $f(T,z)$} & \makecell[tr]{Ra$_B$ = 2.4 $\times 10^6$ \\ Ra$_B$ = 1.3 $\times 10^6$ \\ Ra$_B$ = 1.0 $\times 10^6$} & 0\% & Axisymmetric convection & Thermal isostasy  \\
% Scaling: $\eta_0 \kappa / (d^2 \Delta\rho g)$ with $\eta_0$ from basal heating Ra, can't extrap to higher $\Delta\eta$ with linearized viscosity law. concerned with predicting admittance
 %

\citet{Nimmo1996} &  \makecell[tr]{1.16 \\ 1.47 \\ 2.85}  &  n/a  & Global & constant & \makecell[tr]{Ra$_B$ = 1.6 $\times 10^7$ \\  Ra$_B$ = 7.9 $\times 10^6$\\ Ra$_B$ = 4.0 $\times 10^6$ } & 0\% & Axisymmetric plume & Viscous stress \\



\citet{Solomatov1996a} & \makecell[tl]{$\sim$4 \\ $\sim$2} & n/a  & \makecell[cl]{Beta Regio \\ Average} & $f(T)$  & Ra = $3 \times 10^7$  & 0\% & Cartesian convection & Thermal isostasy  \\
%Beta Regio (avg) scaled so admittance is 30 (15) m/km, fixed $d_m$ = 1600 km 



\citet{Kiefer1998} &  5.4--10.9 & 1.6--2.7  & Global & constant & Ra = $10^6$ & 73 \% & Hemispherical axisymmetric convection & Thermal isostasy \\
% Scaling: $(\rho_m \alpha \Delta T R_p) / (\rho_m - \rho_s)$  internal heating Ra $10^7$

\citet{Vezolainen2003} & 3.5--6 & n/a & Beta Regio & $f(T)$  & Ra = $3 \times 10^7$ & 0\% & 2D Cartesian plume & Thermal isostasy  \\
%  fixed $\Delta T_m$ = 1100 K

\citet{Vezolainen2004} & 5.7 & n/a & Beta Regio & $f(T)$  & Ra = $3 \times 10^7$ & 0\% & 3D Cartesian plume & Thermal isostasy  \\
%  fixed $\Delta T_m$ = 1100 K



%\citet{Orth2011} & 2.2 nondim & n/a  & Global &  Frank-Kamenetskii & Ra$_i$ = 10$^7$ & 0\% & 3D spherical shell & Thermal isostasy \\



\citet{Golle2012} & 3.3 & 2.7*  & Global & $f(z)$ & Ra$_b = 3.8\times 10^8$ & 0\% & Viscoelastic deformation coupled with thermal convection &  Not enough information \\



\citet{Benesova2012} &  3.25 & n/a   & Alta Regio  & $f(z)$ & Ra = $2.8 \times 10^6$ & \textgreater 50\% & 3D spherical convection & Viscous stress \\
% specifically say no thermal isostasy, although Orth thesis says they do implicitly?



\citet{Huang2013} & 2\textendash 3 & 0.75 & Global & $f(T,z)$ & Ra = $1.8\times 10^7$ & 75\% & 3D spherical convection & Viscous stress \\
% goal is to simultaneously match number of plumes and GTR observations. semi-amplitude by eye from maps.  say case 15 is best fit to obvs. Ra fixed, using avg viscosity $2\times 10^{21}$ Pa s

\citet{Yang2016} & n/a & 0.27* & Global & $f(T,z)$ &  Ra = $7.3\times 10^6$ & 80\% & 3D spherical convection & Viscous stress \\
% they say "The gravity anomaly is the summation of that caused by the dynamic topography and by the density heterogeneity itself." and "the topography of volcanic rises is mainly due to dynamic uplift"


\bottomrule


\end{longtable}
%\end{table}
\end{landscape}


\begin{table}
\centering
\caption{Parameters used in this study. \label{tab:params}}
\footnotesize
\begin{tabular}{@{} c l r l p{4cm} @{}}
%\multicolumn{5}{l}{\textbf{Constant values for all planets}} \\
\toprule
Symbol & Description & Value & Units & Ref. \\
\midrule
\multicolumn{5}{c}{\textbf{Constant bulk properties}} \\
$\rho_c$ & Core density & 7200 & kg m$^{-3}$ &  \citet{Thiriet2019}  \\
$c_c$ & Core specific heat at constant volume & 840 & J kg$^{-1}$ K$^{-1}$  & \citet{Thiriet2019}  \\
$k_m$ & Mantle thermal conductivity & 4 & W m$^{-1}$ K$^{-1}$  & \citet{Thiriet2019}  \\
$\alpha_m$ & Thermal expansivity &  $2.5 \times 10^{-5}$ & K$^{-1}$  & \citet{Thiriet2019}  \\
$\kappa_m$ & Thermal diffusivity &  $1 \times 10^{-6}$ & m$^2$ s$^{-1}$  & \citet{Thiriet2019}  \\
$X_{\rm K}$ & Initial K abundance &  305 & wt ppm  & \citet{Jaupart2015} \\
$X_{\rm U}$ & Initial U abundance &  $16 \times 10^{-3}$ & wt ppm  & \citet{Jaupart2015} \\
$X_{\rm Th}$ & Initial Th abundance &  $56 \times 10^{-3}$ & wt ppm  & \citet{Jaupart2015} \\
$H_{4.5}$ & Radiogenic heating rate at 4.5 Gyr & $4.6\times 10^{-12}$ & W kg$^{-1}$ & \citet{Jaupart2015} \\
Ra$_{\rm crit}$ & Critical Rayleigh number & 450 &  & \citet{Thiriet2019}  \\
$E_a$ & Viscosity activation energy & 300 & kJ mol$^{-1}$ & \citet{Karato1993} \\
$A_{rh}$ & Viscosity preexponential factor & 8.7 $\times 10^{15}$ & & \citet{Karato1993} \\
$h_{rh}$ & Grain size & 2.07 & mm & \\
$B$ & Burgers vector & 0.5 & nm & \citet{Karato1993} \\
$m$ & Grain size exponent & 2.5 & & \citet{Karato1993} \\
$\mu$ & Shear modulus & 80 & GPa & \citet{Karato1993} \\
CMF & Core mass fraction & 0.3 & & \\
$L_*$ & Stellar luminosity & 1 & $L_{\rm Sun}$ &  \\
Al & Planetary geometric albedo & 0 & &  \\

\midrule
\multicolumn{5}{c}{\textbf{Solar system models}} \\
$M_p$ & Planet mass &  \makecell[tr]{\textbf{Mars:} 0.11 \\ \textbf{Venus:} 0.82} & \makecell[tl]{$M_\oplus$ \\ $M_\oplus$ } &  \\
$R_p$ & Planet radius &  \makecell[tr]{\textbf{Mars:} 3390 \\ \textbf{Venus:} 6050} & \makecell[tl]{km \\ km } &  \\
$R_c$ & Core radius &  \makecell[tr]{\textbf{Mars:} 1700 \\ \textbf{Venus:} 3330} & \makecell[tl]{km \\ km } &  \makecell[tl]{\citet{Thiriet2019} \\ \citet{Huang2013}} \\
$\rho_m$ & Mantle density & \makecell[tr]{\textbf{Mars:} 3500 \\ \textbf{Venus:} 3300} & \makecell[tl]{kg m$^{-3}$ \\ kg m$^{-3}$} &   \makecell[tl]{\citet{Thiriet2019} \\ \citet{Nimmo1996}} \\
$c_m$ & Mantle specific heat (p or v?) & \makecell[tr]{\textbf{Mars:} 1142 \\ \textbf{Venus:} 1200} & \makecell[tl]{J kg$^{-1}$ K$^{-1}$ \\ J kg$^{-1}$ K$^{-1}$}  & \makecell[tl]{\citet{Thiriet2019} \\  \citet{Nimmo1996}}  \\
$T_s$ & Surface temperature & \makecell[tr]{\textbf{Mars:} 250 \\ \textbf{Venus:} 730} & \makecell[tl]{K \\ K} \\

\midrule
\multicolumn{5}{c}{\textbf{Initial conditions}} \\
$T_{m,0}$ & Initial mantle temperature & 1750 & K & \citet{Thiriet2019}\\
$T_{c,0}$ & Initial core temperature & 2250 & K & \citet{Thiriet2019} \\
$D_{l,0}$ & Initial lid thickness & 300 & km & \citet{Thiriet2019}\\
\bottomrule
\end{tabular}
\end{table}



\subsubsection{Dynamic topography in the solar system}\label{sec:dyn_top_ss}

We aim to quantify the contribution of dynamic topography to total topography for solar system bodies so we can test our theoretical model of dynamic topography, but this is not so easy because the dynamic component of topography can be tricky to parse from observed topographic heights. Out of the terrestrial planets and moons, we have focused on Venus because of the general agreement in the literature that for certain highland locations, its observed topography can be represented by the full dynamic topography (\ref{eq:tau_zz}). 

%Here we give a short overview of attempts to model dynamic topography on stagnant lid planets. One can estimate the apparent depth of isostatic compensation using satellite measurements of the absolute topography and gravity or geoid anomaly. Larger values of the geoid with respect to the topography imply deeper isostatic compensation depths. Variations in the height of the geoid reflect distortions in density boundaries. %Inferences of dynamic topography are therefore quite sensitive to the assumed viscosity structure \citep{Karato2008a}.

%Another complementary approach \citep[e.g.,][]{Smrekar1991, Kucinskas1994} is to assume Airy isostasy and predict the gravity anomaly resulting from a loading at some depth. Comparing this to observed topography produces a guess of the isostatic compensation depth. If these predictions are far from reality, one concludes that isostasy does not provide all the support. We focus on the convection-based approach, which is more applicable to our needs. Nevertheless, the fact that such models have not reproduced the observed relationships between gravity and topography \citep{Kiefer1986} has been criticized \citep{Orth2011} in its conclusion that isostasy is irrelevant for certain features on Venus, and that Venusian ``dynamic" topography is purely tractional. This is because the arguments against isostasy are necessarily based on assumptions about how thick the realistic crust could be, which is not well constrained.

\vspace{0.5cm}

\textit{\color{teal1} Venus.} Although Venus is hypsometrically unimodal, unlike Earth and Mars, satellite altimetry reveals vast lowland plains dotted with areas of higher elevation. These elevated regions can be grouped into five older highland plateaux, steep-sided 2 km-high landmasses, and nine younger, dome-like volcanic rises \citep{Phillips1998}. Several of the volcanic rise features have been labelled active hotspots \citep{Kiefer1991, Smrekar1991, Grimm1992, Smrekar1994, Stofan1995, Smrekar2010}. The arrival of the Magellan spacecraft into Venus orbit in the 1990s brought gravity and topography measurements, and an influx of research followed, attempting to diagnose how its topography is supported. 

Based on the inferred relationships between its gravity or geoid and its topography, Venus' volcanic rises seem consistent with dynamic topography, while the highland plateaux are likely supported by Airy isostasy \citep{Kiefer1986, Grimm1991, Kiefer1991, Smrekar1994, McKenzie1994, Kucinskas1994, Smrekar1996, Nimmo1996, Simons1997, Pauer2006, James2013, Yang2016}. The dichotomy is not precise, and some regions are not fit by either a purely dynamic and a purely isostatic model. 

Under the paradigm that volcanic rises are dynamically-supported, modellers have been attempting to reproduce observed profiles of individual topographic features on Venus using numerical models of mantle plumes \citep{Kiefer1991, Kiefer1992, Moresi1995, Nimmo1996, Smrekar1996, Kiefer1998}. These numerical modelling efforts are summarized in Table \ref{tab:dyn_topo_obvs}. Table \ref{tab:dyn_topo_obvs} also lists predictions of Venus' topographic spectra from 2D and 3D convection simulations \citep{Golle2012, Benesova2012, Huang2013, Yang2016}. We are concerned with the output of convection models, not the absolute value of Venus' topography, since we are still not agreed on the actual dynamic component of its topography globally.


Advances in numerical modelling, such as including strongly-temperature-dependent viscosity (as expected for terrestrial planets) in at least two dimensions, have bestowed the current understanding that spatial variations in lithosphere thickness under thermal isostasy explain essentially all of Venus' full dynamic topography \citep{Kucinskas1994, Moore1995, Moore1997, Solomatov1996a, Orth2011}. That is, models approximating dynamic topography with only the thermal isostasy component have reproduced the heights of observed peaks on Venus. The requisite lateral variations in lithosphere thickness cannot be reproduced if viscosity is constant or depth-dependent. Nevertheless, we focus on studies that do include the tractional component of dynamic topography because the thermal isostasy approximation does not apply to 1D models.

%\begin{itemize}
%\item refer to definition crisis a bit more, maybe refer reader to column in Table \ref{tab:dyn_topo_obvs} where you sort these where possible
%\item do Parsons \& Daly consider thermal thinning? - claimed to not be. their stress balance has a hydrostatic term
%\end{itemize}


%Finally, lithosphere elasticity might complicate the prediction of dynamic topography by way of a non-negligible ``elastic filtering"  \citep{Zhong2002, Golle2012}. Although this effect is small for thin elastic thicknesses (Venus), it is large for large elastic thicknesses (Mars), and elastic thickness is unknown \textit{a priori}. We return to this in section \ref{sec:future-elastic}.



%\begin{itemize}

%\item from \citet{Yang2016}: Lowlands on Venus have negative gravity and geoid anomalies, and they are thought of as surface expressions of mantle downwellings (Bindschadler et al., 1992).



%\item  \citet{Simons1994} discuss something similar, their admittance values favouring the hypothesis that the nature of Venus' surface expressions of convection-crustal thickness coupling is transient rather than steady-state, although this paper and later one by the same authors \citep{Simons1997} argue that the present-day crust of Venus does not thin above upwelling plumes. 
%\item This is yet distinct from crustal thickening due to volcanism (which may be associated with a plume), which would be a mechanism within Airy isostasy.



%\item is the thing described by McKenzie 1994 figure 17 the same thing too? confused -- plume heat lowers viscosity of lower crust and it uplifts and  rifting occurs at the top, thrusting where lower crust at top of domr flows down around sides of dome. then as plume subsides, dome bulge subsides, but viscosity is high again and lower crust doesn't flow back





%\item .\citet{Smrekar1996} also suggest that decreasing positive GTRs across variuos volcanic rises suggests various ages of the plume; i.e., Beta Regio has the largest GTR because it is the most recent / hottest plume.  \citet{Kucinskas1994} interpreted an eastward increase in isostatic compensation in Aphrodite Terra as the decay of a hot mantle plume causing thermal thinning topography

%\item PEople have struggled to deal with lid or lithosphere thickness, trying to constrain it, make assumptions that it can't be larger than Earth's.  \citet{Kucinskas1994} argue that their model's 100-km-thick crust is actually real)



%\end{itemize}







\vspace{0.5cm}

\textit{\color{teal1} The rest.} Earth's dynamic topography is made more complex by plate motion, and is reviewed elsewhere. Hoggard (\citeyear{Hoggard2016}; 2020, in press) concludes that the non-thermal dynamic topography is $\pm1$ km on Earth. Retrievals of Mars' dynamic topography are complicated by the geoid anomaly associated with the 5000-km-wide Tharsis dome \citep{Phillips2001, Wieczorek2004}, which has been attributed to spherical harmonic degree-1 convection \citep{Zhong2001} or a giant impact \citep{Reese2006, Andrews-Hanna2008}. \citet{James2014} do not find a signal of active mantle convection in Mercury's topography. 	





\subsection{Topography in a systems science context}

Topography raises up land to be weathered, drawing carbon out of the atmosphere and transferring it to the oceans, where it eventually subducts into the mantle. This is the presumed primary mechanism by which Earth regulates its climate \citep{Walker1981}. There is a stabilizing feedback: warming the surface means faster weathering of silicate rock, drawing more CO from the atmosphere, weakening the greenhouse, and cooling the surface. Hence, the assumption that silicate weathering is affective is at the cornerstone of the classical circumstellar habitable zone theory \citep{Kasting1993}. That is, predictions of the width of the circumstellar liquid-water habitable zone around a star normally rely on the negative feedback of weathering; without this the habitable zone would be more narrow. However, to work efficiently, silicate weathering probably needs a minimum of exposed rock \citep{Abbot2012}. Although weathering also occurs on the seafloor \citep{Krissansen-Totton2017}, it is unclear whether this is efficient enough to sustain a stabilizing feedback loop. The location and surface area of exposed land exhibits complex far-field teleconnections that affect surface temperatures around the planet \citep{Sohl2017}. The impact of topography on climate has been considered for early Venus \citep{Way2016}. Major questions remain over climate regulation on rocky worlds lacking topographies: how robust would a silicate weathering feedback be? 

Another planetary system that may require land is biology. Uplifted land provides a means to concentrate minerals vital for prebiotic chemistry such as phosphate. Phosphorus is the limiting reagent in biochemical reactions (with respect to carbon and nitrogen), and part of any origins-of-life hypothesis. Further, the formation of aldehydes (precursor molecules of lipids, nucleic acids, and proteins) requires UV hydrolysis that cannot occur in the deep ocean.

\subsection{Statement of the problem}

While 2D and 3D numerical models of dynamic topography exist for stagnant lid planets with temperature-dependent viscosity, for the same assumptions, there are no published scaling laws as simple functions of planet bulk properties. We pursue these theoretical scaling laws, posing the specific questions:

\begin{enumerate}
\item Can we extrapolate predictive models of topography to more massive rocky planets? This would ideally draw on the extant literature modelling the interior dynamics of such planets (although we do not touch on these details in the present work).
\item Can we develop simple scaling relationships to examine the nature of how topography changes with planet mass?
\item Is there a planet mass-limit to useful topography? That is, would the topography at a certain point be so insufficient that it does not participate in other planetary processes?
\end{enumerate}

%The nature of these problems do not lend themselves to a deterministic approach, and ultimately our answers will be probablistic. That is, if someone points out a planet to us, we would not be able to predict its topography, but we might be able to say something about the likelihood of topography on planets with that mass (and given that astrophysical environment).





%\section{Preliminary results}\label{sec:results}

We have used a simple parameterized convection model to produce thermal histories for stagnant lid planets, which are reasonable to the extent that a thermal history which ignores melting processes can be reasonable. From these results we have estimated dynamic topography as a function of various model parameters.

\subsection{Thermal evolution}

\begin{figure}
  \centering
  \includegraphics[width=0.8\linewidth]{thermal_Mars1}
\caption{Sample thermal evolution for a Mars-like planet. Solid blue lines are results from \citet{Thiriet2019}; solid black lines are from this work using identical parameters, and dashed blue lines are results from \citet{Breuer2010} with dashed black lines from this work using identical parameters, except $\beta = 1/3$ and $a_{rh} = 2.44$, and Arrhenius rheology with grain size tuned such that we get the same reference viscosity and temperature pair as their linear rheology, and no . $T_{avg}$ is the potential temperature averaged across the mantle and lid, $q_{\rm sfc}$ is the surface heat flux, $D_l$ is the stagnant lid thickness, $q_{B}$ is the flux into the bottom of the convecting region, Ur is the Urey ratio, and $T_l$ is the temperature at the base of the lid. TODO: calculate Thiriet Ur given their sfc flux and equation for $H(t)$. TODO: add Nimmo \& McKenzie (1997) for Mars to show no-lid thermal model?}
\label{fig:thermal}
\end{figure}

Firstly, figure \ref{fig:thermal} compares our temperature evolution to results from the nearly-identical stagnant lid parameterized convection model of \citet{Thiriet2019}, as well as to a similar model from adjacent authors \citep{Breuer2010}. The latter considers an additional component of the core-mantle heat flux, the energy from inner core freezing. These models are chosen because we share the most assumptions. For example, some thermal models do not explicitly consider stagnant lid evolution, equivalent to assuming a fixed depth for the convecting region. The temperatures shown here are really potential temperatures. Because we have assumed viscosity is pressure-independent, it is constant along this adiabat.

During the first 1.5 Gyr, the model is adjusting to having initialized well out of equilibrium. The core cools down to the mantle temperature during this relatively quick time frame. The mantle responds by shedding its extra heat quickly; some of this was also imparted by the early violence of its primordial radioisotope stock. The stagnant lid shrinks in response to the high flux of heat coming out of the upper thermal boundary layer; it can reach its mandated high basal temperature in less distance. The Urey ratio of the planet drops steadily: at first internal radiogenic heating dominates, and the fluxes still have a strong memory of their initial conditions. Once settled, the Urey ratio is in sync with the boundary layer heat flux, as seen through the surface heat flux ($q_{u}$ is adjusting to bring the whole planet towards thermal equilibrium). As the radioisotope decay rate gradually, naturally declines, the surface cooling lags, keeping Ur to a vaguely asymptotic value of 0.66. This is close to the classical numerically-modelled value of Ur $\sim$0.7 \citep{Schubert1980, McKenzie1981}.

The discrepancy between our values and those of \citet{Thiriet2019} are explained by the main difference in our model, the assumption of a steady-state conductive temperature profile in the lid; otherwise a PDE is needed. The RMS error between our surface heat flow and \citet{Thiriet2019} is $\pm$2.69 mW m$^{-2}$, which is within the \textless~4 mW m$^{-2}$ they mention that will arise from assuming steady-state conduction in the lid. The steady-state lid shrinking and surface flux increase happen sooner than they should because the temperature profile is allowed to shift instantly. Note that this also affects the average temperature, which includes the lid temperature profile. We only show results for a Mars-like planet. We also produced the same Moon and Mercury scenarios as \citet{Thiriet2019}, and were able to match theirs more or less equally well.  

However, especially if compared directly to ``more complete" models which include melting, it is clear that these mantle temperatures are several hundred Kelvin higher than desired. This points to a missing flux of heat out of the interior. Melting and its transport upwards may actually be the most effective way for Venus' mantle to shed heat \citep{Armann2012}. Without an additional heat loss mechanism such as this, the stabilizing feedback effect of temperature-dependent viscosity make it very hard to reduce the mantle temperature. 

\begin{itemize}
\item Useful to compare to Driscoll no melting case, which has no stagnant lid though? compare to e.g. 1 M$_E$ figure 6 in \citet{Kite2009} to demonstrate melting?
\end{itemize}



%\begin{figure}%{wrapfigure}{r}{0.5\textwidth}
%  \centering
%  \includegraphics[scale=0.5]{Mars1_q_error}
%\caption{Error from assuming steady-state conduction in the lid, compared to the PDE solution in \citet{Thiriet2019}}
%\label{fig:q_sfc_error}
%\end{figure}

\begin{figure}
  \centering
  \includegraphics[width=1\linewidth]{h_comparison}
\caption{Variation of the root-mean-square of dynamic topography amplitude $\Delta h$ with Rayleigh number. The thick black and grey lines are based on our thermal history model Venus case, using the \citet{Parsons1983} scaling (equation \ref{eq:PD83}) and the classical stress scaling (equation \ref{eq:dyn_top_stress}), respectively. The prefactor in the stress scaling is taken to be equal to 2. The pink circles show the minimum and maximum values (of a time-evolution) from \citet{Kiefer1998}, the magenta dashed line is the log-log fit from a numerical model (equation \ref{eq:KH92}) in \citet{Kiefer1992}, and the yellow stars are the inferred dynamic topography from a number of individual features by \citet{Nimmo1996}. The lilac shapes are our RMS estimates from the power spectra of \citet{Golle2012} for their instantaeous viscous flow case, \citet{Huang2013}, and \citet{Yang2016}. Asterisks denote that the reported peak topography has been reduced by 0.707 to approximate an RMS.}
\label{fig:RMS_benchmark}
\end{figure}


\subsection{Purely-dynamic topography for solar system analogues}

We can use the thermal history results from the previous section in simple dynamic topography scaling relationships. The Rayleigh number is the most representative parameter of the topographic amplitudes \citep[e.g.,][]{Kiefer1992}. In figure \ref{fig:RMS_benchmark}, we show several scalings from our model in comparison with numerical calculations from the literature (Table \ref{tab:dyn_topo_obvs}), focusing on studies of Venus---these models were broadly tuned to match certian observed features. Overall---as explained in the introduction---different modelling choices complicate one-to-one intercomparison. %it is tricky to intercompare different dynamic topography models, due to a hidden lack of agreement on what constitutes ``dynamic topography"; i.e., whether one includes the thermal isostasy inherent to temperature perturbations by convection \citep{Molnar2015}. 
\citet{Nimmo1996} use the topography model of \citet{Parsons1983}, whose scaling law we employ.


The choice of model geometry and heating mode can also double or halve $\Delta h$ \citep[e.g.,][figure ]{Kiefer1992}. The authors of \citet{Kiefer1992} imply that their predictions in cylindrical geometry are higher than those of \citep{Parsons1983} in Cartesian geometry, although \citet{Nimmo1996} demonstrate the latter approach in axisymmetric geometry (albeit with fluid ``in a box"). 

To a lesser extent, we must also differentiate between ``peak topographic uplift," which may be associated with a single mantle plume/topographic feature---and which forms the bulk of Venus studies that used observed topography as an observable for constraining the interior---versus root-mean-square amplitude, which is more representative of the entire planet (and should be related to the ``average mantle" Rayleigh number. As a very crude approximation, models that report a ``peak uplift" are scaled by 0.707 (the RMS of a sinusoidal signal); these are marked by an asterisk in figure \ref{fig:RMS_benchmark}. On the other hand, when only a global spherical harmonic power spectrum is reported, we calculate the RMS using $\Sigma_l [S(l)/(2l + 1)]^{1/2}$, where $l$ is spherical harmonic degree and $S(l)$ is the power at that degree.

As is obvious from figure \ref{fig:RMS_benchmark}, our calculations are much lower in terms of absolute dynamic topography than what is inferred for Venus. However, our values are comparable when adjusted for our very high Rayleigh numbers. The Rayleigh number of the ``Venusian average" upper mantle is expected to be more like $10^6$ (REF). Our values of $10^8$ are carried up by our low viscosities, which in turn come from the high temperatures our thermal history produces---Venus' upper mantle is usually said to have $\eta\sim 10^{21}$--$10^{22}$, while ours are a few orders of magnitude lower \citep{Benesova2012}. To a lesser extent, Ra is also held aloft by our low temperature contrast across the convecting region. Because the core cools down to the mantle temperature so quickly, $\Delta T_m \sim \Delta T_{rh}$, about 300~K \citet[cf. 1000 K assumed by]{Kiefer1992}.



\subsubsection{Variation of dynamic topography with planet mass, age, and radiogenic abundance}

We are mostly interested in how $\Delta h$ scales with the bulk properties of the planet. We can see this by looking at how topography behaves relative to each model's reference value, regardless of the spread on absolute topography between different choices of models (figure \ref{fig:RMS_v_planet}). The input parameters $M_p$ and $H_{0}$, as well as the planet age, represent some of those parameters within the realm of possibility of constraining observationally for increasingly smaller exoplanets. How they alter dynamic topography follows what we expect based on their relationships with Ra.

 Increasing a planet's mass increases Ra via hotter ambient temperature (lower $\eta_m$), higher gravity, and a deeperconvecting region. Increased radiogenic heating at any time in a planet's history also means that the mantle is running hotter with a higher Ra, all else being constant. An older planet will generally be slightly cooler.As Ra increases, the planform of convection changes: upwellings and downwellings are more narrow, as is the thermal boundary layer. Spectrally, less-broad upwellings means less long-wavelength power in the boundary layer topography (more short-wavelength power, which is supported elastically). If we consider density contrasts within the boundary layer providing isostatic support, then a thinner boundary layer can provide less support \citet{Kiefer1992}.
 
As per the classic scaling, $\Delta h \propto \Delta T_{rh} \delta_{u}$, the main analytic effect of high Ra is to decrease $\delta_{u}$, since $\delta_{u} \propto {\rm Ra}^{-1/3}$, which overshadows the hot mantle's elevated $\Delta T_{rh}$. The topography of the surface essentially reflects the topography and thickness of the boundary layer, reflecting Ra in turn.



\begin{figure}
  \centering
  \includegraphics[width=1\linewidth]{relative_h}
\caption{Dependence of model dynamic topography, $\Delta h_{\rm RMS}$, on planet age (\textit{left}), mass (\textit{centre}), and radiogenic heating rate at 4.5 Gyr ($H_{4.5}$; \textit{right}) for different dynamic topography scalings: \citet{Parsons1983} (equation \ref{eq:PD83}; red line), \citet{Kiefer1992} (equation \ref{eq:KH92}; dark green line), and the basic scaling of $\Delta h$ with stress (equation \ref{eq:dyn_top_stress}; yellow line).  The prefactor in (\ref{eq:dyn_top_stress}) is taken to be equal to 2. All input parameters other than that varied on the $x$ axis are held constant. In the centre and right subplots, the yellow and red lines nearly coincide.}
\label{fig:RMS_v_planet}
\end{figure}



%\section{Discussion}

\subsection{How does a planet's rheology govern its thermal history?}

\begin{itemize}

\item Changing the rheology of the planet sends it down very different temperature paths
\item But rheological parameters hardly affect convective stresses \citep{Reese2005}, and have an insignificant effect on the ```evolved" / steady-state dynamic topography. This is due to the competing effects of $\eta_m$ and $q_{\rm ubl}$ in equation (\ref{eq:RMS}): for example, increasing $E_a$ means a more sensitive viscosity that is higher to start, but this makes it worse at convecting away heat, and $q_{\rm ubl}$ is lower.
\item boundary layer flux is a regulation mechanism in that stress, topography, viscosity etc are all anti-correlated with it... the system wants to reach a state where the bl flux and sfc flux are equal, the lid grows or shrinks to minimize their difference
\item Not all these relationships are obvious without numerically solving the governing ODEs because the lag in stagnant lid growth induces hysteresis...
\item How dynamic viscosity is parameterized/linearized matters as well---depending on which $T_0, \eta_0$ pair we choose we can get very different evolutions even for the same activation energy, so need to be careful.
\end{itemize}

\subsection{Under what conditions is simple dynamic topography useful/relevant?}


\begin{itemize}
\item Prefer thick lid (how thick?)...
\item Much trickier for Earth because of continents/oceans, dynamic topography is negligible compared to other processes creating topography, don't really care about it (I guess some people in Bullard do)
\end{itemize}
%\section{Future work}

\subsection{Importance of melting on thermal history}

As we have mentioned, modelling melting processes is necessary to obtain realistic thermal histories \citep{Nakagawa2012}. \citet{Armann2012} argue that ``heat pipe" magmatism really is the dominant mode of heat loss for Venus. Because in this case heat loss is not controlled by the thermal boundary layer flux, one cannot really apply classical stagnant lid parameterized convection. While \citet{Kite2009} give a smaller ratio of magmatic to conductive heat loss of 0.1 for a Venus-like planet, the net effect on thermal evolution is still important, even more so for younger worlds. One-to-one comparisons of Venus thermal models with and without melting can be found in \citet{Driscoll2014}, for example, showing mantle temperatures up to 500 K lower if melting occurs, assuming 100\% of it reaches the surface extrusively. How much magmatism is extrusive further affects mantle cooling---whereas intrusive magmatism thins the lithosphere, extrusive magmatism thickens it, insulating the mantle and slowing its heat loss \citep{Lourenco2018}.

Melt generation has cascading effects on planetary interior evolution, namely dehydration stiffening and compositional buoyancy \citep{Korenaga2009}. Melting extracts water, which raises viscosity, and in a hot mantle that melts deeper, this can eventually lead to viscosity stratifications that limit surface heat flux---in contrast to what we expect from the temperature-dependent viscosity feedback. Compositional buoyancy follows from the melt extraction of light elements, creating density stratifications that effectively act as temperature contrasts in the context of changing Ra. Overall, these effects combine such that mantle melting reduces surface heat flow by 5--10\%. \citeauthor{Korenaga2009}'s model \citeyear{Korenaga2009} implementing melting in parameterized convection, along with several others \citep{Kite2009, Driscoll2014, Tosi2017, Foley2018}, give a good basis for approaching this additional complexitiy. 


\subsection{Plate flexure and lithospheric strength models} \label{sec:future-elastic}

So far the dynamic topography models we use fall under the ``instantaneous viscous flow" approximation: pretending the planet reacts to stresses instantaneously and permanently. In reality, the planet has a non-instantaneous responses to stress via viscous relaxation and elastic filtering; any load on its surface is transient. Uplift can be filtered by the lithosphere, which behaves essentially like a thin elastic shell. The amount of filtering thus depends on the thickness of the lithosphere that behaves elastically, the primary control of which is temperature \citep{Watts2001}. The more-accurate planetary deformation model would couple flow in the viscous interior with an elastic shell of time-variable thickness (computed in multiple dimensions, e.g., using the local Maxwell time) \citep[e.g.,][]{Dumoulin2013}. 

Although elastic flexure is usually thought of as only being relevant in the support topography at short wavelengths, several recent studies have pointed out that the elastic properties can indeed affect long-wavelength support, and the instantaneous viscous flow approximation will overestimate dynamic topography \citet{Zhong2002, Golle2012, Dumoulin2013}. Models show an elastically-filtered topography lower by up to 10\% for Venus-like (elastic thickness 46 km), and much larger for Mars with its thicker elastic lithosphere. It would take $\sim$10 Gyr (i.e., never) for the instantaneous case to be achieved \citep{Zhong2002, Dumoulin2013}. 

These concerns will be investigated after the time of report writing. Modelling frameworks of coupling numerical mantle convection in spherical geometry with thin elastic shells (of variable thickness) for stagnant lid planets exist in e.g. \citet{Zhong2002, Beuthe2008, Golle2012, Dumoulin2013, Patocka2017}, although it remains to be seen how accurate these could be for parameterized convection.





\subsection{More exotic planets} \label{sec:future-exoticplanets}

So far we have assumed an Earth-like composition in our use of thermodynamic and rheological properties for (dry) olivine \citep{Karato1993}. In reality, we aren't even sure if this can apply to Venus. For planets around other stars, we expect variability in \textit{(i)} the composition of their protoplanetary disk between stars, and hence the composition of planets forming out of it \citep{Bitsch2020}; as well as \textit{(ii)} the radial distribution, for a given star, in the solidified minerals available to build planets (e.g., Fe, Mg, Si, Ca, Al, and Na minerals), for which modelling is improving \citep{Miyazaki2020}. Considering this last point, exoplanets present some extreme end-members: close-in massive rocky planets such as HD 219134 b, 55 Cancri e, and WASP-47 e may be coreless and rich in minerals containing the highest-temperature condensates Ca and Al, making them 10\textendash 20\% less dense than Earth---with yet-unknown consequences for interior dynamics \citep{Dorn2019}. %As for composition's effects on convection, the Mg/Si ratio influences the postperovskite phase transition deep in the mantles of massive rocky planets, for example \citep{Umemoto2017}. 
We have not yet touched on the effects that a higher mantle water content would have on rheology and rock strength.

Capturing this diversity clearly presents a huge gap in our ability to model exoplanet interiors. However, there may be some things we can account for more methodologically. Heat-producing elements K, U, and Th are among these planet building blocks whose abundances vary according to condensation history in the protoplanetary disk. We might expect all main sequence stars to produce the same isotope ratios for a given element if they obey the same rules of nucleosynthesis. Yet the absolute abundance of K, U, and Th could vary. Although the actual material accreted by a planet depends on its stochastic formation history, we can look to stellar catalogues to understand the inter-system variation. Th and U are refractory elements, unlike K, so we expect planetary abundances to follow stellar abundances somewhat. One study found a two-fold variation in the logarithm of Th/H abundances among solar-twin stars \citep{Unterborn2015}. \citet{Frank2014} use a galactic chemical evolution model to predict variations of heat-producing element abundances, specifically with respect to the date of star formation within the galaxy: the later a star forms, the hotter its planets are radiogenically. On the other hand, \citet{Wagner2012} argue that the more intense radiogenic heating in young planets does not play an important role in determining their interior structures. A better statistical treatment considering the prior distribution of radiogenic heating rates may be within reach.


Tidal-locking is another interesting aspect of rocky exopolanets we do not see in the solar system---our observation biases mean that many currently-known small planets are probably tidally locked because they are so close to their star. Depending on the heat circulation efficiency in the atmosphere, such planets would have extreme hemispherical equilibrium temperature contrasts, possibly with strange convective patterns (e.g., a day-side magma ocean?) (ref). How might topography behave on these bodies? 





\subsection{Observables}

The elephant in the room is the impossibility of using measurements to test our model. Interior properties of exoplanets will probably never be accessible to proper constraints. As just one example, \citet{Schaefer2017} use thermodynamic models to predict element partitioning between core and mantle, and the resulting interior structure signals in bulk density. They conclude that while these signals could not be detected from planet mass and radius alone, the different resulting mantle compositions would nevertheless alter properties such as mantle rheology that are so important in models. A priority is generalizing models as much as possible to get an accurate sense of change, rather than try to make deterministic predictions. A statistical answer is in line with the large sample size of planets in space. We would still like to figure out which tiny details do matter in order to know our enemy.

A handful of astrophysical methods have been studied to this end, although for now they are hypotheses. \citet{McTier2018} postulate a way to extract ``bumpiness" from the light-curves of planets that transit their star, but for the smaller topographic amplitudes expected for massive planets, the signal would probably be undetectable. Exo-cartography, or, solving the inverse problem of 2D albedo distributions from light-curves, could in theory discriminate between land and ocean surface coverage for even Earth-sized planets \citep{Cowan2018, Farr2018, Kawahara2020, Aizawa2020}. This would require next-generation space observatories\footnote{https://asd.gsfc.nasa.gov/luvoir/}\footnote{https://www.jpl.nasa.gov/habex/} equipped with coronagraphs or starshades to be funded, built, and launched.

\begin{itemize}
\item How to justify that the project can could say anything useful about real life
%\item cool final point
\end{itemize} 


\bibliography{../exogeodynamics}


%\setcounter{equation}{0}
%\renewcommand{\theequation}{A\arabic{equation}}
%\setcounter{table}{0}
%\renewcommand{\thetable}{A\arabic{table}}
%\begin{appendices}
%\section{A note on terminology}
In the exoplanet literature, ``Earth-like" planets are usually classed as such according to their planetary radius (or mass) and orbital period (or equivalently, distance from star) \citep{Guimond2018}. This is because these are the first (and for now, only) observables we get for small planets. However, even a planet with mass and period precisely measured at near Earth-like values cannot be expected to remind us of home \citep{Tasker2017}. \citet{Moore2017} caution us against propagating the terms ``super-Earth" and ``habitable zone," associated with a planet's radius and period respectively, as they eschew scientific emotional distance with what is essentially clickbait:
\begin{quote}
We could just as well call a two-Earth-mass exoplanet a super-Venus as a super-Earth, but it is abundantly clear that those terms do not mean the same thing to most audiences and that they both imply vastly more than is known about any exoplanet ... Although metaphorically rich and easy to digest, such terms come with a lot of baggage that is dangerous to both the public perception of the work being done by the community and to the clarity of thought required to advance the field along several disciplinary fronts.
\end{quote}

With this in mind, we try to avoid the term ``super-Earth" and temporarily call them massive rocky planets. 
%\section{Model description}

\begin{table}
\centering
\caption{Parameters used in this study. \label{tab:params}}
\footnotesize
\begin{tabular}{@{} c l r l p{4cm} @{}}
%\multicolumn{5}{l}{\textbf{Constant values for all planets}} \\
\toprule
Symbol & Description & Value & Units & Ref. \\
\midrule
\multicolumn{5}{c}{\textbf{Constant bulk properties}} \\
$\rho_c$ & Core density & 7200 & kg m$^{-3}$ &  \citet{Thiriet2019}  \\
$c_c$ & Core specific heat (p or v?) & 840 & J kg$^{-1}$ K$^{-1}$  & \citet{Thiriet2019}  \\
$k_m$ & Mantle thermal conductivity & 4 & W m$^{-1}$ K$^{-1}$  & \citet{Thiriet2019}  \\
$\alpha_m$ & Thermal expansivity &  $2.5 \times 10^{-5}$ & K$^{-1}$  & \citet{Thiriet2019}  \\
$X_{\rm K}$ & Initial K abundance &  305 & wt ppm  & \citet{Jaupart2015} \\
$X_{\rm U}$ & Initial U abundance &  $16 \times 10^{-3}$ & wt ppm  & \citet{Jaupart2015} \\
$X_{\rm Th}$ & Initial Th abundance &  $56 \times 10^{-3}$ & wt ppm  & \citet{Jaupart2015} \\
$H_{4.5}$ & Radiogenic heating rate at 4.5 Gyr & $4.6\times 10^{-12}$ & W kg$^{-1}$ & \citet{Jaupart2015} \\
Ra$_{\rm crit, u}$ & Critical Rayleigh number & 450 &  & \citet{Thiriet2019}  \\
$E_a$ & Viscosity activation energy & 300 & kJ mol$^{-1}$ & \citet{Karato1993} \\
$A_{rh}$ & Viscosity preexponential factor & 8.7e15 & & \citet{Karato1993} \\
$h_{rh}$ & Grain size & 2.07 & mm & \\
$B$ & Burgers vector & 0.5 & nm & \citet{Karato1993} \\
$m$ & Grain size exponent & 2.5 & & \citet{Karato1993} \\
$\mu$ & Shear modulus & 80 & GPa & \citet{Karato1993} \\
CMF & Core mass fraction & 0.3 & & \\
$L_*$ & Stellar luminosity & 1 & $L_{\rm Sun}$ &  \\
Al & Planetary geometric albedo & 0 & &  \\

\midrule
\multicolumn{5}{c}{\textbf{Solar system models}} \\
$M_p$ & Planet mass &  \makecell[tr]{\textbf{Mars:} 0.11 \\ \textbf{Venus:} 0.82} & \makecell[tl]{$M_\oplus$ \\ $M_\oplus$ } &  \\
$R_c$ & Core radius &  \makecell[tr]{\textbf{Mars:} 1700 \\ \textbf{Venus:} 3330} & \makecell[tl]{km \\ km } &  \makecell[tl]{\citet{Thiriet2019} \\ \citet{Huang2013}} \\
$\rho_m$ & Mantle density & \makecell[tr]{\textbf{Mars:} 3500 \\ \textbf{Venus:} 3300} & \makecell[tl]{kg m$^{-3}$ \\ kg m$^{-3}$} &   \makecell[tl]{\citet{Thiriet2019} \\ \citet{Nimmo1996}} \\
$c_m$ & Mantle specific heat (p or v?) & \makecell[tr]{\textbf{Mars:} 1142 \\ \textbf{Venus:} 1200} & \makecell[tl]{J kg$^{-1}$ K$^{-1}$ \\ J kg$^{-1}$ K$^{-1}$}  & \makecell[tl]{\citet{Thiriet2019} \\  \citet{Nimmo1996}}  \\
$T_s$ & Surface temperature & \makecell[tr]{\textbf{Mars:} 250 \\ \textbf{Venus:} 730} & \makecell[tl]{K \\ K} \\

\midrule
\multicolumn{5}{c}{\textbf{Initial conditions}} \\
$T_{m,0}$ & Initial mantle temperature & 1750 & K & \citet{Thiriet2019}\\
$T_{c,0}$ & Initial core temperature & 2250 & K & \citet{Thiriet2019} \\
$D_{l,0}$ & Initial lid thickness & 300 & km & \citet{Thiriet2019}\\
\bottomrule
\end{tabular}



%
%\begin{tabular}{@{} c l r l  @{}}
%\multicolumn{4}{l}{\textbf{Exoplanet parameter study}} \\
%\toprule
%Symbol & Description & Range & Units \\
%\midrule
%$M_p$ & Planet mass & Range & kg \\
%CMF & Core mass fraction & Range &  \\
%$E_a$ & Viscosity activation energy & Range & kJ mol$^{-1}$ \\
%$h$ & Grain size & Range & mm \\
%\bottomrule
%\end{tabular}

\end{table}







We start with a simple model of thermal history based on parameterized convection. From this we can estimate dynamic topography directly using scaling relationships in the literature.

Table \ref{tab:params} lists the parameters used in this work. The key free parameters we are interested in tuning are the planet mass, $M_p$, core mass fraction, CMF, viscosity activation energy $E_a$ and prefactor, and initial mantle and core temperatures $T_m$ and $T_c$. The initial temperature contains information about the formation history of the planet, namely the leftover gravitational energy of accretion and core segregation. Although $\alpha$, $\rho_m$, and $\kappa_m$ vary with depth \citep{Anderson1987}, geophysical observables are most sensitive to the upper mantle value \citep{Kiefer1992}, which is what we use here.

From the input parameters in Table \ref{tab:params} we get the derived bulk properties: the radius of the planet, $R_p$, based on \citet{Zeng2016},
\begin{equation}
\frac{R_p}{R_E} = (1.07 - 0.21\; {\rm CMF})\left(\frac{M_p}{M_E}\right)^{1/3.7},
\end{equation}
which has surface area SA$_p$; the radius of the core, using the scaling relationship from \citet{Zeng2017},
\begin{equation}
R_c = R_p \; {\rm CMF}^{0.5},
\end{equation}
which has surface area SA$_c$; the surface gravity, $g_{\rm sfc} = 6.674\times 10^{-11}M_p/R_p^2$; and the thermal diffusivity of the mantle, $\kappa_m = k_m/(\rho_m c_m)$. For our Mars and Venus models, we use the observed $R_p$ and $M_p$ which are known fairly precisely, along with $R_c$ assumptions from the literature. 


We can also find the surface temperature, $T_s$, assuming the surface is in blackbody equilibrium,
\begin{align}
T_s &= \left(\frac{q_* \pi R_p^2}{\sigma_{\rm SB} \; {\rm SA}_p}\right)^{1/4},\\
q_* &= \frac{L_*(1-{\rm Al})}{4 \pi a^2};
\end{align}
where $q_*$ is the incident stellar radiation in W m$^{-2}$, Al is the geometric albedo, $L_*$ is the stellar luminosity, and $\sigma_{\rm SB}$ is the Stefan–Boltzmann constant, but for solar system planets we take the measured $T_s$ (see Table \ref{tab:params}). $T_s$ turns out to have no effect on $\Delta h$.




\subsection{Temperature-dependent viscosity, stagnant lid convection}\label{sec:viscosity-model}

Essentially we want to solve equation (\ref{eq:T_ODE}), which we do using the explicit Runge-Kutta method of order 5(4).
We use a temperature-dependent Arrhenius rheology law \citep{Karato1993} as a first step with parameters in Table (\ref{tab:params}), as this helps us understand the physical basis of tweaking rheology. We choose $h_{rh}$ to match the linearized rheology of \citet{Thiriet2019}.

\subsubsection{Heat fluxes}
Heat fluxes are calculated using the common scalings from parameterized convection models. Throughout, heat fluxes per unit volume or area are given by lowercase $q$, and the planet-integrated value in uppercase $Q$. The radiogenic heat flux in W kg$^{-1}$ is:
\begin{align}
q_{\rm rad} &= H_0\sum^{\rm K, U, Th}_n \left[ h_{n}  e^{\lambda_n (t_f - t)} \right], \nonumber \\
h_n &= \frac{c_n p_n}{\sum c_n p_n}, \\
c_n &= \frac{n_nX_n}{n_{\rm U} X_{\rm U}} \nonumber ,
\end{align}
where we are summing over the heat-producing elements K, U, and Th, $H_{0}$ is the present-day heat production in W kg$^{-1}$, $h_n$ is the relative contribution of the $n^{th}$ isotope, $c_n$ is radiogenic element abundance relative to U, $n_n$ is the natural isotopic abundance in terms of mass, $X_n$ is the bulk planet abundance in terms of mass, $p_n$ is the heating power in W kg$^{-1}$, $t_f$ is the present-day age, and $\lambda_n$ is the decay constant in s$^{-1}$ \citep{Korenaga2006}. We use the canonical values for decay constants and isotopic natural abundances \citep{Jaupart2015}.

Heat travels by conduction through the upper and lower thermal boundary layers, so these fluxes depend on thermal conductivity, $k_m$, and the boundary layer thickness, $\delta$, with subscripts $u$ and $c$ referring to the upper and lower layers respectively: 
\begin{align}
q_{u, c} &= k_m \frac{\Delta T_{u, c}}{\delta_{u, c}} \label{eq:q_u}\\
\delta_{u, c} &= d \left(\frac{{\rm Ra}_{{\rm crit}, u, c}}{{\rm Ra}_{rh, u, c}}\right)^\beta,\label{eq:d_u}
\end{align}
where we equate Ra$_{rh}$ with the interior Rayleigh number from equation \ref{eq:Ra}. We assume $\beta = 1/3$, so convecting layer depth $h$ cancels out.\footnote{Numerical studies give $\beta$ around 0.3 \citep{Thiriet2019}.} For $\Delta T_u$ we use $T_c - T_l$, and $\Delta T_c = T_c - T_m$. $T_l$ is the temperature at the top of the convecting region. Viscosity is taken at the isothermal middle for the upper boundary layer, and at $T_c + T_m/2$ for the lower, after \citet{Thiriet2019}. Gravity is evaluated at the surface for $\delta_u$ and at the core radius for $\delta_c$. The critical Rayleigh numbers are Ra$_{{\rm crit}, u}$ = Ra$_{{\rm crit}}$, and Ra$_{{\rm crit}, c}$ = 0.28Ra$_{i}^{0.21}$, where Ra$_{i}$ is the interior Rayleigh number as in equation (\ref{eq:Ra}) \citep{Thiriet2019}.




\subsubsection{Temperatures and heat flow in the lid}

We distinguish between the upper thermal boundary layer of the convecting region and the stagnant lid. Specifically, the temperature at the top of the mantle thermal boundary layer, $T_l$, would not be set at the planetary surface temperature. Conceptually, the ``lid" in a stagnant lid refers to a mechanical transition, and is dynamically decoupled from the convecting region. We are not distinguishing petrological (compositional) layers at this point, and so we avoid referring to our lid as a ``lithosphere," although they share some roles.

Fluid dynamics experiments show that the temperature jump between the isothermal core of the convecting region at $T_m$ and the top of the thermal boundary layer at $T_l$ is proportional to the viscous temperature scale $\Delta T_{rh}$: the rate of viscosity change with temperature \citep{Davaille1993},
\begin{align}
\label{eq:Tl}
T_{l} &= T_m - a_{rh} \Delta T_{rh} \\
\Delta T_{rh} &= \frac{\eta(T_m)}{{\rm d}\eta/{\rm d}T_m\vert_{Tm}} = \frac{R_b T_m^2}{E_a},
\end{align}
where $R_b$ is the universal gas constant, and $a_{rh} = 2.44$ for $\beta = 1/3$ based on fits to 3D convection models \citep{Thiriet2019}. (\ref{eq:Tl} has the same form regardless of Arrhenius or linear rheology law. 



In the stagnant lid of thickness $D_l$ above this layer, the temperature decreases from $T_l$ to $T_s$ by conductive heat transport, which in spherical coordinates is:
\begin{align}
T_{\rm lid}(r) &= \frac{-a_0}{6k_l} r^2 + \frac{c_1}{k_m r} + c_2 \\
    c_1 &= k_m \frac{T_l - T_s - a_0/(6 k_m) \left(R_p^2 - R_l^2\right)}{1/R_l - 1/R_p} \\
    c_2 &= T_s + \frac{a_0}{6 k_m} R_p^2 - \frac{c_1}{k_m R_p}
\end{align} 
where $k_l = k_m$ is thermal conductivity, $a_0 = \rho H_0$ is the lithospheric heat production in W m$^{-3}$, assumed to be constant with $r$. For now we are assuming radiogenic heating in the lid is equivalent to in the interior; we might anticipate more radiogenic heating in the lid because it overlaps with the lithophilic element-rich crust in a real planet.

Evaluating the associated conductive heat flux, $-k_l \d T/\d r$, at $r = R_p$ gives the surface heat flux:
\begin{align}
q_{\rm lid} &= -\frac{a_0}{3k_m}r - \frac{c_1}{k_m r^2},\\
q_{s} &= -\frac{a_0}{3k_m}R_p - \frac{c_1}{k_m R_p^2}.
\end{align}



\subsubsection{Lid thickness}

%As a zeroth order case, we can assume a constant lid thickness over the planet's evolution.
%
%At intermediate complexity, we assume flux is continuous where the upper thermal boundary layer meets the stagnant lid:
%\begin{align}
%\label{eq:t-continuity}
%&T_{\rm lid}(z=-d_l) = T_m + \Delta T_\eta \\
%&\frac{H_l}{2k}d_l^2 + \frac{q_{ubl}\left(T_m\right)}{k}d_l + \left(T_m - T_s - \frac{R_b}{E_a}T_m^2\right)  = 0
%\end{align}
%where $q_{ubl}$ is a function of only $T_m$ if $\beta=1/3$ ($\delta_{ubl}$ independent of mantle depth). The roots of this equation give $d_l$ as a function of $T_m(t)$ and $H_l(t)$, which allows $d_l$ to be calculated at each time-step. It can be shown that there is one positive root for our range of parameters.


We account for the fact that the lid does not instantly grow or shrink in response to a change in $q_{u}$. There is some lag where $D_l$ adjusts such that the flux out of the top of the lid is moving towards equilibrium with the flux into the base of the lid \citep{Thiriet2019}:
\begin{equation}\label{eq:D_l}
\frac{\d D_l}{\d t} = \frac{q_{\rm lid}\vert_{R_l} -q_u}{\rho_m c_{m} (T_m - T_l)} . 
\end{equation}
From this we can calculate $R_l = R_p - D_l$; we also account for the mass of the convecting region $M_m$ changing with $R_l$. This ignores the time-dependence of heat conduction in the lid (i.e., requiring a P.D.E.), which is associated with an error of a few mW m$^{-2}$ \citep{Thiriet2019}.




\subsection{Height of dynamic topography}

We consider three scaling laws to estimate the root-mean-square (RMS) amplitude of dynamic topography. The first is written using equation (\ref{eq:sigma_conv}). Boundary layer theory predicts that convective stress scales as \citep{Solomatov1995}:
\begin{align}
\tau_{zz} \propto \rho_m g \alpha_m \Delta T_{rh} \delta_u 
\end{align}
where $\alpha_m$ is the mantle thermal expansivity (the volume change due to a temperature change), and $\Delta T_{rh}$ and $\delta_u$ are the temperature difference across and thickness of the upper thermal boundary layer from (\ref{eq:Tl}) and (\ref{eq:d_u}). \citet{Reese2005} give proportionality constant $C_1 = 0.1$ for interior shear stress and $C = 2$ for lid shear stress.

\begin{itemize}
\item isn't it only the normal component of stress that you care about for dynamic topography?
\item does the stress scaling include thermal isostasy density contrasts - check solomatov 1995 - i think probably 
\end{itemize}

Thus dynamic topography scales with $T_m$ and $q_{u}$ as:
\begin{equation}\label{eq:dyn_top_stress}
\Delta h \sim C_1 \alpha_m \Delta T_{rh} \delta_u.
\end{equation}
We can also write this analytically in terms of Ra:
\begin{equation}
\Delta h \propto \left(\frac{\rm Ra}{\rm Ra_{\rm crit}}\right)^{-\frac{1}{3}}.
\end{equation}


The second is provided by the cartesian geometry, constant-viscosity case of \citet{Parsons1983} equation (33), with scaling exponent $\beta$ on surface-flux Ra taken to equal 0.5 (their no-internal-heating case):
\begin{equation}\label{eq:PD83}
\Delta h \sim C_2 \left[\frac{\alpha_m q_{s} \eta_m(T) \kappa_m}{\rho_m g k_m}\right]^\frac{1}{2},
\end{equation}
where $C_2 = 5.4$ to match the predictions of RMS dynamic topography in \citet{Lees2019}. This should be equivalent to equation (\ref{eq:dyn_top_stress}); the approximation expressed in the latter comes from the notion that most of the contribution to surface topography is in the displacement of stress boundaries in the thermal boundary layer.

The third scaling comes from fits to cylindrical geometry, constant-viscosity numerical experiments of \citet{Kiefer1992}:
\begin{equation}\label{eq:KH92}
0.7 \Delta h = 66 {\rm Ra}^{-0.121},
\end{equation}
where the factor of 0.7 = $(\rho_m - \rho_w) / \rho_m$ is scaling their water-loaded model to subaerial topography.

Although both (\ref{eq:PD83}) and (\ref{eq:KH92}) are scalings for ``peak topography" over a given rising plume, we naively treat them as the global RMS dynamic topography, $\Delta h_{\rm RMS}$, assuming that the values of $\tau_{zz}$ they are based on reflects the global RMS convective stress.


%\end{appendices}



\end{document}

