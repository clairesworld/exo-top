\section*{Abstract}

Topography is a crucial component of the Earth system: having exposed fresh rock lets surface temperatures self-regulate via silicate weathering. However, there are limits to a lithosphere’s capacity to support mountains or valleys over geologic time. We see in our solar system that the range in a body’s elevations tends to decrease with increasing planet mass. Meanwhile, most currently-known exoplanets are in between the sizes of Earth and Neptune---a regime totally exotic to the solar system---and many have apparent bulk densities consistent with a rocky composition. Our aim is to extrapolate solar system trends in topography to massive rocky exoplanets using well-tested models from geodynamics. This must invoke current work in understanding the limits of terrestrial planet thermal evolution models. Given these models, we can predict the range of dynamically-supported, and eventually, elastically-supported topography. Specifically, we are after how massive an exoplanet can be before it becomes hypsometrically featureless.