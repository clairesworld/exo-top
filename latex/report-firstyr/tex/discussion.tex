\section{Discussion}

\subsection{How does a planet's rheology govern its thermal history?}

\begin{itemize}

\item Changing the rheology of the planet sends it down very different temperature paths
\item But rheological parameters hardly affect convective stresses \citep{Reese2005}, and have an insignificant effect on the ```evolved" / steady-state dynamic topography. This is due to the competing effects of $\eta_m$ and $q_{\rm ubl}$ in equation (\ref{eq:RMS}): for example, increasing $E_a$ means a more sensitive viscosity that is higher to start, but this makes it worse at convecting away heat, and $q_{\rm ubl}$ is lower.
\item boundary layer flux is a regulation mechanism in that stress, topography, viscosity etc are all anti-correlated with it... the system wants to reach a state where the bl flux and sfc flux are equal, the lid grows or shrinks to minimize their difference
\item Not all these relationships are obvious without numerically solving the governing ODEs because the lag in stagnant lid growth induces hysteresis...
\item How dynamic viscosity is parameterized/linearized matters as well---depending on which $T_0, \eta_0$ pair we choose we can get very different evolutions even for the same activation energy, so need to be careful.
\end{itemize}

\subsection{Under what conditions is simple dynamic topography useful/relevant?}


\begin{itemize}
\item Prefer thick lid (how thick?)...
\item Much trickier for Earth because of continents/oceans, dynamic topography is negligible compared to other processes creating topography, don't really care about it (I guess some people in Bullard do)
\end{itemize}