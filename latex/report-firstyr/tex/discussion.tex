\section{Discussion}

\subsection{How does a planet's rheology govern its thermal history?}\label{sec:dis-rheology}

\begin{figure}
\centering
\begin{subfigure}{.5\textwidth}
  \centering
  \includegraphics[width=\linewidth]{Ea-effect-evol}
  \caption{Activation energy}
  \label{fig:Ea}
\end{subfigure}%
\begin{subfigure}{.5\textwidth}
  \centering
  \includegraphics[width=\linewidth]{h-effect-evol}
  \caption{Grain size}
  \label{fig:h_rh} 
\end{subfigure} 

\begin{subfigure}{.5\textwidth}
  \centering
  \includegraphics[width=\linewidth]{a_rh-effect-evol}
  \caption{Rheological temperature scale prefactor}
  \label{fig:a_rh} 
\end{subfigure}
\caption{The effects on thermal evolution of Arrhenius diffusion creep rheological parameters $E_a$ \textit{(a)} and $h_{rh}$ (\textit{b}), and the rheological temperature scale proportionality constant, $a_{rh}$ (\textit{c}). $T_m$ is mantle potential temperature, $\Delta h$ is the topography from the \citet{Parsons1983} model, $q_{u}$ is the flux through the upper thermal boundary layer, $\eta_m$ is the mantle viscosity, $\delta_{u}$ is the upper thermal boundary layer thickness, and Ur is the Urey ratio. Colours denote low-to-high values from violet to red.}
\label{fig:rheology}
\end{figure}


\citet{Urey1955} shows how mantle viscosity could have a stabilizing feedback on temperature. Increasing temperature decreases viscosity, which increases Ra in (\ref{eq:Ra}). Increasing Ra is associated with a thinner thermal boundary layer and more heat flow out of the mantle (\ref{eq:q_Ra}), which in turn cools the mantle and raises its viscosity. This feedback would buffer the interior at a near-constant viscosity, a corollary of which is a narrow range for the mantle viscosities of terrestrial exoplanets. The feedback has not been proven to occur on Earth, and is not guaranteed; its efficiency depends the heat flux out of the mantle being able to adjust quickly to a change in temperature, with respect to the time scale of radiogenic decay \citep{Korenaga2008a}. This may not be the case if the upper mantle is very dehydrated, for example.



This negative feedback does appear to regulate viscosity in our model. Figure \ref{fig:rheology}a,b illustrates some trade-offs between thermal evolution output parameters and the input parameters in the Arrhenius law for viscosity,
\begin{equation}\label{eq:eta_Arrhenius}
\eta_m(T) = b \exp\left(\frac{E_a}{R^* T_m}\right),
\end{equation}
where $R^*$ is the gas constant, $b$ is a prefactor, and $E_a$ is the activation energy in J mol$^{-1}$ \citep{Karato1993}. The prefactor is given by
\begin{equation}\label{eq:eta_b}
b = \frac{\mu}{2 A_{rh}} \left(\frac{h_{rh}}{B}\right)^m,
\end{equation}
where $\mu$ is the shear modulus in Pa, $A$ is a preexponential factor in s$^{-1}$ (of order $10^{15}$ for diffusion creep), $B$ is the Burgers vector in m (quantifying the distortion in the crystal lattice due to dislocation), $h_{rh}$ is the grain size of the rock in m, and $m$ is an exponent characteristic to the type of creep. We treat grain size $h_{rh}$ as a proxy for $b$.

Changing the dependence of viscosity on temperature pushes the planet to different temperature paths (figure \ref{fig:rheology}a,b), which are rapidly amplified, even if $\eta_m$ later converges across evolved runs to within an order of magnitude. The grain size sets how high $\eta_m$ is initially, and the activation energy controls how quickly $\eta_m$ changes with $T_m$. In both cases of either large grain size or high activation energy, the interior starts to lose heat out of the top less efficiently because of initially sluggish convection, and the planet maintains a higher $T_m$ in the quasi-steady state. 

Meanwhile, as predicted by boundary layer theory \citep{Solomatov1995, Solomatov2000, Reese2005}, the viscosity-law parameters (figure \ref{fig:rheology}a,b) have an insignificant effect on the final dynamic topography. This is due to the temperature-viscosity stabilizing feedback. $\Delta h$ essentially reflects $\delta_u$, while changes in $\delta_u$ are largely due to changes in Ra. Increasing $E_a$ or $h_{rh}$ results in an initially lower Ra (by orders of magnitude) via the higher $\eta_m$, an effect which is eventually diminished as $T_m$ and $\eta_m$ evolve. Thus variations in $\delta_u$ and $\Delta h$ are diminished as well. This ultimate effect of stabilizing $\delta_u$ is further helped by the competing effects of $d_m$ and Ra in setting $\delta_u$ (\ref{eq:d_u}): higher viscosity leads to thicker lids and thinner $d_m$.

The upper thermal boundary layer heat flux $q_{u}$ acts as a regulation mechanism in that it is anticorrelated with $\Delta h$ and $\eta_m$. For a given quasi-steady state value of $q_{u}$, the rheological temperature scale $\Delta T_{rh}$ multiplied by a constant $a_{rh}$ sets the temperature contrast across the thermal boundary layer, dictating how thick $\delta_{u}$ can be (equation \ref{eq:q_Ra}). As shown in figure \ref{fig:rheology}c, doubling $a_{rh}$ can double $\Delta h$. Note that the value of $a_{rh}$ in parameterized models can only be constrained by numerical convection models. In contrast to \ref{fig:rheology}a and \ref{fig:rheology}b, in which rheological self-regulation means that there is little leeway on $\delta_{u}$ with the choice of diffusion creep law, \ref{fig:rheology}c shows how increasing $a_{rh}$ does permit $\delta_{u}$ to increase, supporting higher $\Delta h$.



\subsection{Implications for planetary systems science}

\begin{figure}
  \centering
  \includegraphics[width=0.8\linewidth]{ocean_vol}
\caption{The volume of water that will overwhelm a planet's peak dynamic topography, as a function of its age \textit{(left)}, mass \textit{(centre)}, and radiogenic heating rate at the 4.5-Gyr mark \textit{(right)}. Values are scaled to 4.5 Gyr, 1 Venus mass, and 4.6 pW kg$^{-1}$, respectively.}
\label{fig:ocean}
\end{figure}

The effects of topography on the larger workings of the planet are complex and important; indeed, this is a major motivation of our work. As a starting point, we consider that if a planet's peak topography is too low, the entire land area could be submerged more easily by a surface ocean. Figure \ref{fig:ocean} shows the maximum allowable ocean volume before this occurs, approximated as $4\pi/3 \left[(R_p + \Delta h_{peak} )^3 - R_p^3\right]$. The actual volume of surface water depends on the volatile delivery during accretion (a stochastic process) and the cumulative outgassing of the planet (tied strongly to its thermal history). This preliminary calculation shows that massive planets' greater surface area overcomes their lower topography in terms of available surface water storage space. The effect of dynamic topography on subaerial land area may be insignificant for larger planets, especially compared to the effects of water partitioning between the surface and interior \citep{Komacek2016}.



\subsection{Limitations}

\subsubsection{Plate flexure and lithospheric strength models} \label{sec:future-elastic}

So far the dynamic topography models we use fall under the ``instantaneous viscous flow" approximation: assuming the surface of the planet reacts to stresses instantaneously and permanently. In reality, the planet has a non-instantaneous response to stress via viscous relaxation and elastic filtering, and any load on its surface is transient. Topographic uplift is filtered by the lithosphere, which behaves like a thin elastic shell. The amount of filtering depends on the thickness of the part of the lithosphere behaving elastically, the primary control of which is temperature \citep{Watts2001}.

Elastic properties of the lithosphere are usually thought to be relevant only for short-wavelength topography (see section \ref{sec:top_mechs}), Yet several recent studies have shown that elastic filtering can affect long-wavelength topography as well: the instantaneous viscous flow approximation will overestimate long-wavelength dynamic topography by \textless10\% for thin elastic lithospheres like Venus, but can be $\sim$50\% for thick ones \citep{Zhong2002, Golle2012, Dumoulin2013}. Although this may seem like a relatively small error for Venus-like conditions, we do not know how elastically thick planetary lithospheres are $\textit{a priori}$.

{\color{red} say that you can use their calculations of percent difference from IVF for different elastic thicknesses to provide estimate of how much this matters. it's a limitation}

\subsubsection{Interior structure}\label{sec:future-exoticplanets}

So far we have assumed a Mars or Venus-like bulk composition in our choice of thermodynamic and rheological parameters and interior structure. In reality, we expect variation in \textit{(i)} the between-system bulk composition of protostars and protoplanetary disks, and hence the subsequent composition of planets \citep{Bitsch2020}; as well as \textit{(ii)} the radial distribution, for a given star, in the solids that form planets (e.g., Fe-, Mg-, Si-, Ca-, Al-, and Na-minerals) \citep{Dorn2019, Miyazaki2020}. To the first order, this would affect the planet's overall interior structure (that is, the radial distribution of iron, silicate, water, and gas).

In principle, some constraints on exoplanets' interior structures are possible. Astrophysical observations of planet mass and radius can be compared to theoretical predictions of mass-radius relationships based on linear combinations of equations of state for different planetary materials, namely hydrogen gas, water, silicate, and iron \citep{Seager2007, Valencia2007, Rogers2010, Dorn2018}. While radius, mass, and the relative abundances of these four components are degenerate with each other, this degeneracy is somewhat ameliorated through combining mass-radius observations either with stellar abundances of refractory elements, in particular the Fe/Si ratio \citep{Dorn2017a, Brugger2017}; or with estimates of a planet's Love number from light curves \citep{Kellermann2018}. 

Realistically, it is unlikely we will be certain that a known exoplanet does not have a substantial upper layer of H and He, forming a supercritical fluid at depth which mixes to some degree with a silicate layer below. Our convection model would not apply to such a planet.

%Finally, if we believe that a planet has a predominantly rocky composition, then we are able to constrain its relative abundances of Mg/Si and Fe/Si, since these ratios have limited variability across nearby stars \citep{Hinkel2014}. \citet{Dorn2017, Dorn2018a} demonstrate a Bayesian approach to modelling the interior structures of rocky exoplanets using these stellar abundance constraints, which would be useful if we eventually require self-consistent mantle density profiles. More detailed planetary formation models have been used to study the expected breadth of Fe/Mg ratios in rocky planets due to giant impacts \citep{Scora2020}.



\subsubsection{Convection dynamics at higher masses}

Rocky planets more massive than Earth have interior pressures much higher than Earth's. At these higher pressures, postperovskite dissociates into oxides, leading to new phase transitions \citep[up to four, meaning a five-layered mantle;][]{vandenBerg2019} and a less-viscous lower mantle via a pressure-weakening effect \citep{Umemoto2011, Karato2011, Tackley2013, Umemoto2017, Shahnas2018, Ritterbex2018, vandenBerg2019}. For very high pressures, the lowest part of the mantle could melt into a super basal magma ocean \citep{Labrosse2007}. Density and viscosity contrasts due to phase changes in the mantle affect the patterns of convection. For example, there may be one or more high-pressure, high-viscosity middle layers of sluggish convection bounded by low-viscosity convecting zones in the upper and lower mantle. This viscosity stratification reduces surface heat loss, the same as when dehydration causes viscosity stratification (section \ref{sec:melting}). How well we can extrapolate parameterized convection to larger planets with multiple mantle phase transitions remains to be tested. %\citet{Tachinami2014} have found no thermal convection at all for planets \textgreater 5~M$_\oplus$ due to adiabatic compression in the deep mantle, depending on the pressure dependence of thermal expansivity.
{\color{red} say something about how you will probably not work too much on modelling this but it's a potential limitation}



\subsection{Under what conditions is parameterized dynamic topography useful?}

Our preliminary results agree with previous work finding larger dynamic topography at lower Ra and older planet ages \citep{Breuer2015}. Although we have not considered any elastic filtering of topography, thinner elastic lithospheres such as that assumed for Venus also increase the relative importance of dynamic topography at long wavelengths \citep{Golle2012, Dumoulin2013, Breuer2015}.

Meanwhile, on Earth, dynamic topography has neglible effects on the overall hypsometric curve---depending on one's definition, the dynamic component of Earth's topography is not higher than the oceans are deep \citep{Molnar2015, Hoggard2016}. The surface of Venus is thought to be expressing mantle convection, but only in certain areas, and many of its highest-topography regions, such as Ishtar Terra in the north, are consistent with Airy isostasy in terms of their gravity signal \citep{Grimm1991, Yang2016}. So why do we choose to model dynamic topography at all? One reason is that it is requires the lowest-complexity models. Isostasy, conversely, depends on density profiles that require assumptions about petrology. Dynamic topography stems directly from a thermal history model, which would be a required starting point for any geophysical process on exoplanets.

1D parameterized models cannot capture the spatial distribution and possible localization of dynamic topography. The mode of mantle heating affects the planform of convection \citep[e.g.,][]{Choblet2009}, and thus the shape of topographic swells and associated spherical harmonic power spectrum. While internal heating is associated with diffuse upwellings, basal heating leads to broad plumes. Increasing Ra results in narrower upwellings and enhanced localization; topography is higher but more sparse \citep{Bunge1996}. The localization of topographic peaks is important if we are concerned with system-wide consequences that relate to the maximum topography, or how it is distributed on the globe.

%To our knowledge there is no published scaling law for the dynamic topography on a temperature-dependent viscosity, mixed-heating, stagnant lid planet. Although we rely on the present scalings now, we would be interested in learning how wrong they make us.

