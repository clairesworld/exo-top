\section{Preliminary results}\label{sec:results}


We have used simple parameterized convection models to produce thermal histories for stagnant lid planets. From the results of these models, we have estimated dynamic topography as a function of planet age, mass, and initial radiogenic isotope abundance. Parameters are defined in Table \ref{tab:params}.

\subsection{Thermal evolution}\label{sec:thermal}

\begin{figure}
  \centering
  \includegraphics[width=0.8\linewidth]{thermal_Mars1}
\caption{Sample thermal evolution for a Mars-like planet. Solid blue lines are results from \citet{Thiriet2019}; solid black lines are from this model with the same initial conditions, and dashed lilac lines are results from \citet{Breuer2010} with dashed black lines from this model with the same initial conditions. Model parameters are identical between the two literature models and our model. The grain size in our Arrhenius rheology is tuned such that we get the same reference viscosity and temperature pair as the linear rheology in \citet{Thiriet2019} and \citet{Breuer2010}. $T$ is the interior potential temperature (averaged across the mantle and lid for the solid lines; the temperature of the isothermal mantle for the dashed lines), $q_{s}$ is the surface heat flux, $D_l$ is the stagnant lid thickness, $q_{c}$ is the flux into the bottom of the convecting region, Ur is the Urey ratio, Ra is the Rayleigh number based on temperature contrast, $\eta_m$ is the mantle viscosity, and $T_l$ is the temperature at the base of the lid. Coloured lines are only shown when that output parameter is reported in the reference. %TODO: calculate Thiriet Ur given their sfc flux and equation for $H(t)$.%TODO: add Nimmo \& McKenzie (1997) for Mars to show no-lid thermal model?
}
\label{fig:thermal}
\end{figure}

Firstly, figure \ref{fig:thermal} compares our temperature evolution for a Mars analogue planet to results from the nearly-identical stagnant lid parameterized convection model of \citet{Thiriet2019}, as well as to a similar model from adjacent authors \citep{Breuer2010}. Both of these published models differ from ours in that we assume a steady-state conductive temperature profile in the lid, while they model the time-dependence of heat conduction. \citet{Breuer2010} differs from \citet{Thiriet2019} in initial conditions. \citet{Breuer2010} also includes an additional component of the core-mantle heat flux, the energy from inner core freezing (including this flux reduces $q_c$ for the first billion years). 


During the first $\sim$1.5 Gyr of evolution, the temperatures of the mantle and core are adjusting to having initialized far out of equilibrium (in terms of $T_l$, $T_m$, and $D_l$). The core rapidly cools to the mantle temperature during this time frame, while the lid thickness adjusts such that the heat flux into its base equals the heat flux out of its surface. Thermal evolution is largely independent of initial conditions beyond about 1.5 Gyr, seen the increasing similarity betwen runs initialized differently. The Urey ratio of the planet drops steadily: as the radioactive heating rate declines, the surface cooling rate lags behind, and Ur approaches a quasi-asymptotic value of 0.66. This is close to the classical numerically-modelled value of Ur $\sim$0.7 \citep{Schubert1980, McKenzie1981}.


The discrepancy between our values and those of \citet{Thiriet2019} are explained by our simplification of steady-state stagnant lid conduction. The RMS error between our surface heat flow and that of  \citet{Thiriet2019} is $\pm$2.69 mW m$^{-2}$, which is within the \textless~4 mW m$^{-2}$ from assuming steady-state conduction in the lid, according to \citet{Thiriet2019}. The thinning of the lid and increase of the surface heat flux happen sooner in steady state because the temperature profile is allowed to shift instantaneously. Note that this also affects the average temperature (figure \ref{fig:thermal}, upper left panel), which includes the lid temperature profile. We only show results for a Mars-like planet. We also produced the same Moon and Mercury scenarios as \citet{Thiriet2019}, and were able to match the published results equally well.  



Although not currently illustrated here, we can compare our results to thermal models that include melting \citep[e.g.,][]{Hauck2002, Kite2009}. The mantle temperatures in figure \ref{fig:thermal} tend to be higher in our cases because of this cooling mechanism missing in our model. Melt processes may indeed be the most effective way to remove heat from Venus' interior \citep{Armann2012}.

\begin{figure}
  \centering
  \includegraphics[width=1\linewidth]{h_comparison}
\caption{Variation of the RMS dynamic topography amplitude $\Delta h_{\rm RMS}$ with Rayleigh number. The solid black and grey lines are based on our full thermal history model for a Venus-like planet (parameters given in Table \ref{tab:params}), with $\Delta h$ calculated at each time step using the \citet{Parsons1983} scaling (equation \ref{eq:PD83_A}) and the parameterized stress scaling (equation \ref{eq:dyn_top_stress_A}), respectively. The dashed black and grey lines show the same scalings, but as analytical functions of Ra (equations \ref{eq:PD83} and \ref{eq:dyn_top_stress}), with $\Delta T_{rh}$, $\Delta T_m$, and $d_m$ fixed at their final values from the thermal history model. The pink circles show the minimum and maximum values (of a time-dependent model) from \citet{Kiefer1998}, the magenta dashed line is the log-log fit from a numerical model in \citet{Kiefer1992}, given by (\ref{eq:KH92}) and plotted only for the Ra domain they consider, and the yellow stars are the inferred dynamic topography from a number of individual features by \citet{Nimmo1996}. The lilac shapes are our RMS estimates from the power spectra of \citet{Huang2013} and \citet{Yang2016}. \;\; *The reported peak topography has been reduced by 0.707 to approximate an RMS. \;\; $^\dagger$The reported value of the basal-heating Rayleigh number has been converted analytically to the thermal Ra assuming a 1000-K temperature contrast across the convecting cell. \;\; $^\ddagger$The RMS topography has been estimated from the power spectrum.}
\label{fig:RMS_benchmark}
\end{figure}


\subsection{Purely-dynamic topography for solar system analogues}\label{sec:results-comparison}


For Venus-like planets, figure \ref{fig:RMS_benchmark} compares scalings of $\Delta h_{\rm RMS}$ with Ra from our thermal model with numerical predictions of $\Delta h_{\rm RMS}$ from the literature. The literature selection included here is subsample of the studies listed in Table \ref{tab:dyn_topo_obvs}, chosen because they calculate the full dynamic topography from stagnant lid mantle convection using some variation of (\ref{eq:tau_zz}) (as opposed to using an approximation based on thermal isostasy, e.g.).

Several rough adjustments have been made to facilitate comparison. We must differentiate between ``peak topographic uplift," which may be associated with a single mantle plume or topographic feature, versus $\Delta h_{\rm RMS}$, which is more representative of the entire planet. As a very crude approximation, models that report a ``peak uplift" are scaled by 0.707 (the RMS of a sinusoidal signal); these are marked by an asterisk in figure \ref{fig:RMS_benchmark}. When only a global spherical harmonic power spectrum is reported, marked by a double dagger in figure \ref{fig:RMS_benchmark}, we calculate the RMS using $\Sigma_l [S(l)/(2l + 1)]^{1/2}$, where $l$ is spherical harmonic degree and $S(l)$ is the power at that degree. When the basal heating Rayleigh number is reported (along with $q_c$ and $\Delta T_m$), this value is naively converted to thermal Ra assuming a 1000-K temperature contrast across the convecting cell, Ra = Ra$_B (k \Delta T_m)/(q_c d)$, where Ra$_B$ is the basal heating Rayleigh number, noting that an exact comparison cannot be made because these two Rayleigh numbers are based on different assumptions about what is fixed in the model.

It is worth repeating that the dynamic topography predictive models available in the literature are not based on consistent assumptions. These assumptions include: the type of rheological law (temperature-dependent, depth-dependent, or constant), model geometry (cartesian, cylindrical, or spherical), mode of mantle heating (basal or internal), mechanical boundary conditions (free-slip or no-slip), and presence of a stagnant lid, the first three of which are listed in Table \ref{tab:dyn_topo_obvs} for each study. That is, not all variables other than Ra are held constant to plot $\Delta h$. In this way, figure \ref{fig:RMS_benchmark} illustrates some of the scatter in $\Delta h_{\rm RMS}$ associated with varying model assumptions.

All the model predictions of log($\Delta h_{\rm RMS}$) versus log(Ra) scatter around a line of slope $-1/3$ (i.e., the grey and black dashed lines), which is predicted from the scalings in (\ref{eq:dyn_top_stress}) and (\ref{eq:PD83}). An exception is the fit from \citet{Kiefer1992}, which has a shallower slope. This model is the only one that employs cylindrical geometry, which may explain some of the discrepancy. The two models that assume constant viscosity \citep{Nimmo1996, Kiefer1998} do produce overlapping results, despite only \citet{Kiefer1998} considering internal heating. The two RMS data extracted from spherical harmonic power spectra (lilac symbols) surprisingly do not accord with each other, despite being based on very similar models. This is likely due to human error in digitising these spectra (they must be digitised by hand accurately, and we cannot rely on interpolation when extracting the power at each degree). Finally, we have introduced a discrepancy by trying to combine multiple approaches in retrieving the RMS value of topography---from either a peak $\Delta h$ over a plume \citep{Kiefer1992, Moresi1995, Nimmo1996} or from the power spectrum \citep{Huang2013, Yang2016}. Both of these (preliminary) approaches are erroneous for different reasons. Only \citet{Kiefer1998} report a bona fide global RMS value for $\Delta h$.



Topography based on thermal history models (solid lines) is non-identical to topography based on the equivalent scalings with Ra (dashed lines) because of hysteresis in the thermal models. A given value of Ra does not have a unique combination of $\delta_u$, $\Delta T_m$, or $\Delta T_{rh}$, and the effective exponent on Ra is not necessarily equal to $-1/3$. It is inaccurate to assume a constant value of $\delta_u$, for example, and change topography only by changing Ra. The different $y$-intercepts of equations (\ref{eq:dyn_top_stress}) and (\ref{eq:PD83}) confirm our earlier warning that the choice of constant prefactors in these equations should be re-evaluated. In section \ref{sec:dyn_top_forward} we discussed how setting $C_1=2$ is inappropriate because this fit applies to shear convective stress, not normal convective stress. Our use of $C_2 = 5.4$ is not optimal because this value is based on fits to an isoviscous model \citep{Lees2019}.



Although our predictions of absolute dynamic topography (the solid grey and black lines) are much lower than what is inferred for Venus, figure \ref{fig:RMS_benchmark} shows that $\Delta h_{\rm RMS}$ is comparable between these and previous results when adjusted for our unrealistically high values of Ra. The ``Venusian average" upper mantle is expected to have an interior Ra $\sim 10^6$ \citep{Kiefer1998}. Our values are two orders of magnitude higher because our viscosities are 1--2 orders of magnitude lower than typically assumed \citep[cf.][]{Benesova2012}. Note that although the form of (\ref{eq:Ra}) suggests $d$ has the largest influence on Ra, the cube of $d$ varies only by two-fold during our model runs, while $\eta_m$ varies by ten-fold, so changes in Ra mostly reflect changes in $\eta_m$. To a lesser extent, Ra is kept high by our low temperature contrast across the convecting region \citep[$\sim$300 K; cf. 1000 K assumed by][]{Kiefer1992}, since the core cools down to the mantle temperature so quickly, and $\Delta T_m$ essentially reduces to the rheological temperature scale (equation \ref{eq:Tl}). This suggests that a Rayleigh number based on $\Delta T$ is not appropriate for our purposes; perhaps we should adapt our model to use a Rayleigh number based on internal heating.





The dependence of $\Delta h$ on Ra is elucidated in \citet{Kiefer1992}. As Ra increases, the planform of convection changes: upwellings and downwellings grow are more narrow, as does the thermal boundary layer. Spectrally, less-broad upwellings means less long-wavelength power in the boundary layer topography (more short-wavelength power, which is supported elastically). If we consider density contrasts within the boundary layer leading to thermal isostasy, then a thinner boundary layer can provide less buoyant support. The topography at the surface essentially reflects the topography and thickness of the upper thermal boundary layer, which reflects Ra. 


\subsection{Dynamic topography as a function of planet age, mass, and radiogenic element abundance}

We are interested in how $\Delta h$ scales with the bulk properties of the planet. This is best seen by normalizing $\Delta h$ with respect to a reference value (figure \ref{fig:RMS_v_planet}). We focus on the input parameters $M_p$ and $H_{4.5}$, as well as the planet age, as they are within the realm of possibility of constraining observationally for increasingly smaller exoplanets. We use an uninformed range of $H_{4.5}$ with no attention to its actual expected variation across planets. As for mass, the minimum value is around that of Mercury, and the maximum is around the mass limit for rocky planets \citep{Rogers2015}. Our extrapolation of parameterized convection to these massive planets is overly naive. Section \ref{sec:future-exoticplanets} expands upon these points. The minimum age plotted corresponds to the point in our thermal histories where the memory of initial conditions disappears.

Preliminarily, topography increases with age, decreases with mass, and decreases with radiogenic heating rate. How these parameters alter dynamic topography follows from what we expect based on their relationships with Ra. Increasing a planet's mass increases Ra via a hotter temperature (lower $\eta_m$), higher gravity, and a deeper convecting region. Increased radiogenic heating at any time in a planet's history also means that the mantle is running hotter with a higher Ra, all else held constant. As our thermal histories show, an older planet will generally be slightly cooler and have a lower Ra.
 




\begin{figure}
  \centering
  \includegraphics[width=1\linewidth]{relative_h}
\caption{Dependence of model dynamic topography, $\Delta h_{\rm RMS}$, on planet age (\textit{left}), mass (\textit{centre}), and radiogenic heating rate at 4.5 Gyr ($H_{4.5}$; \textit{right}) for different dynamic topography scalings: \citet{Parsons1983} (equation \ref{eq:PD83}; red line), \citet{Kiefer1992} (equation \ref{eq:KH92}; dark green line), and the basic scaling of $\Delta h$ with stress (equation \ref{eq:dyn_top_stress}; yellow line).  The prefactor in (\ref{eq:dyn_top_stress}) is taken to be equal to 2. All input parameters other than that varied on the $x$ axis are held constant. In the centre and right subplots, the yellow and red lines nearly coincide.}
\label{fig:RMS_v_planet}
\end{figure}


