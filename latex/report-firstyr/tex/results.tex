\section{Preliminary results}\label{sec:results}


We have used simple parameterized convection models to produce thermal histories for stagnant lid planets. From the results of these models, we have estimated dynamic topography as a function of planet age, mass, and initial radiogenic isotope abundance. Parameters are defined in Table \ref{tab:params}.

\subsection{Thermal evolution}\label{sec:thermal}

\begin{figure}
  \centering
  \includegraphics[width=0.9\linewidth]{thermal_Mars1}
\caption{Sample thermal evolution for a Mars-like planet. Solid blue lines are results from \citet{Thiriet2019}; solid black lines are from this model with the same initial conditions, and dashed lilac lines are results from \citet{Breuer2010} with dashed black lines from this model with the same initial conditions. Model parameters are identical between the two literature models and our model. The grain size in our Arrhenius rheology is tuned such that we get the same reference viscosity and temperature pair as the linear rheology in \citet{Thiriet2019} and \citet{Breuer2010}. $T$ is the interior potential temperature (averaged across the mantle and lid for the solid lines; the temperature of the isothermal mantle for the dashed lines), $q_{s}$ is the surface heat flux, $D_l$ is the stagnant lid thickness, $q_{c}$ is the flux into the bottom of the convecting region, Ur is the Urey ratio, Ra is the Rayleigh number based on temperature contrast, $\eta_m$ is the mantle viscosity, and $T_l$ is the temperature at the base of the lid. Coloured lines are only shown when that output parameter is reported in the reference. %TODO: calculate Thiriet Ur given their sfc flux and equation for $H(t)$.%TODO: add Nimmo \& McKenzie (1997) for Mars to show no-lid thermal model?
}
\label{fig:thermal}
\end{figure}

Firstly, figure \ref{fig:thermal} compares our temperature evolution for a Mars analogue planet to results from the nearly-identical stagnant lid parameterized convection model of \citet{Thiriet2019}, as well as to a similar model from some co-authors \citep{Breuer2010}. Both of these published models differ from ours in that we assume a steady-state conductive temperature profile in the lid, while they model the time-dependence of heat conduction. \citet{Breuer2010} differs from \citet{Thiriet2019} in initial conditions. \citet{Breuer2010} also includes an additional component of the core-mantle heat flux, the energy from inner core freezing (including this flux reduces $q_c$ for the first billion years). 


During the first $\sim$1.5 Gyr of evolution, the temperatures of the mantle and core are adjusting to having been initialized far out of equilibrium (in terms of $T_l$, $T_m$, and $D_l$). The core rapidly cools to the mantle temperature during this time frame, while the lid thickness adjusts such that the heat flux into its base equals the heat flux out of its surface. Thermal evolution is largely independent of initial conditions beyond about 1.5 Gyr, seen as the increasing similarity, as the planet ages, between runs that were initialized differently. The Urey ratio of the planet drops steadily: as the radioactive heating rate declines, the surface cooling rate lags behind, and Ur approaches a quasi-asymptotic value of 0.66. This is close to the classical numerically-modelled value of Ur $\sim$0.7 \citep{Schubert1980, McKenzie1981}.


The discrepancy between our values and those of \citet{Thiriet2019} are explained by our simplification of steady-state stagnant lid conduction. The RMS error between our surface heat flow and that of  \citet{Thiriet2019} is $\pm$2.69 mW m$^{-2}$, which is within the \textless4 mW m$^{-2}$ error from assuming steady-state conduction in the lid, according to \citet{Thiriet2019}. The thinning of the lid and increase of the surface heat flux happen sooner in steady state because the temperature profile is allowed to shift instantaneously. Note that this also affects the average temperature (figure \ref{fig:thermal}, upper left panel), which includes the lid temperature profile. We only show results for a Mars-like planet. We also produced the same Moon and Mercury scenarios as \citet{Thiriet2019}, and were able to match the published results equally well.  



Although not currently illustrated here, we can compare our results to thermal models that include melting \citep[e.g.,][]{Hauck2002, Kite2009}. The mantle temperatures in figure \ref{fig:thermal} tend to be higher in our cases because of this cooling mechanism missing in our model. Melt processes may indeed be the most effective way to remove heat from a stagnant lid planet's interior \citep{Armann2012}.

\begin{figure}
  \centering
  \includegraphics[width=0.8\linewidth]{Ra_hprime2.pdf}
\caption{Variation of nondimensionalized dynamic topography amplitude with Rayleigh number, showing peak ($\Delta h^\prime_{\rm peak}$; top panel) and RMS ($\Delta h^\prime_{\rm RMS}$; bottom pabel), where $\Delta h = \alpha_m \Delta T_m d_m \Delta h^\prime$ as defined in the text. The black star and plus sign are the quasi-steady state values from our thermal histories (``Venus" and ``Mars" case, respectively) using the \citet{Parsons1983} scaling from 2D cartesian convection (equation \ref{eq:PD83_0}). The dashed black line shows the scaling $\Delta h^\prime \propto$ Ra$^{-1/3}$, equivalent to \citet{Parsons1983}. The pink circles show the minimum and maximum values (of a time-dependent 2D spherical axisymmetric model) from \citet{Kiefer1998}, the magenta dashed line is the log-log fit from a 2D cylindrical numerical model in \citet{Kiefer1992}, given by (\ref{eq:KH92}), the yellow rhombi are the 3D cartesian data points from \citet{Lees2019} where the peak values are extracted from a grid expansion, and the lilac triangles are our RMS estimates from the power spectra of the various 3D spherical models in \citet{Huang2013}. All models are isoviscous except for \citet{Huang2013}, which uses temperature- and pressure-dependent viscosity. The remaining thin lines correspond to data from ASPECT simulations with fixed temperature contrast (blue) or fixed heat flux (grey), and cylindrical (solid lines) or cartesian (dashed lines) geometry. Ra numbers and scaling parameters from literature models are quoted from the authors.}
\label{fig:RMS_benchmark}
\end{figure}


\subsection{Purely-dynamic topography for solar system analogues}\label{sec:results-comparison}


Figure \ref{fig:RMS_benchmark} compares scalings of nondimensional dynamic topography with Ra in terms of both the peak value, $\Delta h^\prime_{\rm peak}$, and RMS, $\Delta h^\prime_{\rm RMS}$. Topography is nondimensionalized as $\Delta h = \alpha_m \Delta T_m d_m \Delta h^\prime$. We originally intended to compare the \citet{Parsons1983} scaling with data from \citet{Nimmo1996} and \citet{Moresi1995}, who calculate the full dynamic topography from stagnant lid mantle convection using some variation of (\ref{eq:tau_zz}). However, small but key differences in model setups made it difficult to relate them to each other---both \citet{Nimmo1996} and \citet{Moresi1995} work with fixed-flux convection and report Ra$_F$, not equivalent to fixed-$\Delta T$ convection and the thermal Ra number. Further, the fixed-flux case is associated with a different nondimensionalization, $\Delta h = ( \alpha_m q_u d_m^2 / k_m) \Delta h^\prime$. These different Ra numbers can be approximately inter-converted via Ra$_F$ = Ra Nu, where Nu is the Nusselt number, or the ratio of convective to conductive heat transfer. The fixed-flux topography nondimensionalisation can be multiplied by Nu to obtain the equivalent fixed-$\Delta T$ nondimensionalisation. 

Because \citet{Nimmo1996} and \citet{Moresi1995} do not report values of Nu for us to use here, we run our own numerical convection simulations using ASPECT \citep[published under the GPL2 license;][]{Bangerth2018}, using the same setup as these earlier authors (cylindrical box, aspect ratio 1, free-slip, constant viscosity, fixed flux). That is, in figure \ref{fig:RMS_benchmark}, the solid grey line should be identical to \citet{Nimmo1996}. \citet{Moresi1995} would differ because it is not an isoviscous model. As expected, the solid blue line overlaps with \citet{Kiefer1992}; they are based on identical model setups, however, the \citet{Kiefer1992} result is a log-log fit to a narrow range of Ra numbers, so the solid blue and dashed mauve lines diverge at higher values of Ra. Data from isoviscous 3D cartesian numerical convection runs \citep{Lees2019} is close to the 2D cylindrical results, suggesting that the difference between 2D and 3D geometry is slight in an isoviscous scenario (3D cartesian runs are closer to 2D cylindrical than to 2D cartesian geometry because 3D cartesian geometry can capture cylindrical-like motion). The decrease in $\Delta h^\prime_{\rm peak}$ between cylindrical and cartesian geometry agrees with tests on this matter in \citet{Kiefer1992}. Fixed-$\Delta T$ runs show slightly lower topography than fixed-flux runs, which may be due to how in the latter case, the temperature at the mantle base is slightly higher underneath the centre of the upwelling plume.

The RMS topography, meanwhile, shows a much better agreement between different 2D model geometries, and between fixed-flux and fixed-$\Delta T$ runs. This agreement suggests that the RMS of the plume uplift is less sensitive to the thermal structure of the plume, in contrast to the peak uplift. Isoviscous 2D spherical axisymmetric runs in \citet{Kiefer1998} also agree, as expected; note the latter model also includes internal heating. The 15 cases presented in \citet{Huang2013}, however, show lower RMS topographies, seemingly because they use a pressure- and temperature-dependent viscosity in 3D spherical geometry. These data are extracted from the spherical harmonic power spectra the authors report, as $\sqrt{\sum {S_l/(2l + 1)}}$, where $S_l$ is the power per degree in m$^2$. This discrepancy motivates further study into topography scaling laws for 2D, non-isoviscous models, which are not possible to systematically compare at this point using existing literature results. 

%The literature selection included here is subsample of the studies listed in Table \ref{tab:dyn_topo_obvs}, chosen because they calculate the full dynamic topography from stagnant lid mantle convection using some variation of (\ref{eq:tau_zz}) (as opposed to using an approximation based on thermal isostasy, e.g.).

%Several rough adjustments have been made to facilitate comparison. We must differentiate between ``peak topographic uplift," which may be associated with a single mantle plume or topographic feature, versus $\Delta h_{\rm RMS}$, which is more representative of the entire planet. As a very crude approximation, models that report a ``peak uplift" are scaled by 0.707 (the RMS of a sinusoidal signal); these are marked by an asterisk in figure \ref{fig:RMS_benchmark}. When only a global spherical harmonic power spectrum is reported, marked by a double dagger in figure \ref{fig:RMS_benchmark}, we calculate the RMS using $\Sigma_l [S(l)/(2l + 1)]^{1/2}$, where $l$ is spherical harmonic degree and $S(l)$ is the power at that degree. When the basal heating Rayleigh number is reported (along with $q_c$ and $\Delta T_m$), this value is naively converted to thermal Ra assuming a 1000-K temperature contrast across the convecting cell, Ra = Ra$_B (k \Delta T_m)/(q_c d)$, where Ra$_B$ is the basal heating Rayleigh number, noting that an exact comparison cannot be made because these two Rayleigh numbers are based on different assumptions about what is fixed in the model.

%It is worth repeating that the predictive models of dynamic topography available in the literature are not based on consistent assumptions. These assumptions include: the type of rheological law (temperature-dependent, depth-dependent, or constant), model geometry (cartesian, cylindrical, or spherical), mode of mantle heating (basal or internal), mechanical boundary conditions (free-slip or no-slip), choice of fixed parameter and corresponding Rayleigh number ($\Delta T$, $q_c$, or $q_u$), and presence of a stagnant lid, the first three of which are listed in Table \ref{tab:dyn_topo_obvs} for each study. That is, not all variables other than Ra are held constant to plot $\Delta h$. 



%All the model predictions of log($\Delta h_{\rm RMS}$) versus log(Ra) scatter around a line of slope $-1/3$ (i.e., the grey and black dashed lines), which is predicted from the scalings in (\ref{eq:dyn_top_stress}) and (\ref{eq:PD83}). An exception is the fit from \citet{Kiefer1992}, which has a shallower slope. This model is the only one that employs cylindrical geometry, which may explain some of the discrepancy. The two models that assume constant viscosity \citep{Nimmo1996, Kiefer1998} do produce overlapping results, despite only \citet{Kiefer1998} considering internal heating. The two RMS data extracted from spherical harmonic power spectra (lilac symbols) surprisingly do not accord with each other, despite being based on very similar models. This is likely due to human error in digitising these spectra (they must be digitised by hand accurately, and we cannot rely on interpolation when extracting the power at each degree). Finally, we have introduced a discrepancy by trying to combine multiple approaches in retrieving the RMS value of topography---from either a peak $\Delta h$ over a plume \citep{Kiefer1992, Moresi1995, Nimmo1996} or from the power spectrum \citep{Huang2013, Yang2016}. Both of these (preliminary) approaches are erroneous for different reasons. Only \citet{Kiefer1998} report a bona fide global RMS value for $\Delta h$.

%
%
%Topography based on thermal history models (solid lines) is non-identical to topography based on the equivalent scalings with Ra (dashed lines) because of hysteresis in the thermal models. A given value of Ra does not have a unique combination of $\delta_u$, $\Delta T_m$, or $\Delta T_{rh}$, and the effective exponent on Ra is not necessarily equal to $-1/3$. It is inaccurate to assume a constant value of $\delta_u$, for example, and change topography only by changing Ra. The different $y$-intercepts of equations (\ref{eq:dyn_top_stress}) and (\ref{eq:PD83}) confirm our earlier warning that the choice of constant prefactors in these equations should be re-evaluated. In section \ref{sec:dyn_top_forward} we discussed how setting $C_1=2$ is inappropriate because this fit applies to shear convective stress, not normal convective stress. Our use of $C_2 = 5.4$ is not optimal because this value is based on fits to an isoviscous model \citep{Lees2019}.

Dimensionalizing our thermal history model result when Ra $\sim$ $10^9$ gives values of $\Delta h_{\rm RMS}$ around 42 m. Although this prediction is clearly much lower than what is inferred for Venus (see Table \ref{tab:dyn_topo_obvs}), figure \ref{fig:RMS_benchmark} shows that our $\Delta h_{\rm RMS}$ is reasonable when adjusted for our unrealistically high values of Ra. The ``Venusian average" upper mantle is expected to have an interior Ra $\sim 10^6$ \citep{Kiefer1998}. Our values are two orders of magnitude higher because our viscosities are 1--2 orders of magnitude lower than typically assumed \citep[cf.][]{Benesova2012}. Note that although the form of (\ref{eq:Ra}) suggests $d$ has the largest influence on Ra, the cube of $d$ varies only by two-fold during our model runs, while $\eta_m$ varies by ten-fold, so changes in Ra mostly reflect changes in $\eta_m$. Further, in our parameterized model, the boundary layer thickness is independent of the layer depth. %To a lesser extent, Ra is kept high by our low temperature contrast across the convecting region \citep[$\sim$300 K; cf. 1000 K assumed by][]{Kiefer1992}, since the core cools down to the mantle temperature so quickly, and $\Delta T_m$ essentially reduces to the rheological temperature scale (equation \ref{eq:Tl}). This suggests that a Rayleigh number based on $\Delta T$ is not appropriate for our purposes; perhaps we should adapt our model to use a Rayleigh number based on internal heating.





The dependence of $\Delta h$ on Ra is elucidated in \citet{Kiefer1992}. As Ra increases, the planform of convection changes: upwellings and downwellings grow are more narrow, as does the thermal boundary layer. Spectrally, less-broad upwellings means less long-wavelength power in the boundary layer topography (more short-wavelength power, which is supported elastically). If we consider density contrasts within the boundary layer leading to thermal isostasy, then a thinner boundary layer can provide less buoyant support. The topography at the surface essentially reflects the topography and thickness of the upper thermal boundary layer, which reflects Ra. 


\subsection{Dynamic topography as a function of planet age, mass, and radiogenic element abundance}

We are interested in how $\Delta h$ scales with the bulk properties of the planet. This is best seen by normalizing $\Delta h$ with respect to a reference value (figure \ref{fig:RMS_v_planet}). We focus on the input parameters $M_p$ and $H_{4.5}$, as well as the planet age, as they are within the realm of possibility of constraining observationally for increasingly smaller exoplanets. We use an uninformed range of $H_{4.5}$ with no attention to its actual expected variation across planets. As for mass, the minimum value is around that of Mercury, and the maximum is around the mass limit for rocky planets from \citet{Rogers2015}. Our extrapolation of parameterized convection to these massive planets is overly naive (section \ref{sec:future-exoticplanets} expands upon this point). The minimum age plotted corresponds to the point in our thermal histories where the memory of initial conditions disappears.

Preliminarily, topography increases with age and core mass fraction, and decreases with mass and radiogenic heating rate. How these parameters alter dynamic topography follows from what we expect based on their relationships with Ra. Increasing a planet's mass increases Ra via a hotter temperature (lower $\eta_m$), higher gravity, and a deeper convecting region at a given time. Meanwhile, decreasing a planet's core size increases Ra only via a deeper convecting region. Increased radiogenic heating at any time in a planet's history also means that the mantle is running hotter with a higher Ra, all else held constant. As our thermal histories show, an older planet will generally be slightly cooler and have a lower Ra.
 




\begin{figure}
  \centering
  \includegraphics[width=1\linewidth]{relative_h}
\caption{Dependence of model dynamic topography, $\Delta h$, on (from left to right) planet age, total mass, core mass fraction, and radiogenic heating rate at 4.5 Gyr ($H_{4.5}$ for different dynamic topography scalings, based on the RMS and peak topography respectively: \citet{Parsons1983} (equation \ref{eq:PD83_0}; coral line) and \citet{Kiefer1992} (equation \ref{eq:KH92}; green line). Input parameters other than that varied on the $x$ axis are held constant. Endpoints for the core mass fraction are chosen such that the initial depth (but not volume) of the convecting region is the same for the smallest core mass fraction and the largest planet mass, and vice versa.}
\label{fig:RMS_v_planet}
\end{figure}


