\section{Future work}

\subsection{Plate flexure and lithospheric strength models}

The dynamic topography a planet could support based on rock strength alone is unrealistically high. In reality, this topography is ``filtered" by the elastic properties of the lithosphere. 

\begin{itemize}
\item Some preliminary equations, from e.g. \citet{Golle2012}
\item How much (\%) do we expect this to matter for steady state Earth-sized planets?
\end{itemize}

\subsection{More exotic planets}

\begin{itemize}
\item Tidally locked planets - could we adapt our model for potentially large spatial contrasts in $T_m, T_s$?
\item K/U/Th ratios vary depending on condensation history in protoplanetary disc; this introduces uncertainty... How much do we expect radiogenic isotope budgets to vary from looking at stellar catalogues?
\end{itemize}


\subsection{Observables}

\citet{McTier2018} postulate a way to obseve ``topography", but for the smaller topographic amplitudes expected for super-Earth planets we wouldn't expect a detectable signal...  How to justify that the project can could say anything useful about real life

\subsection{Implications for planetary climate}

Briefly (although this is the most interesting part)

