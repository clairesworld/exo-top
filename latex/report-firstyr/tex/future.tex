\section{Future work}

\subsection{Scalings from numerical convection}

This model depends heavily on scalings of dynamic topography obtained from isoviscous 2D or 3D numerical experiments. Yet no such experiment to date reports dynamic topography scalings with Ra for 2D temperature-dependent viscosity fluids. We have created an inconsistency by combining an isoviscous scaling law for topography with thermal history model that employs Arrhenius rheology. Further, temperature-dependent rheology would pertain to realistic rocky planets. This lack of scaling law represents a gap in the literature, which we plan to fill with our own numerical experiments using ASPECT. We hope to answer how poorly results agree between isoviscous and Arrhenius models. We intend to use these new scaling laws as a basis for our first paper, before concerning ourselves with applications of the topography model.

\subsection{Importance of melting on thermal history}\label{sec:melting}

As we have mentioned in section \ref{sec:thermal}, modelling melting processes is necessary to obtain realistic thermal histories \citep{Nakagawa2012}. % \citet{Armann2012} argue that ``heat pipe" magmatism is the dominant mode of heat loss for Venus; in this scenario, heat loss would not be controlled by the thermal boundary layer flux. Although \citet{Kite2009} give a smaller ratio of magmatic to conductive heat loss of 0.1 for a Venus-like planet, the net effect on thermal evolution is still important, even more so for younger worlds. 
One-to-one comparisons of Venus thermal models with and without melting can be found in \citet{Driscoll2014}, for example, showing mantle temperatures up to 500 K lower if melting occurs. Overall, mantle melting has cascading effects on planetary interior evolution. For example, melting extracts water from the mantle, which raises viscosity, and in a hot mantle which can melt at greater depths, deep water extraction may lead to viscosity stratifications. These stratifications have a net effect of limiting surface heat flux \citep{Korenaga2009}. This temperature-viscosity behaviour contradicts what we expect from the temperature-dependent viscosity feedback. The ratio of extrusive magmatism to intrusive magmatism further affects mantle cooling: whereas intrusive magmatism thins the lithosphere, extrusive magmatism thickens it, potentially insulating the mantle and slowing its heat loss \citep{Lourenco2018}. 

Soon we hope to test whether modelling melting can indeed lower the Ra numbers of our parameterized convection runs to give us realistic dynamic topographies (using Venus as a benchmark). \citet{Korenaga2009} and several others \citep{Kite2009, Driscoll2014, Tosi2017, Foley2018} provide a basis for implementing melting in a parameterized convection model.




\subsection{Posterior probabilities}

It could be useful to investigate the prior distributions of our model input parameters that we would expect in reality, where possible. First, radiogenic isotope abundances, and thus $H_{4.5}$, may have some variation we can account for statistically. The distribution of Th abundances has been studied for a sample of solar-twin stars \citep[it varies two-fold;][]{Unterborn2015}. This distribution relates to $H_{4.5}$ via a nebula condensation model, and either measurements of U abundance in the same sample, or assuming that U and Th vary similarly. Potassium-40 abundances in planet building blocks are more difficult to predict from stellar abundances because unlike refractory U and Th, K is a volatile element and therefore may be variably depleted during planet formation.

Second, core mass fractions have been estimated for a handful of known exoplanets. We discussed interior structure constraints in section \ref{sec:future-exoticplanets}; if a given planet is much denser than pure silicate, a large iron component is required to explain the planet's mass-radius observation, and constraints can be placed on the planet's core size \citep{Wagner2012, Suissa2018}. The potentially iron-rich planets in these studies, such as CoRoT-7b and Kepler-10b, could be seen as an end-member of core-mass-fraction variation across planets. Another end-member is represented by potentially coreless planets HD 219134, 55 Cnc e, and WASP-47 e \citep{Dorn2019}. The TESS mission is currently discovering more small rocky exoplanets on which to test these ideas \citep[see][]{Jontof-Hutter2019}.

Running many simulations where we randomly select input parameters from more-realstic prior distributions in a Monte Carlo-type analysis \citep[e.g.,][]{Dorn2018a} would be a fairly simple and non-labourious application to understand the actual probability distribution of planetary RMS topography in rocky exoplanets.


% that's it, just H_0 and CMF for now



\subsection{Planetary implications}

We have already suggested that topography is an overlooked factor in the behaviour of larger-scale cycles controlling planetary climate. Namely, topography is one way to think about subaerial land fraction (e.g., in the context of the carbon cycle). A prediction of exoplanet land area could be much less detailed than studying the emergence of continents on early Earth, for example, the timing of which is thought to control subaerial land fraction. While we have motivated this in figure \ref{fig:ocean} with rough estimates of ocean capacity as a function of mass, age, and radiogenic heating, analysis of the actual land area would need to consider how much water the planet accretes, and how water partitions in the magma ocean and later, the mantle---this would help us estimate how much water could possibly end up above ground (before necessarily needing to know how much water evaporates to the atmosphere). 

Because water is such a key molecule in studying planets, there is already a growing literature attending to exactly these questions, although many focus on planets with plate tectonics \citep[e.g.,][]{Elkins-Tanton2008, Cowan2014, Komacek2016}. Thus, in the further future we hope to work with these results from the literature, and others forthcoming, to explore the effect of planet mass on land area via topography and water partitioning. Applying our model to estimate absolute land area could tie topography directly to current studies on silicate weathering and planetary carbon cycles \citep{Graham2020}. Silicate weathering efficiency also depends on uplift rate in some parameterizations, and modifying our dynamic topography model to predict the associated uplift rates could be fortuitous. 

%As for other conceivable applications, the distribution of topography, in additon to the RMS height, is expected to have large effects on the planetary climate via atmospheric circulation. However, if a planet is tidally-locked to its host star---which is ubiquitous for rocky planets in the liquid water habitable zone around M-dwarf stars, the easiest rocky planets to detect---then its mantle circulation may re-orient the planet itself such that it always has one substellar and one antistellar plume \citep{Leconte2018}, which would imply that the distribution of dynamic topography is already known for these planets, This could potentially link our model directly to general circulation models. An understanding of exactly what parameters would be most useful for other modellers can be used as a principle to direct our future work.  





\subsection{Model validation}

How can we use measurements to test our model? A handful of astrophysical methods have been studied to this end, although for now they are hypotheses. \citet{McTier2018} postulate a way to extract ``bumpiness" from the light-curves of planets that transit their star, but for the smaller topographic amplitudes expected for massive planets, the signal would probably be undetectable. Exo-cartography, or, solving the inverse problem of 2D albedo distributions from light-curves, could in theory discriminate between land and ocean surface coverage for even Earth-sized planets \citep{Cowan2018, Farr2018, Kawahara2020, Aizawa2020}. Some observational tests of the silicate weathering feedback have also been proposed, which work by constraining CO$_2$ abundances in exoplanetary atmospheres for planets at different instellations \citep{Bean2017,  Checlair2020}---any indication of a silicate weathering feedback could imply the existence of subaerial land, and therefore topography, as well. This would require next-generation space observatories\footnote{LUVOIR, HabEx} equipped with coronagraphs or starshades to be funded, built, and launched.

