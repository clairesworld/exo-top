\section{Future work}

\subsection{Importance of melting on thermal history}\label{sec:melting}

As we have mentioned in section \ref{sec:thermal}, modelling melting processes is necessary to obtain realistic thermal histories \citep{Nakagawa2012}. \citet{Armann2012} argue that ``heat pipe" magmatism is the dominant mode of heat loss for Venus; in this case heat loss is not controlled by the thermal boundary layer flux. Although \citet{Kite2009} give a smaller ratio of magmatic to conductive heat loss of 0.1 for a Venus-like planet, the net effect on thermal evolution is still important, even more so for younger worlds. One-to-one comparisons of Venus thermal models with and without melting can be found in \citet{Driscoll2014}, for example, showing mantle temperatures up to 500 K lower if melting occurs. The ratio of extrusive magmatism to intrusive magmatism further affects mantle cooling: whereas intrusive magmatism thins the lithosphere, extrusive magmatism thickens it, insulating the mantle and slowing its heat loss \citep{Lourenco2018}.

Mantle melting has cascading effects on planetary interior evolution. For example, melting extracts water from the mantle, which raises viscosity, and in a hot mantle which can melt at greater depths, deep water extraction may lead to viscosity stratifications. These stratifications have a net effect of limiting surface heat flux \citep{Korenaga2009}. This temperature-viscosity behaviour contradicts what we expect from the temperature-dependent viscosity feedback. \citet{Korenaga2009}, along with several others \citep{Kite2009, Driscoll2014, Tosi2017, Foley2018}, provide a basis for implementing melting in a parameterized convection model.

\subsection{Scalings from numerical convection}

This model depends heavily on scalings of dynamic topography obtained from 2D or 3D numerical experiments, however, no such experiment to date reports dynamic topography scalings for 2D temperature-dependent viscosity fluids, pertaining to realistic rocky planets. This is a gap in the literature, which we plan to see if we can fill with our own numerical experiments (using the ASPECT convection code, for example). We hope to answer how poorly results agree between isoviscous and temperature-dependent-viscosity models.

\subsection{Planetary implications}

more robust estimates of land area that consider (simplistic) water delivery and partitioning. figure x didn't consider that larger planets also have more water in the mantle absolutely


\subsection{Prior distributions of input parameters}

It could be useful to explore more-realistic expected ranges of our model input parameters, where possible. Firstly, radiogenic isotope abundances, and thus $H_{4.5}$, may have some variation we can account for statistically. The distribution of Th abundances has been studied for a sample of solar-twin stars \citep[it varies two-fold;][]{Unterborn2015}, this relates to $H_{4.5}$ via a nebula condensation model, and either measurements of U abundance in the same sample, or assuming that U and Th vary similarly. Potassium-40 abundances in planet building blocks are more difficult to predict from stellar abundances because unlike refractory U and Th, K is a volatile element and therefore may be variably depleted during planet formation.

Secondly, core mass fractions have been estimated for a handful of known exoplanets. We discussed interior structure constraints in section \ref{sec:future-exoticplanets}; if a given planet is much denser than pure silicate, a large iron component is required to explain the planet's mass-radius observation, and constraints can be placed on the planet's core size \citep{Wagner2012, Suissa2018}. The potentially iron-rich planets in these studies, such as CoRoT-7b and Kepler-10b, could be seen as an end-member of core-mass-fraction variation across planets. Another end-member is represented by potentially coreless planets HD 219134, 55 Cnc e, and WASP-47 e \citep{Dorn2019}. The TESS mission is currently discovering more small rocky exoplanets on which to test these ideas \citep[see][]{Jontof-Hutter2019}.


% that's it, just H_0 and CMF for now




\subsection{Model validation}

How can we use measurements to test our model? A handful of astrophysical methods have been studied to this end, although for now they are hypotheses. \citet{McTier2018} postulate a way to extract ``bumpiness" from the light-curves of planets that transit their star, but for the smaller topographic amplitudes expected for massive planets, the signal would probably be undetectable. Exo-cartography, or, solving the inverse problem of 2D albedo distributions from light-curves, could in theory discriminate between land and ocean surface coverage for even Earth-sized planets \citep{Cowan2018, Farr2018, Kawahara2020, Aizawa2020}. This would require next-generation space observatories\footnote{LUVOIR, HabEx} equipped with coronagraphs or starshades to be funded, built, and launched.

