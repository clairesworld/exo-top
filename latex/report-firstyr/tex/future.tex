\section{Future work}

\subsection{Importance of melting on thermal history}\label{sec:melting}

As we have mentioned in section \ref{sec:thermal}, modelling melting processes is necessary to obtain realistic thermal histories \citep{Nakagawa2012}. \citet{Armann2012} argue that ``heat pipe" magmatism is the dominant mode of heat loss for Venus; in this case heat loss is not controlled by the thermal boundary layer flux. Although \citet{Kite2009} give a smaller ratio of magmatic to conductive heat loss of 0.1 for a Venus-like planet, the net effect on thermal evolution is still important, even more so for younger worlds. One-to-one comparisons of Venus thermal models with and without melting can be found in \citet{Driscoll2014}, for example, showing mantle temperatures up to 500 K lower if melting occurs. The ratio of extrusive magmatism to intrusive magmatism further affects mantle cooling: whereas intrusive magmatism thins the lithosphere, extrusive magmatism thickens it, insulating the mantle and slowing its heat loss \citep{Lourenco2018}.

Mantle melting has cascading effects on planetary interior evolution. For example, melting extracts water from the mantle, which raises viscosity, and in a hot mantle which can melt at greater depths, deep water extraction may lead to viscosity stratifications. These stratifications have a net effect of limiting surface heat flux \citep{Korenaga2009}. This temperature-viscosity behaviour contradicts what we expect from the temperature-dependent viscosity feedback. \citet{Korenaga2009}, along with several others \citep{Kite2009, Driscoll2014, Tosi2017, Foley2018}, provide a basis for implementing melting in a parameterized convection model.


\subsection{Plate flexure and lithospheric strength models} \label{sec:future-elastic}

So far the dynamic topography models we use fall under the ``instantaneous viscous flow" approximation: assuming the surface of the planet reacts to stresses instantaneously and permanently. In reality, the planet has a non-instantaneous response to stress via viscous relaxation and elastic filtering, and any load on its surface is transient. Topographic uplift is filtered by the lithosphere, which behaves like a thin elastic shell. The amount of filtering depends on the thickness of the part of the lithosphere behaving elastically, the primary control of which is temperature \citep{Watts2001}.

Elastic properties of the lithosphere are usually considered relevant only for short-wavelength topography (see section \ref{sec:top_mechs}), Yet several recent studies have shown that elastic filtering can affect long-wavelength topography as well: the instantaneous viscous flow approximation will overestimate long-wavelength dynamic topography by \textless10\% for thin elastic lithospheres like Venus, and can be $\sim$50\% for thick ones \citep{Zhong2002, Golle2012, Dumoulin2013}. Although this may seem like a relatively small error for Venus-like conditions, we do not know how elastically thick planetary lithospheres are $\textit{a priori}$.

\subsection{Convection dynamics at higher massses}

Rocky planets more massive than Earth have interior pressures much higher than Earth's. At these higher pressures, postperovskite dissociates into oxides, leading to new phase transitions \citep[up to four, meaning a five-layered mantle;][]{vandenBerg2019} and a less-viscous lower mantle via a pressure-weakening effect \citep{Umemoto2011, Karato2011, Tackley2013, Umemoto2017, Shahnas2018, Ritterbex2018, vandenBerg2019}. For very high pressures, the lowest part of the mantle could melt into a super basal magma ocean \citep{Labrosse2007}. Density and viscosity contrasts due to phase changes in the mantle affect the patterns of convection. For example, there may be one or more high-pressure, high-viscosity middle layers of sluggish convection bounded by low-viscosity convecting zones in the upper and lower mantle. This viscosity stratification reduces surface heat loss, the same as when dehydration causes viscosity stratification (section \ref{sec:melting}). How well we can extrapolate parameterized convection to larger planets remains to be tested. %\citet{Tachinami2014} have found no thermal convection at all for planets \textgreater 5~M$_\oplus$ due to adiabatic compression in the deep mantle, depending on the pressure dependence of thermal expansivity.


\subsection{Bulk compositions} \label{sec:future-exoticplanets}

So far we have assumed a Mars or Venus-like bulk composition in our choice of thermodynamic and rheological parameters and interior structure. In reality, we expect variation in \textit{(i)} the between-system bulk composition of protostars and protoplanetary disks, and hence the subsequent composition of planets \citep{Bitsch2020}; as well as \textit{(ii)} the radial distribution, for a given star, in the solids that form planets (e.g., Fe-, Mg-, Si-, Ca-, Al-, and Na-minerals) \citep{Dorn2019, Miyazaki2020}. 

Some constraints on exoplanets' bulk compositions and iron core mass fractions are possible. The depth of a planet's mantle may be estimated from constraints on its core mass fraction. Astrophysical observations of planet mass and radius can be compared to theoretical predictions of mass-radius relationships based on linear combinations of equations of state for different planetary materials, namely hydrogen gas, water, silicate, and iron \citep{Seager2007, Valencia2007, Rogers2010, Dorn2018}. While radius, mass, and the relative abundances of these four components are degenerate with each other, this degeneracy is somewhat ameliorated through combining mass-radius observations either with stellar abundances of refractory elements, in particular the Fe/Si ratio \citep{Dorn2017a, Brugger2017}; or with estimates of a planet's Love number from light curves \citep{Kellermann2018}. Further, if a given planet is much denser than pure silicate, a large iron component is required to explain the planet's mass-radius observation, and constraints can be placed on the planet's core size \citep{Wagner2012, Suissa2018}. The potentially iron-rich planets in these studies, such as CoRoT-7b and Kepler-10b, could be seen as an end-member of core-mass-fraction variation across planets. Another end-member is represented by potentially coreless planets HD 219134, 55 Cnc e, and WASP-47 e \citep{Dorn2019}. The TESS mission is currently discovering more small rocky exoplanets on which to test these ideas \citep[see][]{Jontof-Hutter2019}.

Radiogenic isotopes are particuarly relevant elemental abundances due to their effects on planetary thermal evolution. These isotopes may have some variation we can account for statistically. The distribution of Th abundances has been studied for a sample of solar-twin stars \citep[it varies two-fold;][]{Unterborn2015}, this relates to $H_{4.5}$ via a nebula condensation model, and either measurements of U abundance in the same sample, or assuming that U and Th vary similarly. Potassium-40 abundances in planet building blocks are more difficult to predict from stellar abundances because unlike refractory U and Th, K must somewhow be transported in from the cool outer solar system. Further, we can consider that stars forming later in the galaxy's lifetime would contain more U and Th since these high-atomic number elements are only synthesized in supernovae---\citet{Frank2014} use a galactic chemical evolution model to predict U and Th abundances with respect to the date of star formation. 


%Finally, if we believe that a planet has a predominantly rocky composition, then we are able to constrain its relative abundances of Mg/Si and Fe/Si, since these ratios have limited variability across nearby stars \citep{Hinkel2014}. \citet{Dorn2017, Dorn2018a} demonstrate a Bayesian approach to modelling the interior structures of rocky exoplanets using these stellar abundance constraints, which would be useful if we eventually require self-consistent mantle density profiles. More detailed planetary formation models have been used to study the expected breadth of Fe/Mg ratios in rocky planets due to giant impacts \citep{Scora2020}.



\subsection{Model validation}

How can we use measurements to test our model? A handful of astrophysical methods have been studied to this end, although for now they are hypotheses. \citet{McTier2018} postulate a way to extract ``bumpiness" from the light-curves of planets that transit their star, but for the smaller topographic amplitudes expected for massive planets, the signal would probably be undetectable. Exo-cartography, or, solving the inverse problem of 2D albedo distributions from light-curves, could in theory discriminate between land and ocean surface coverage for even Earth-sized planets \citep{Cowan2018, Farr2018, Kawahara2020, Aizawa2020}. This would require next-generation space observatories\footnote{LUVOIR, HabEx} equipped with coronagraphs or starshades to be funded, built, and launched.

