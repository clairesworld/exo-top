\section{Background}

The interior of a planet affects its surface character \citep{Kereszturi2016, Meadows2018, Shahar2019, Dehant2019}. The changing motions inside are continuously modulating important things like the volume of its oceans, the composition and pressure of its atmosphere, its surface elasticity, its magnetic field, and its overall climate and habitability \citep{Noack2014, Foley2016, Wordsworth2016, Tosi2017, Wordsworth2018}. This has not been given the attention it deserves, normally because interiors are not directly accessible to astrophysical observation. Nevertheless, clever advances brought on by the characterization era of exoplanet science\footnote{i.e., atmospheric spectroscopy} now permit some level of characterizing interiors as well, through measurements of a planet's bulk density, the host star chemistry, and the detection of atmospheric species \citep{Santos2017, Dorn2017, Dorn2017a, Dorn2018, Bower2019, Madhusudhan2020}. 

This is especially exciting because the average solar system turns out to be quite unlike our own. We have been learning this since the very first detection of an exoplanet, a bafflingly close-in Jupiter-mass world \citep{Mayor1995}. As exoplanet occurrence rate studies reveal \citep{Kunimoto2020}, every Sun-like star\footnote{FGK spectral class} in the solar neighbourhood is statistically likely to host at least one planet of mysteriously intermediate mass, between Earth and Neptune---before the discovery of exoplanets we had no reason to believe these existed. Unfortunately they are all too far away to reach by spacecraft, so we cannot know what the surfaces of these ``super-Earth" or ``sub-Neptune" planets look like.\footnote{Although some level of ``exocartographic" 2D retrieval and/or spatially-resolved direct imaging may be possible in the future \citep{Farr2018, Madurowicz2020a}.} In spite of this, there is a growing theoretical literature on how different planetary processes scale with mass: internal structure \citep{Valencia2006, Zeng2017}, tidal response \citep{Tobie2019}, outgassing \citep{Kite2009, Noack2017, Dorn2018a}, geodynamos \citep{Gaidos2010}, tectonics \citep{ONeill2007, Korenaga2010}, and so on. 

Here we are interested in the planet mass-scaling of a particular expression of interior dynamics not considered in the exoplanet context: the occurence of topography. Like others have done---whether or not this is the best approach---we use Earth-size planets as a starting point, and this report is mostly limited to interacting with previous work as such. Topography is intimately linked to temperatures and mantle flows inside the planet.

\subsection{Terrestrial planet interiors through time}

The late stage of planetary accretion leaves a molten and unrecognizable world \citep{Elkins-Tanton2012}. The magma ocean planet crystallizes to rock, if not orbiting too close. During this process the gravititational differentiation of its interior and the partitioning of volatile species between the mantle and the primary crust and atmosphere sets the ``intitial state" of the solid planet \citep{Tosi2019}. %that flows viscously on million-year timescales.\footnote{Actually, viscous deformation supercedes elastic deformation in a century, but these flows are so slow they do not have appreciable effects yet.} 
Temperature is such an ubiquitous property when describing the physics of any body that this is what we talk about first when we talk about how a planet evolves. Then perhaps the most idealized configuration for a planet is one at thermal equilibrium. This planet as a whole is losing heat at the same rate it is producing heat internally. The ratio of planetary heat production to heat loss is the Urey ratio,
\begin{align}\label{eq:Ur}
{\rm Ur} = \frac{Q_{\rm radiogenic}}{Q_{\rm surface}},
\end{align}
a value of 1 denoting thermal equilibrium.

We can imagine a young planet having enough rapidly-decaying radiogenic isotopes and enough gravitational energy leftover from accretion that its baby Ur can be much larger than 1. Three things happen with age: its hot core transfers extra heat into the mantle until they are in balance; its interior potassium-40, uranium-235, uranium-238 and thorium-232 continue to decay (but more slowly); and the surface loses heat to the atmosphere and eventually to space. This thermal evolution is described via the energy balances
\begin{align}\label{eq:T_ODE}
\begin{split}
M_m c_{m} \frac{{\rm d}T_m}{{\rm d}t} &= -Q_{\rm u} + Q_{\rm rad} + Q_c, \\
M_c c_{c} \frac{{\rm d}T_c}{{\rm d}t} &= -Q_c,
\end{split}
\end{align}
where $M_m$ is the mantle mass, $c_{m}$ is the mantle specific heat capacity, $Q_{\rm rad}$ is the mass-integrated radiogenic heat flux, $Q_{u}$ is the magnitude of the surface-integrated heat flux out of the top of the mantle (the subscript $u$ upper boundary layer). The analagous notation with subscript $c$ applies to the core. 

The above is an over-simplification; it fails to capture a real rocky planet's thermal history. Geochemical evidence of the Earth's radiogenic heat production only accounts for a fraction of its heat loss; geochemically-derived estimates of the Ur of Earth are more like 0.2\textendash 0.5 \citep{Korenaga2008a}. Meanwhile, equilibrium solutions to (\ref{eq:T_ODE}) result in a geophysically-derived Ur $\sim$0.7 \citep{Schubert1980, McKenzie1981}. This discrepancy between geochemistry and geophysics is known as the classical Urey ratio paradox, and has led many to speculate after missing heat sources in the mantle, multilayered mantle convection, or other solutions \citep[see review in][]{Korenaga2008a}. An even simpler solution is to treat Earth's present Ur as transient.

While in more recent times, 2D thermo-chemical numerical convection-melting models \citep{Nakagawa2012} have indeed matched Earth's current Ur, clearly plate tectonics-associated heat fluxes represent an additonal complexity on top of the energy balance in equation (\ref{eq:T_ODE}), as do other components like the crystallization of the inner core. %Some geophycists have argued that we require a sufficiently high radiogenic heating in the present day such that Ur \textgreater 0.7, otherwise temperatures in the recent past would be unrealistically high (the ``thermal catastrophe") \citep{Christensen1984}. 
In a review of the matter, \citet{Korenaga2008a} puts forward that Ur for modern Earth is indeed as low as geochemistry suggests---a ``defining feature" of our planet. We include this discussion early on in order to communicate these limits and pitfalls of thermal history models.
%\begin{itemize}
%\item when is a planet at ``steady state"? \citep{Benesova2012} define statistically steady state: (mean  Nusselt  number  at  both  surface  and  core  mantle  boundary  does  not  change  with  time)
%\end{itemize}


For although the preliminary analyses contained in this work do not go into any detail of statistics and probability, it is important to have a sense of the uncertainty inherent in thermal modelling. Of course the parameters that go into it are uncertain; the inherent structure set by the model is also uncertain. \citet{Seales2020} provide a thorough study of these and other different uncertainties for planets (assuming plate tectonics). Structural uncertainties can lie in loosened or tightened feedbacks loops: in this case there is a parameter $\beta$ that describes the strength of the relationship between heat loss and convective vigour. While classical models give $\beta = 1/3$, $\beta$ can take a range of values including negative \citep{Korenaga2003}. The distribution of $\beta$ represents a structural uncertainty which can either amplify or dampen the \textit{parameter} uncertainty. Now imagine that a planet's convective heat transport style could change considerably depending on planet mass (see section \ref{sec:superearth-dynamics}). This represents another structural uncertainty, made worse by the fact that we cannot actually measure exoplanet masses very well.\footnote{For the current data, we can measure planet masses safely to within 25\% \citep{Otegi2019}.} 

The nature of the question we are after does not allow any kind of deterministic approach, and ultimately our answers will be probablistic. That is, if someone points out a planet to us, we would not be able to predict its topography, but we might be able to say something about the likelihood of topography on planets with that mass (and given that astrophysical environment).

%\begin{itemize}
%\item Maybe apply Seales uncertainty framework to list the kinds of uncertainty in this work?
%\end{itemize}

\citet{Seales2020} are able to validate their test ensemble model runs against Earth, and conclude ambiguity. Any application to exoplanets must figure out how to be comfortable with this. % Our main requirement from thermal history models is that they quantify temperatures and heat fluxes through the mantle and lid; parameterized models can largely provide a means for this.

\subsection{Parameterized convection models}

The temperature difference across the mantle drives convection. Full 3D mathematical descriptions of convecting mantles exist to varying degrees of complexity \citep[e.g.,][]{McKenzie1974, Schubert1992}. These are costly models to run, so for a number of reasons many runners opt for parameterized convection instead \citep{Sharpe1979, Schubert1980, Davies1980}. %Such models are ``parameterized" because they rely on nondimensionalities and scaling laws, which can be tuned to results from full numerical models. 
Work on temperature-dependent viscosity convection goes back at least five decades, and we do not attempt to summarize it all here, but some enduring points are included.

The simplest of these cases is described by a rectangular convecting cell, with no internal heating, and with a lower layer (core) as a (finite) source of heating \citep[e.g.,][]{McKenzie1981}. The core of the cell is isothermal, with the only temperature variations occurring across the boundary layers at the surface and base---this temperature profile is seen in laboratory experiments of convection \citep[e.g.,][]{Davaille1993}. The upper thermal boundary layer arises as hot upwelling material spreads out at the top of the convecting cell (McKenzie, 1967).

Because most of the interesting behaviour of this convecting cell is contained in the thermal boundary layers, parameterized models work by scaling boundary layer heat fluxes. Parameterized models will be valid as long as dynamic processes in the thermal boundary layers happen faster than the planet loses heat \citep{Sharpe1979, Korenaga2008a}. As detailed in the methods section of this report, this can be done using a handful of nondimensional parameters. The Rayleigh number quantifies the ``vigour" of convection. This number is a ratio which asks whether a material will diffuse away its heat before it can convect. As such, it is analytically equal to the ratio of time scales for heat transport by conduction (cooling) to heat transport by convection (overturning):

\begin{equation}\label{eq:Ra}
\mathrm{Ra} = \frac{\tau_{\rm cool}}{\tau_{\rm overturn}}= \frac{\alpha \rho g \Delta T d^3 }{ \kappa \eta(T)},
\end{equation}
where $\alpha$ is thermal expansivity, $\rho$ is density, $g$ is gravity, $\Delta T$ is the temperature contrast across the layer, $d$ is the thickness of the layer, $\kappa$ is the thermal diffusivity, and $\eta$ is the dynamic viscosity. There exists a critical Rayleigh number, Ra$_{\rm crit}$, below which convection is so weak it halts altogether. Because Ra depends on $d$, we can conceptualize the thermal boundary layer thickness as the value of $d$ at which Ra = Ra$_{\rm crit}$. The value of Ra$_{\rm crit}$ depends on the wavelength of the disturbance that initiates convection, but often a constant value is assumed for a planetary mantle. \citet{Turcotte2002} analytically derive its minimum value to be 657.5 for heating from below, and 2755 for internal heating. 

If Ra is the best descriptor of a convecting system, $\eta$, discussed next, is ultimately responsible for determining Ra.



\subsubsection{Our understanding of mantle rheology}\label{sec:rheology}

We would expect rheology---the way material deforms and flows---to make up a large amount of our model uncertainty for interior dynamics \citep{Dumoulin2013}. Despite good effort, the rheology of even our own planet is difficult to measure. Viscosity changes with time and may be nonlinear. Throughout this work we take ``viscosity" to mean dynamic viscosity, $\eta$. If we wish to refer to kinematic viscosity, $\nu = \eta / \rho$, we will say so explicitly. 

Experiments show that viscosity has an Arrhenius dependence on temperature,
\begin{equation}\label{eq:eta_Arrhenius}
\eta(T) = b \exp\left(\frac{Q}{R_b T}\right),
\end{equation}
where $R_b$ is the gas constant, $b$ is a prefactor, and $Q = E_a + pV$ is the activation enthalpy, with pressure $p$ and activation volume $V$ \citep{Karato1993}. The prefactor is
\begin{equation}\label{eq:eta_b}
b = \frac{\mu}{2 A_{rh}} \left(\frac{h_{rh}}{B}\right)^m,
\end{equation}
where $\mu$ is the shear modulus, $A$ is a preexponential factor (of order $10^{15}$ for diffusion creep), $B$ is the Burgers vector (quantifying the distortion in the crystal lattice due to dislocation), $h_{rh}$ is the typical grain size of the rock, and $m$ is an exponent characteristic to the type of creep.

In many applications of parameterized convection, this rheology law can be linearized, e.g., via the Frank-Kamenetskii parameterization,\footnote{Named for the Soviet scientist David A. Frank-Kamenetskii, who developed thermal combustion theory in the 1930s.}
\begin{equation}
\eta(T) = \eta_0 \, \exp{\left[-\gamma \left(T - T_0\right)\right]}
\end{equation}
where $\eta_0$ and $T_0$ are a reference viscosity and temperature, usually taken at the surface, and $\gamma = Q/(R_b T_i^2)$ with internal temperature $T_i$. We can get more complicated using pressure- and stress-dependent rheology, such as the damped Frank-Kamenetskii approximation described in \citet{Noack2013}, which is more accurate with respect to the Arrhenius law for a wider range of parameters.

It is of course nontrivial to choose which parameter values to use. To help, we may group viscosities along two axes, \textit{(i)} creep mode and \textit{(ii)} water content, for which different classes of values for $b$ and $E_a$ have been obtained using progressively fancy statistical inversion methods \citep{Karato1993, Korenaga2008, Karato2010, Mullet2015, Jain2019}. The key difference between diffusion and dislocation creep is that the latter is nonlinear as it depends on stress---``Newtonian" versus ``non-Newtonian" rheology is interchangeable with diffusion versus dislocation creep, respectively. Water weakens rock and helps it flow, represented by lower values of $b$ and $E_a$ for ``wet" rheologies \citep{Karato1993}.

%\subsubsection{Diffusion creep or dislocation creep?}
%
%\begin{itemize}
%\item Diffusion creep depends on grain size, dislocation creep does not but depends on strain rate (nonlinear) \citep{Turcotte2014}.
%\item Can be accounted for by changing the parameter values in equations (\ref{eq:eta_Arrhenius}) and (\ref{eq:eta_b}). Although diffusion creep is nonlinear, one can still use linearized equations giving an effective viscosity... see equation 7-191 in T\& S textbook for general viscosity. see s\& m 96
%\item \citet{Karato1995} detected a preferred orientation in upper mantle grains using seismic anisotropy, which is consistent with dislocation creep in this region.
%\item On this basis, \citet{Reese1998} argue that stagnant lid convection with dislocation creep is a good model for generic rocky planet interiors. 
%\end{itemize} 
%
%\subsubsection{Dry or wet?}
%
%\begin{itemize}
%\item \citet{Nakagawa2015}
%\item The distinction between rheologic behaviour of wet versus dry materials is less pronounced than the mode of creep. This is represented by smaller changes of parameter values in equations (\ref{eq:eta_Arrhenius}) and (\ref{eq:eta_b}).
%\item Introducing water changes viscosity by...
%\item Nevertheless, the fact that Earth's interior probably contains more water than Venus or Mars is thought to have played a major role in their divergent evolutions \citep{Kaula1990}; e.g., water is probably necessary for the initiation of plate tectonics; low-viscosity aesthenosphere is a consequence of Earth's higher mantle volatile content
%\end{itemize} 


\subsubsection{Self-regulation of viscosity}

The forms of (\ref{eq:eta_Arrhenius}) and (\ref{eq:Ra}) suggest a negative feedback that dates back to \citet{Urey1955}: increasing temperature decreases viscosity, which increases Ra. Increasing Ra means a thinner thermal boundary layer and more heat flow out of the mantle. This in turn cools the mantle down again to something viscous again. This would buffer the interior mantle at a near-constant viscosity, and limit the feasible ranges we could expect for the mantle viscosities of terrestrial exoplanets (an attractive idea). Another way of conceptualizing the same thing is that planets want to stay at thermal equilibrium (Ur $\sim$ 1).%: if their surface heat flux is higher than their internal heat production (Ur \textgreater  1), they will cool down as they are losing more heat. The lower temperature means less vigourous convection (lower Rayleigh number), meaning a diminishing heat flux out of the top, meaning a rise in viscosity. 
In theory this thermostat operates as long as viscosity is inversely temperature dependent, all else being the same. 

It has never been ``proven" to operate on Earth. One can see how in realistic situations this might not always work perfectly. There are subtle nonlinear effects here that can confuse the viscosity thermostat. \citet{Solomatov1996} bring up grain size---usually this is a constant in equation (\ref{eq:eta_b}), however, mineral phase transitions cause grain size to shrink and then grow back again slowly. 

\citet{Korenaga2008a} generalize this to show whether or not the mantle viscosity is self-regulating comes down to the ratio of heat flux adjustment temperature scale to the radiogenic decay time scale, named the Tozer number;\footnote{commemorating D. C. Tozer's early work on viscosity self-regulation} its value must be $\gg$ 1 for efficient self-regulation. Solving stagnant lid parameterized convection---i.e., the thermal boundary layer reacts rapidly than the mantle cools---is associated with a Tozer number of ??. This agrees with Earth's geochemically-inferrered Ur $\sim$0.2--0.5, analytically Ur $\to 1 - \rm{Tz}^{-1}$ if Tz \textgreater  1. Thus if Tz is sufficiently large, Ur $\sim$ 1. Plate tectonics is associated with a lower Tz. Naturally-expected reasons for low Tz include dehydration due to melting \citep{Korenaga2009}
.



\subsection{Regimes of mantle convection}

Possibly the most important fork in a rocky planet's path through time is whether or not it develops (and sustains) plate tectonics. Earth is the only known planet to have taken this route, and this has possibly allowed it to become alive \citep[see][]{Lenardic2016}. Plate motions, as well as completing the circuit for exchange of material between interiors and atmospheres,\footnote{Though plate tectonics is not necessary for volatile exchange \citep{Lenardic2018}; e.g., Io resurfaces just as quickly as Earth.} can expel heat from planet interiors more efficiently. On the other hand, the ``stagnant lid," one-plate planets illustrated in figure \ref{fig:stagnant_lid} will run hotter for the same rheology and surface heat flow \citep[e.g.,][]{Stevenson2003}. 

This is because the lid is too viscous to participate in any convection; it only moves heat using conduction, and thus blankets the whole planet. Convection occurs only below the lid where temperatures are hot enough to lower viscosity and facilitate convection \citep{Morris1984, Christensen1984, Hansen1993, Solomatov1995}. This fact inspires many questions about whether stagnant lid planets can form ``continents" and other hypsometric details. The character of certain tesserae on Venus imply tectonic origin, questioning whether this world always had a purely stagnant lid \citep{Bindschadler1991, Lenardic1991}.

This work considers a stagnant lid regime under the premise that it represents a ``default" rocky planet \citep{ORourke2012}; we avoid the unknown complexities introduced by mobile plates. Stagnant lid planets are more common in at least our own solar system. There is a third ``episodic" or ``transitional" regime where the entire lithosphere occasionally sinks, triggering catastrophic resurfacing episodes \citep{Moresi1998}. We mention this for completion but do not worry about it any more. Actually, many if not most planets spend their time in this regime, with stagnant lid and Earth-like plates being endmembers \citep{Foley2014}. Planets need not stick to one regime over their lives, and under what conditions a planet develops plate tectonics has a vast literature and not a semblance of agreement \citep[e.g.,][]{Valencia2007, ONeill2007, Foley2012, Tackley2013, Noack2014a, Korenaga2010, Miyagoshi2018}, although it probably requires a weak lithosphere that bends and snaps in areas of localized shear stress. The question itself is probably a red herring, and in reality one cannot predict plate tectonics from state variables \citep{Lenardic2016, Weller2018}.



 %Although sometimes the lid is equated with the lithosphere, we do not consider a planet that has a low-viscosity aesthenosphere layer below a lithosphere,\footnote{Venus probably has no low-viscosity channel in the upper mantle \citep{Kiefer1986, Nimmo1996}.} and thus avoid these precise terms (we do not know if all planets have aesthenospheres). Note that geographically, the lid contains the (oceanic) crust, which represents a petrological transition; the less-dense crustal material floats to the surface. We are not considering petrology at this point.

Laboratory experiments using corn or golden syrup show that stagnant lids naturally develop in convection cells with large viscosity contrasts across the thermal boundary layer of about an order of magnitude \citet{Davaille1993, Giannandrea1993}. Silicate rock does have a strongly temperature-dependent rheology \citep{Karato1993}, and indeed Venus and Mars appear to have stagnant-ish lids. 

This behaviour has been reproduced numerically, inheriting ideas from boundary layer theory \citep{Solomatov1995, Moresi1995a, Solomatov1996a}. Maps of convective regimes in viscosity contrast-Rayleigh number space can be found in \citet{Solomatov1996a} and more recently in \citet{Huttig2011} and \citet{Miyagoshi2015}, with stagnant lids favouring higher Rayleigh numbers \textgreater 10$^{6-7}$ and viscosity contrasts \textgreater 10$^4$. \citet{Solomatov1995} formally defines the transition between an episodic and a stagnant lid as a transiton in the instability mode of the cold thermal boundary layer: rather than the entire boundary layer being convectively unstable, in a stagnant lid regime it is just a thin sublayer near its base. %Fluid dynamicists have worked since the 1980s and 1990s to increase the modelling complexity of stagnant lid convection, such as using non-Boussinesq approximations (i.e., considering adiabatic compression) to calculate the minimum viscosity contrast for stagnant lid development \citep{Miyagoshi2015}.%; in exoplanetary applications, these show thicker lids for increasing planet mass \citep{Miyagoshi2018}.


%\begin{itemize}
%\item consider this bit from \citet{Papuc2008} on divergent thermal histories of stangnant/mobile lids (quoted) ``convection involves only the hottest, least viscous parts of the bottom of the lithosphere (Davies, 2007b, Reese et al., 1998, Solomatov and Moresi, 1996), so there is less buoyancy available to drive convection. Accordingly the mantle heats up and its viscosity decreases until it can convect vigorously with the smaller buoyancy forces. By this means planets with a stagnant lithosphere adjust their internal state so they remove the same amount of heat as mobile-plate planets. In other respects the behavior is similar to mobile-lid planets, and a similar relationship between Nusselt number and Rayleigh number applies (Gurnis, 1989, Davies, 2007b, Davaille and Jaupart, 1993). "
%\end{itemize}

\begin{figure}

  \centering
  %\includegraphics[width=0.5\linewidth]{title}
% \textit{( - for ideas consider fig 4.1 in melosh tectonics chapter})}
\caption{Structural model of a stagnant lid planet.}
\label{fig:stagnant_lid}
\end{figure}

%Implementing this lid literally on top of our parameterized convection model follows the approach in \citet{Thiriet2019}. On the other side of the upper thermal boundary layer we add a cold shell, which can only transfer heat by conduction due to its enormous viscosity and subcritical Rayleigh number. This shell extends from the thermal boundary layer to the surface of the planet. The temperature decrease across the upper thermal boundary layer takes a fixed value set by the temperature-derivative of viscosity \citep{Davaille1993}. This is distinct from the surface temperature of the planet; hence the top of the upper boundary layer is found somewhere below the surface.









\subsection{Stress and topography}

Earth's topography is much more complicated than anything a parameterized convection model could predict. It is hypsometrically bimodal; continents and oceans belong to different groups. Yet this is a complexity not seen on Mercury, Venus, the Moon, or Titan \citep{Keller2009, Lorenz2011}. (Mars has dichotomous crustal thickness and shows bimodality for different reasons than Earth: a further complexity.) This could imply that topography on these solar system bodies is, to the first order, controlled by single processes.

Early explorations of topography in a ``planetology" context looked at the limit where topography is supported by the strength of its own surface rock \citep{Jeffreys1929, Scheuer1981}. At the most basic limit, one can keep piling rock onto a surface until the yield stress $Y$ of the rock is reached and it collapses under its own weight:
\begin{equation}\label{eq:h_max}
h_{\rm max} \propto \frac{Y}{\rho g},
\end{equation}
where $h_{\rm max}$ is the height of the load when its downwards gravitational force equals $Y$, $\rho$ is the density of the rock, and $g$ is acceleration due to gravity. Plugging in $Y = 100$ MPa (the crushing strength of granite), $\rho = 2700$ kg m$^{-3}$ (the density of the Earth's crust), and $g = 10$ ms$^{-2}$ gives $h_{\rm max}$ of about 3.7 km. This is too low for Earth (Mt. Everest is over twice this high), so clearly the load is supported by more than just itself.

%With simple maths, \citet{Scheuer1981} give a scaling factor of 4 for a square mountain of comparable height and base. If one desires, versions of equation \ref{eq:h_max} can be derived for moutains with more complex shapes; however, $h_{\rm max}$ remains a general upper limit in all realistic cases \citep{Scheuer1981}.\footnote{Dr. Peter Young actually described to Scheuer an arbitrarily-high mountain whose profile is a smooth exponential.} 

%Stress is not as simple as a single number in a lookup table. Foremost, a mountain wouldn't appear as a sudden stress on a body that is initially unstressed, so we usually talk about stress \emph{differences} imposed by topography. The ``strength" of a material has multiple physical meanings---we have crushing strength as well as shear strength, tensile strength, etc.---and this all varies with pressure, temperature, load duration, and fracture history \citep{Lundborg1968, Byerlee1978}. Naturally there is a huge literature on the strength of rocks, e.g. all of the work on faults; we won't deal with it here.
%
%

Stength is not as simple as a single number; the full theory of elasticity can describe how stresses are distributed in a rigid body. Contour plots of the stress second invariant underneath an idealized mountain show that the maximum stress is located at a depth comparable to the load width, and equal to \nicefrac{1}{2} to \nicefrac{1}{3} of the load's weight, $\rho gh$. This is expressed as Jefferys' Theorem \citep{Melosh2011}: 
\begin{quote}
The minimum stress difference required to support a surface load of $\rho g h$ is \nicefrac{1}{2} to \nicefrac{1}{3} times $\rho gh$. This stress is usually sustained over a region comparable in dimensions to the load.
\end{quote}
This refers to the minimum stress; the real stresses can be larger. This means we have a minimum stress somewhere in the body of $Y = \nicefrac{1}{2}\, \rho g h_{\rm max}$. Strength drops off quickly with temperature and is zero above the melting temperature, so the cold lithosphere will have to provide for the whole planet \citep{Melosh2011}.

Subbing in Earth values, %$g = -\nicefrac{4}{3} \, \pi G \mean{\rho} R_p$, where $\mean{\rho} = 5200$ kg m$^{-3}$ and $R_p =6340$ km are the average planetary density and radius respectively, and assuming 100 MPa for $Y$ (the crushing strength of granite) and $\rho = 2700$ kg m$^{-3}$, 
we obtain $h_{\rm max}$ of approximately 8 km---much closer to Earth's tallest mountains. For Mars this predicts 50 km-high topography, which is also not far off reality \citep{Melosh2011}. If strength and density are constants, this already gives us a scaling for $h_{\rm max}$ with $R_p$ simply via gravity, which drops off very steeply as $R_p \to \infty$. This doesn't scale up in $R_p$ forever; asteroids and smaller bodies are more like rubble piles whose yield stress is related to their coefficient of internal friction, $f_f$, and $h_{\rm max} = f_f \mean{R}$.

\subsubsection{Mechanisms of topographic support}

\begin{figure}
  \centering
  %\includegraphics[width=0.5\linewidth]{title}
\caption{Illustration of different topographic support mechanisms}
\label{fig:topographic_support}
\end{figure}

To understand what drives this topography, we need to look at what forces balance topographic loads. Figure \ref{fig:topographic_support} illustrates three ways to do this. The first two are not the focus of this document, but are associated with the most dramatic topography on planets:
\begin{enumerate}
\item \emph{Elastic flexure} of a shell supports loads with elastic stresses developed in the lithosphere (to which the preceeding paragraphs apply, namely);
\item \emph{Airy or Pratt isostasy} occurs when high mountains made of low-density crust are underlain by either deeper low-density roots (Airy), or even lower-density material than the surrounding plains but of the same thickness (Pratt), so any mass anomaly is said to be compensated.
\end{enumerate}
Whether or not a load is supported by (1) or (2) depends on the ratio of the width of the load to the flexural parameter of the lithosphere.
%\begin{equation}
%\alpha_{\rm flex} = \left[\frac{1}{3(1 - \nu^2)}\frac{Ed_e^3}{\rho_m g}\right]^{1/4},
%\end{equation}
%where $d_e$ is the lithosphere thickness, $\rho_m$ is the underlying mantle density, $\nu$ is Poisson's ratio, and $E$ is Young's modulus. 
If the width of a load is smaller than this, elastic stress in the lithosphere will balance it. If the width is much larger, buoyancy forces will support it. Hence flexure is associated with short-wavelength topography and isostasy is associated with long-wavelength topography. 

The last mechanism is borne by convection patterns in the interior. Historically the term has always not referred to quite the same things, so it is helpful to break down \emph{dynamic topography} further \citep{Orth2011, Molnar2015, Hoggard2020} into:
\begin{enumerate}
\setcounter{enumi}{2}
\item \emph{Flow-induced tractions} on the base of the lithosphere. These tractions are exerted by the deformation of density boundaries within the viscously-flowing material below the thermal boundary layer.
\item \emph{Thermal isostasy,} variations in the thickness and thermal structure of the upper boundary layer \citep{Fowler1985}. On Earth this can be split into lithospheric and aesthenospheric components, but for planets without plates, this collapses to just variations in the viscous lid \citep[see][]{Orth2011}. Technically this is an isostatic situation, but a distinct one directly related to thermal convection in the mantle---as opposed to compositional density contrasts of Airy and Pratt isostasy. Thermal isostasy makes up for the majority of ``dynamic topography."
\end{enumerate}
The distinction matters, not necessarily because different parties disagree with their competing importance (at least on Earth), but because ``dynamic topography" does not consistently refer to either (3), (4), or both. In some circumstances, the two mechanisms can even produce identical signals \citep{Molnar2015}. This is a problem when comparing models, as we will see in section \ref{sec:dyn_top_ss}. Analytically we expect (4) to dominate (3); it can be shown that the gravity anomaly associated with (3), with no compensation, is $\Delta g(x) = 2\pi G \Delta \rho \Delta h(x)$, and the gravity anomaly associated with the density structure due to flow from an idealized temperature perturbation is $2\pi/3 \; G\rho\Delta h(x)$ \citep{McKenzie1977}. Thus for a given gravity anomaly, it is harder for completely uncompensated topography to make up the signal. 
%It can be shown that the topography associated with (3) is about a third of the topography associated with (4) \citep{McKenzie1968, McKenzie1977, Molnar2015}. 


\subsubsection{Dynamic topography forward models}

%\citet{McKenzie1968, 1977} derives analytic half-space equations for $\Delta h$ as a function of a harmonic temperature perturbation $T(x) = T_0 \cos(2\pi x/\lambda)$, associated with both static and dynamic density contrasts, In both cases the solution for $\Delta h(x)$ is propotional to $\alpha T_0 \lambda / (2\pi) \cos(2\pi x/\lambda)$, with the static scenario a factor of \nicefrac{4}{3} higher because it does not have to overcome resistance to flow from an additional vertical shear stress term (Stokes equation). 

Numerical models can calculate $\Delta h$ as a function of wavelength from the velocity and temperature fields obtained as the solutions to equations of motion and the heat transport equation. Analogous with equation (\ref{eq:h_max}), if the stresses of convection are creating topography, then the vertical component of this stress, $\tau_{zz}$, is balanced at the surface by the hydrostatic pressure of a topographic load, 
\begin{equation}\label{eq:sigma_conv}
p_0 = \rho g \Delta h = \tau_{zz},
\end{equation}
where $\Delta h$ is the amplitude of dynamic topography. Total vertical stress is given by 
\begin{equation}\label{eq:tau_zz}
\tau_{zz} = 2\eta \;\partial u_z / \partial z - p_1,
\end{equation}
where the first term is viscous stress and the second term is the pressure perturbation from thermal convection \citep{Parsons1983}. Combining (\ref{eq:sigma_conv}) and (\ref{eq:tau_zz}) with self-consistent pressure and temperature inherently includes both topography mechanisms (3) and (4), producing the ``full" dynamic topography. 

To integrate $\tau_{zz}$ over depth, wavelength-dependent integration kernels $H(k, z)$ based on stream functions can be useful, which describe how pressure boundaries deform with flow:
\begin{equation}\label{eq:H_kernel}
\hat{h}(k) = \alpha \int^{0}_{-\infty} H(k, z) \hat{T}(k, z) {\rm d}z,
\end{equation}
where $k$ is the wavenumber, $T(k, z)$ is the temperature distribution at depth $z$, and \string^ denotes the Fourier transform. How accurately this kernel can predict dynamic topography depends on how accurate the assumed viscosity structure is \citep{Hager1989, Karato2008a}. Note that if the distribution of topography were purely due to isostatic equilibrium with no flow, then $H(k, z) = 1$, while in the \citet{Parsons1983} and \citet{Kiefer1986} dynamic topography models and their genealogies, $H(k, z) \le 1$ \citep{Molnar2015}. \citet{McKenzie1968, McKenzie1977} shows this; isostatic topography is higher because it does not have to overcome resistance to flow from an additional vertical shear stress term. 

To separate the thermal and flow-induced components of (\ref{eq:tau_zz}), expressions for the maximum topography due solely to thinning of the lid or lithosphere can be derived using the HOT approximation: $\Delta h \sim 0.5\alpha (T_m - T_s) z_{l, 0}$, where $z_{l, 0}$ is the average lithospheric thickness \citep[e.g.,][]{Orth2011}.

%\begin{itemize}
%\item essentially, pure traction topography is really small compared to the density contrast component, for the same observation
%\item also analytically can show where convectively supported changes to isostatically as a fuction of $\lambda/z_0$. 
%\item thermal thinning: \begin{equation}\label{eq:thermal-thinning}
%h_{tt} \approx 0.5\,\alpha_{\rm lid} (T_m - T_s) z_{10},
%\end{equation}
%where $h_{tt}$ is the maximum thermal elevation from total thinning of the lithosphere, $T_s$ is the surface temperature, and $z_{10}$ is the average lithosphere thickness \citep{Kucinskas1994}. 
%\end{itemize}


In principle, one can obtain the relevant $\tau_{zz}$ from parameterized convection, since the upper thermal boundary layer contributes most of the stress that balances topography \citep{Parsons1983, Solomatov1995}. Thus we expect $\Delta h$ to scale with Ra and depend on the relative contributions of internal (radiogenic) and basal (core) heating \citep{McKenzie1977}. The distribution of surface topography essentially reflects the shape of upwellings and downwellings within the planet, which is contained in Ra. 



%\subsubsection{Non-instantaneous responses to stress: relaxation and elastic filtering}
%
%The mantle flows viscously, and any load on its surface will be transient. The timescale for viscous relaxation is obtained by 
%balancing the minimum stress from Jeffreys' theorem with the stress from the stress-strain rate relationship for Newtonian viscosity, and letting $h = h(t)$,
%\begin{align}
%\begin{split}
%h(t) &= h_0 e^{-t/\tau_R} \\
%\tau_R &= \frac{\eta}{0.3\rho g w}.
%\end{split}
%\end{align}
%For Earth this is about 6000 years. This was the principle used for early estimates of Earth's viscosity based on postglacial rebound. I'M DUMB - AREN'T MOUNTAIN CHAINS BUILT OVER MILLIONS OF YEARS?
%
%Dampening of topography by the elastic behaviour of the lithosphere is another important reason why dynamic topography is smaller than equation (\ref{eq:dyn_top_stress}) predicts.
%
%\begin{itemize}
%\item loads on the surface supported by buoyancy from viscous fluid below as well as elastic stress from lithosphere itself
%\item elastic theory tells us the maximum stress associated with elastic flexure under a load:
%\begin{equation}
%\sigma_{\rm flex} = \frac{3}{2 \pi^2}\frac{\lambda^2}{d_e^2}\rho_c g h,
%\end{equation}
%where $\rho_c$ is the crust density and $\lambda$ is the wavelength of the load.
%\item \citet{Zhong2002}
%\end{itemize}
%

\begin{landscape}
\thispagestyle{empty}
%\begin{table}

\footnotesize


\begin{longtable}{ @{} p{4cm} r r p{2cm} p{2cm} r p{1.5cm} p{3.2cm} p{3.1cm} @{} } 
\caption{Predictions from numerical convection of dynamic topgraphies on Venus, with important model parameters noted. Internal heating is calculated as $(q_s - q_b)/q_s$, where $q_s$ and $q_b$ are the surface and basal heat fluxes respectively. Reported values of Ra are distringuished between equation (\ref{eq:Ra}) and Ra$_b = g\alpha d^4 q_b/(c_m \kappa^2 \eta)$. When dimensionalized, using parameter values from Table \ref{tab:params}.\;\; *reported dimensionlessly \;\;   **calculated from the spherical harmonic power spectrum using $\Sigma_l [S(l)/(2l + 1)]^{1/2}$.} \label{tab:dyn_topo_obvs}\\



\toprule
\; & \multicolumn{2}{c}{\textsc{Dynamic topography} (km)} \\
\cline{2-3} \\
\textsc{Reference} & \textsc{Peak} & \textsc{RMS} & \textsc{Location} & \textsc{Viscosity} & Ra & \textsc{Internal heating} & \textsc{Model type} & \textsc{Dominant component}\\
\midrule 

\citet{Kiefer1991} & \makecell[tr]{7.5 \\ 5.2 \\ 3.6} & n/a  & general Regiones &  $f(z)$ & \makecell[tr]{Ra = $10^5$ \\ Ra = $10^6$ \\ Ra = $10^7$} & 0\% & Cylindrical plume & Viscous stress  \\
% after  \citep{Hager1985}


%\citet{Kiefer1992} &  7.5  & n/a & Global & Ra = $10^6$ & $\eta_m$ constant with high-$\eta$ stagnant lid & Numerical cylindrical plume & (3) & for $D_{\rm lid}$ = 130 km; gives scaling laws with Ra, assumes no internal heating (bottom of page 203). h from figure 9 constant visc \\


\citet{Moresi1995} & \makecell[tr]{5.8 \\ 3.8 \\ 5.1} & n/a & Atla Regio  &  \makecell[tl]{$f(T)$ \\ $f(T,z)$ \\ $f(T,z)$} & \makecell[tr]{Ra$_b$ = 2.4 $\times 10^6$ \\ Ra$_b$ = 1.3 $\times 10^6$ \\ Ra$_b$ = 1.0 $\times 10^6$} & 0\% & Axisymmetric convection & Thermal isostasy  \\
% Scaling: $\eta_0 \kappa / (d^2 \Delta\rho g)$ with $\eta_0$ from basal heating Ra, can't extrap to higher $\Delta\eta$ with linearized viscosity law. concerned with predicting admittance
 %

\citet{Nimmo1996} &  \makecell[tr]{1.16 \\ 1.47 \\ 2.85}  &  n/a  & Global & constant & \makecell[tr]{Ra$_b$ = 1.6 $\times 10^7$ \\  Ra$_b$ = 7.9 $\times 10^6$\\ Ra$_b$ = 4.0 $\times 10^6$ } & 0\% & Axisymmetric plume & Viscous stress \\



\citet{Solomatov1996a} & \makecell[tl]{$\sim$4 \\ $\sim$2} & n/a  & \makecell[cl]{Beta Regio \\ Average} & Frank-Kamenetskii  & Ra = $3 \times 10^7$  & 0\% & Cartesian convection & Thermal isostasy  \\
%Beta Regio (avg) scaled so admittance is 30 (15) m/km, fixed $d_m$ = 1600 km 



\citet{Kiefer1998} &  5.4\textendash 10.9 & 1.6\textendash 2.7  & Global & constant & Ra = $10^6$ & 73 \% & Spherical axisymmetric convection & Thermal boundary layer thinning \\
% Scaling: $(\rho_m \alpha \Delta T R_p) / (\rho_m - \rho_s)$  internal heating Ra $10^7$



\citet{Vezolainen2004} & 5.7 & n/a & Beta Regio & Frank-Kamenetskii  & Ra = $3 \times 10^7$ & 0\% & 3D Cartesian plume & Thermal isostasy  \\
%  fixed $\Delta T_m$ = 1100 K



%\citet{Orth2011} & 2.2 nondim & n/a  & Global &  Frank-Kamenetskii & Ra$_i$ = 10$^7$ & 0\% & 3D spherical shell & Thermal isostasy \\



\citet{Golle2012} & 3.3 & 2.7**  & Global & $f(z)$ & Ra$_b = 3.8\times 10^8$ & 0\% & Viscoelastic deformation coupled with thermal convection &  Not enough information \\



\citet{Benesova2012} &  \makecell[tr]{3.25 \\ n/a} & \makecell[tr]{n/a \\ TODO**}  & \makecell[tl]{Alta Regio \\ Global} & $f(z)$ & Ra = $2.8 \times 10^6$ & \textgreater 50\% & 3D spherical convection & Viscous stress \\
% specifically say no thermal isostasy, although Orth thesis says they do implicitly?



\citet{Huang2013} & 2\textendash 3 & 0.75 & Global & $f(T,z)$ & Ra = $1.8\times 10^7$ & 75\% & 3D spherical convection & Viscous stress \\
% goal is to simultaneously match number of plumes and GTR observations. semi-amplitude by eye from maps.  say case 15 is best fit to obvs. Ra fixed, using avg viscosity $2\times 10^{21}$ Pa s

\citet{Yang2016} & $\sim$5 & TODO** & Global & $f(T,z)$ &  Ra = $7.3\times 10^6$ & 80\% & 3D spherical convection & Viscous stress \\
% they say "The gravity anomaly is the summation of that caused by the dynamic topography and by the density heterogeneity itself." and "the topography of volcanic rises is mainly due to dynamic uplift"


\bottomrule


\end{longtable}
%\end{table}
\end{landscape}


\subsubsection{Dynamic topography in the solar system}\label{sec:dyn_top_ss}

We want to quantify the contribution of dynamic topography to total topography in the solar system so we can test our model, but this is not so easy because the data can be ambiguous. The dynamic component of topography can be tricky to parse from observed topographic heights. Here we give a short overview of attempts to do this. One can estimate the apparent depth of isostatic compensation using satellite measurements of the absolute topography and gravity or geoid anomaly. Larger values of the geoid with respect to the topography imply deeper isostatic compensation depths. Variations in the height of the geoid reflect distortions in density boundaries. %Inferences of dynamic topography are therefore quite sensitive to the assumed viscosity structure \citep{Karato2008a}.

Another complementary approach \citep[e.g.,][]{Smrekar1991, Kucinskas1994} is to assume Airy isostasy and predict the gravity anomaly resulting from a loading at some depth. Comparing this to observed topography produces a guess of the isostatic compensation depth. If these predictions are far from reality, one concludes that isostasy does not provide all the support. We focus on the convection-based approach, which is more applicable to our needs. Nevertheless, the fact that such models have not reproduced the observed relationships between gravity and topography \citep{Kiefer1986} has been criticized \citep{Orth2011} in its conclusion that isostasy is irrelevant for certain features on Venus, and that Venusian ``dynamic" topography is purely tractional. This is because the arguments against isostasy are necessarily based on assumptions about how thick the realistic crust could be, which is not well constrained.

\vspace{0.5cm}

\textit{\color{teal1} Venus.} Although Venus is hypsometrically unimodal, unlike Earth and Mars, satellite altimetry reveals vast lowland plains, possibly basaltic, dotted with areas of higher elevation. These elevated regions can be grouped into five older highland plateaux, steep-sided 2 km-high landmasses first seen as continents (it is still unknown how these formed), and nine younger, dome-like volcanic rises \citep{Phillips1998}. Dynamically, Venus is a more complicated planet than it may appear, and may not be a ``true" stagnant lid planet \citep{Breuer2015}. Several of the volcanic rise features have been labelled active hotspots \citep{Kiefer1991, Smrekar1991, Grimm1992, Smrekar1994, Stofan1995, Smrekar2010}. The arrival of the Magellan spacecraft in the 1990s brought on an influx of research into venusian dynamics, allowing us to diagnose its dynamic topography.\footnote{As is sadly the case with a lot of Venus research, the methodologies exhausted the data by the end of the 90s and Venus' dynamic topography is unsolved.}

Generally, Venus exhibits higher geoid-to-topography ratios than Earth, with volcanic rises such as Beta Regio and Atla Regio having paradoxically-deep apparent depths of compensation under Airy isostasy. These depths (up to 100s of kilometres) do not seem consistent with the expected crust thicknesses, which suggests an additional, non-isostatic support mechanism. Along with the strong correlation between the geoid and the observed topography, unlike Earth, this could suggest that these volcanic rises are close to completely supported by convection \citep{Kiefer1986, Kiefer1991, Smrekar1994, McKenzie1994, Smrekar1996, Nimmo1996, Simons1997, Pauer2006, James2013, Yang2016}. Meanwhile, highland plateaus have low geoid-to-topography ratios and are likely to be isostatically compensated, with the possible exception of the northern Ishtar Terra\footnote{hosting Venus' highest point Maxwell Montes, a few kilometeres taller than Mount Everest.} \citep{Grimm1991, Kucinskas1994, Simons1997}. %The fact that the isostatically-compensated highland plateaux are also the most ancient features is consistent with the idea that isostasy is representative of a long-term ``steady state" support mechanism. 
The dichotomy is not precise, and some regions are not fit by either a purely dynamic and a purely isostatic model \citep[e.g.,][]{Yang2016}. Analysis of the relationship between the geoid and the topography in the spectral domain (i.e., the admittance) shows for the most part that while short-wavelength topography ($l >$ 35) is isostatically compensated, long-wavelength ($l < 9$) is not, and more consistent with dynamic support \citep{Herrick2012, Konopliv1994, Arkani-Hamed1996, Pauer2006, Steinberger2010, Benesova2012, Huang2013, Yang2016}. 


Under the paradigm that volcanic rises are dynamically-supported, modellers have been trying to match the observed topographic profiles of individuals using numerical models of mantle plumes \citep{Kiefer1991, Kiefer1992, Moresi1995, Nimmo1996, Smrekar1996, Kiefer1998}. This is sometimes done with the intention of inverting the observered topographic observations to put constraints on unknown mantle parameters such as Ra and radial viscosity structure. Note that the error on Magellan topography is about 80 m \citep{Pettengill1992}, while the error introduced by modelling decisions is rather large. It is difficult to compare these models because they assume different rhelogical structures, aspect ratios, flow geometry (Cartesian or cylindrical axisymmetric), heating modes, Ra, and possibly different definitions of dynamic topography altogether. This is summarized in Table \ref{tab:dyn_topo_obvs}. Note that we care about the global RMS of dynamic topography, not so much the heights of individual features; Table \ref{tab:dyn_topo_obvs} also lists predictions of Venus' topographic spectra using dynamic topography, geoid, and admittance from 3D convection simulations \citep{Golle2012, Benesova2012, Huang2013, Wei2014}. For our purposes we are concerned with the output of convection models, not the absolute value of Venus' topography, since we have not agreed on its actual dynamic component. This remains tricky, as we will see in section \ref{sec:results}. 


Advances in numerical modelling, such as being able to include strongly-temperature-dependent viscosity (as expected for terrestrial planets) in 2- and 3-D, have bestowed the current understanding that spatial variations in lid thickness with thermal isostasy describes much more of Venus' topography \citep{Solomatov1996a, Orth2011}, agreeing with earlier theoretical relationships between geoid and observed topography that invoked thermal isostasy \citep{Kucinskas1994, Moore1995, Moore1997}. This behaviour cannot be reproduced exactly if viscosity is constant or only varies with depth, because \emph{lateral} viscosity variations affect the lid or lithospheric thicknesses and thus any interpretation of the observed signal \citep{Orth2011}. The gravity model of \citet{Smrekar1991} suggests that dynamic topography is not required to fit the long wavelength geoid-to-topography ratio, and it is explainable by thermal thinning alone, although the authors favour some support of dynamic topography. \citet{Kucinskas1994} conclude that Beta and Atla Regiones are compensated by thermal thinning, with Ovda and Thetis highlands in Airy isostasy. Nevertheless, we focus on studies that do include the tractional component of dynamic topography because it stems more obviously from 1D parameterized models that have no attention to lateral variations in $D_l$ or $\delta_u$.

%\begin{itemize}
%\item refer to definition crisis a bit more, maybe refer reader to column in Table \ref{tab:dyn_topo_obvs} where you sort these where possible
%\item do Parsons \& Daly consider thermal thinning? - claimed to not be. their stress balance has a hydrostatic term
%\end{itemize}


Finally, \citet{Zhong2002} and \citet{Golle2012} show that lithosphere elasticity might even make the real situation more complicated, by way of a non-negligible ``elastic filtering" of the topography. This effect is small for thin elastic thicknesses (Venus) and large for large ones (Mars), suggesting an additional parameter to pay attention to. We return to this in section \ref{sec:future-elastic}.



%\begin{itemize}

%\item from \citet{Yang2016}: Lowlands on Venus have negative gravity and geoid anomalies, and they are thought of as surface expressions of mantle downwellings (Bindschadler et al., 1992).



%\item  \citet{Simons1994} discuss something similar, their admittance values favouring the hypothesis that the nature of Venus' surface expressions of convection-crustal thickness coupling is transient rather than steady-state, although this paper and later one by the same authors \citep{Simons1997} argue that the present-day crust of Venus does not thin above upwelling plumes. 
%\item This is yet distinct from crustal thickening due to volcanism (which may be associated with a plume), which would be a mechanism within Airy isostasy.



%\item is the thing described by McKenzie 1994 figure 17 the same thing too? confused -- plume heat lowers viscosity of lower crust and it uplifts and  rifting occurs at the top, thrusting where lower crust at top of domr flows down around sides of dome. then as plume subsides, dome bulge subsides, but viscosity is high again and lower crust doesn't flow back





%\item .\citet{Smrekar1996} also suggest that decreasing positive GTRs across variuos volcanic rises suggests various ages of the plume; i.e., Beta Regio has the largest GTR because it is the most recent / hottest plume.  \citet{Kucinskas1994} interpreted an eastward increase in isostatic compensation in Aphrodite Terra as the decay of a hot mantle plume causing thermal thinning topography

%\item PEople have struggled to deal with lid or lithosphere thickness, trying to constrain it, make assumptions that it can't be larger than Earth's.  \citet{Kucinskas1994} argue that their model's 100-km-thick crust is actually real)



%\end{itemize}







\vspace{0.5cm}

\textit{\color{teal1} The rest.} Out of the terrestrial planets, we have focused on Venus because its observed topography is the most direct representation of dynamic topographic components (3) and (4). Large impact basins dominate Mercury and the Moon, while Mars and Titan have atmospheres which get involved in sedminentation and erosion \citep{Smrekar2018}. Io has some of the highest topography in the solar system \citep{Jaeger2003}, such as the 17.5-km Bo\"{o}saule Montes, but its tidal heating changes convection so the surface does not express it in the same way. Estimates of dynamic topography on Mars are complicated by its dichotomous crust---its geoid is dominated by the 5000-km-wide source of skewing that is Tharsis \citep{Phillips2001, Wieczorek2004}. This has been attributed to spherical harmonic degree-1 convection \citep{Zhong2001} as well as a giant impact \citep{Reese2006, Andrews-Hanna2008}. Earth's ``dynamic topography" is made more complex by plate motion, and is reviewed elsewhere. Hoggard (\citeyear{Hoggard2016}, \citeyear{Hoggard2020}) asks for $\pm1$ km on Earth from plume traction (component 3).






\subsection{Scaling up to super-}

Exoplanets are providing such a breakthrough to science in part because they are so common. Even basic measurements can bring the statistical power to complement more detailed studies of the cosmically-unrepresentative solar system. Planets called super-Earths appear to be a particular popular class of exoplanet, both around FGK stars as well as cooler M stars \citep{Petigura2013, Foreman-Mackey2014, Dressing2015}. These are distingtuished from sub-Neptunes by their radius, which is treated as a proxy for bulk composition \citep{Lopez2014}: ``super-Earths" are implied to be mainly rocky, with a thin atmosphere like Earth. Larger than this, the paradigm states that catastrophic gas accretion occurs. In contrast, ``sub-Neptunes" have a thick envelope possibly like the giant planets of our solar system.

Although the variability within both of these classes is still very unknown, planetary occurrence rate statistics seem to show distinct regimes: a paucity of planets around 1.5 $R_\oplus$ after correcting for observation bias, depending on orbital period and stellar mass \citep{Fulton2017, Cloutier2020}. This has been ascribed to photoevaporation, volatile-poor formation, core-powered mass loss, or some combination---hypotheses we may be able to discriminate between with newfound planets in opportune locations \citep{Cloutier2020b}.


\subsubsection{A note on terminology}
In the exoplanet literature, ``Earth-like" planets are usually classed as such according to their planetary radius (or mass) and orbital period (or equivalently, distance from star) \citep{Guimond2018}. This is because these are the first (and for now, only) observables we get for small planets. However, even a planet with mass and period precisely measured at near Earth-like values cannot be expected to remind us of home \citep{Tasker2017}. \citet{Moore2017} caution us against propagating the terms ``super-Earth" and ``habitable zone," associated with a planet's radius and period respectively, as they eschew scientific emotional distance with what is essentially clickbait:
\begin{quote}
We could just as well call a two-Earth-mass exoplanet a super-Venus as a super-Earth, but it is abundantly clear that those terms do not mean the same thing to most audiences and that they both imply vastly more than is known about any exoplanet ... Although metaphorically rich and easy to digest, such terms come with a lot of baggage that is dangerous to both the public perception of the work being done by the community and to the clarity of thought required to advance the field along several disciplinary fronts.
\end{quote}

With this in mind, we try to avoid the term ``super-Earth" and, for lack of anything catchier, call them massive rocky planets. 

\subsubsection{Observational constraints}

In the state-of-the-art, data have a limited but exciting capacity to constrain the interiors of exoplanets. To this end the most popular principle coming from geophysical theory is to develop planetary mass-radius relationships based on equations of state for different hypothetical mixtures of materials, namely hydrogen gas, water, silicate, and iron (first seen in \citeauthor{Seager2007} \citeyear{Seager2007} and \citeauthor{Valencia2007} \citeyear{Valencia2007}; reviewed in \citeauthor{Dorn2018} \citeyear{Dorn2018}). Then we fit observational data of mass and radius to these models. The actual distribution of observed planets in mass-radius space does not follow these equations-of-state trends, however. Rather than falling on a line, the data show wide scatter, and planets of a few $R_\oplus$ have measured masses of 2--15 $M_\oplus$ \citep{Jontof-Hutter2019}.

There is a degeneracy for a given mass-radius pair and the relative abundances of these four components\citep{Rogers2010, Zeng2016}. Planets with bulk densities much higher than pure silicate present the least-bad situation because in these models only iron is denser than rock, so we can at least put constraints on their core sizes \citep{Suissa2018}. These methods have been applied to a few low-mass planets, all close-in to their stars (because they produce the signals needed to measure radii). \citet{Wagner2012} look at CoRoT-7b and Kepler-10b, whose high densities appear to require large cores $\sim$60\% by mass, comparable to Mercury. The degeneracy tends to be worse for bulk densities greater than silicate rock: \citet{Southworth2017} compare the measured mass and radii of the apparently less-dense GJ 1132b to equation-of-state curves, concluding that a wider range of compositions fits the data, from Earth-like to water-rich. At the least, we can usually say if a planet's bulk density is too low to be mostly rock and iron.

In theory, one can reduce degeneracy by combining mass-radius observations with stellar abundanes of refractory elements, in particular the Fe/Si ratio \citep{Dorn2017, Brugger2017}, or estimates of a planet's Love number from light curves \citep{Kellermann2018}. N-body similations have been used to study the breadth of Fe/Mg ratios due to giant impacts \citep{Scora2020}.

Constraints can generally be improved with more precise radius measurements; we will ultimately be limited by models. For example, replacing half the Mg with Fe in mantle olivine for Kepler-33b leads to a reduction in core fraction from 53\% to 33\% \citep{Hakim2018}. \citet{Unterborn2016} predict structure (core radius, element partitioning between mantle and core) based on mineral physics to show how mass-radius relationships are more complictated than just linear combinations of equations of state. Effects like thermal inflation (important for water-rich worlds) are often ignored in models \citep{Thomas2016}. Further, which wavelength we use to obtain planetary radius also can introduce bias into mass-radius relationships \citep{Madhusudhan2015}.


As a starting point---despite all this---we assume a roughly Earth-like structure and composition, as this is what we understand best. We discuss this more in section \ref{sec:future-exoticplanets}

% despite degeneracies, can write scaling relationships for some things like core size, interior pressure, moment of inertia... \citep{Zeng2017, Suissa2018}



\subsubsection{Interior dynamics} \label{sec:superearth-dynamics}

The nature of planetary mantles on these larger bodies has inspired geophysical modelling attempts. This is tricky because central pressures are much higher: it might be risky to simply extrapolate our high-pressure laboratory measurements of rocks to these enormous pressures. Nevertheless, if the core-mantle boundaries at the mass limit of rocky exoplanets are not much greater than 630 GPa and 5000 K \citep{Unterborn2019}, then this regime is in the realm of modern lab measurements using shock compression \citep{Millot2015, Bolis2016}. \citet{Unterborn2019} note that uncertainties introduced from extrapolating equations of state are much less significant in the outcome of mass-radius relationships that uncertainties due to the unknown composition.

An important consequence of these higher pressures on convection involves the postperovskite dissociation into oxides. This leads to new phase transitions \citep[up to four, meaning a five-layered mantle;][]{vandenBerg2019} and less-viscous lower mantles via a pressure-weakening effect \citep{Umemoto2011, Karato2011, Tackley2013, Umemoto2017, Shahnas2018, Ritterbex2018, vandenBerg2019}. It may even melt the lowest mantle into a ``super basal magma ocean" \citep{Labrosse2007}. 

This affects the patterns of motion in the planet's mantle, for example, we might have high-pressure, high-viscosity middle layer(s) of sluggish convection bounded by low-viscosity convecting zones in the upper and lower mantle. This has cascading consequences, namely, the overall cooling rate of the planet would decrease \citep{Shahnas2018}. That is, the existence of a ``stratospheric" mantle layer inhibits the buoyant rise of hot plumes \citep{Kameyama2013, Miyagoshi2014}. Others have found the opposite behaviour---a second conductive stagnant lid covering the core---but the resulting effect of inefficient cooling remains the same \citep{Stamenkovic2012, Kameyama2016}. \citet{Tachinami2014} have found no thermal convection at all for planets \textgreater 5~M$_\oplus$ due to adiabatic compression in the deep mantle, depending on the pressure dependence of thermal expansivity.

\subsection{Topography in a planetary system}

Topography raises up land to be weathered, drawing carbon out of the atmosphere and transferring it to the oceans, where it eventually subducts into the mantle. This is the presumed primary mechanism by which Earth regulates its climate \citep{Walker1981}. There is a stabilizing feedback: warming the surface means faster weathering of silicate rock, drawing more CO from the atmosphere, weakening the greenhouse, and cooling the surface. Hence, the assumption that silicate weathering is affective is at the cornerstone of the classical circumstellar habitable zone theory \citep{Kasting1993}. That is, predictions of the width of the circumstellar liquid-water habitable zone around a star normally rely on the negative feedback of weathering; without this the habitable zone would be more narrow. However, to work efficiently, silicate weathering probably needs a minimum of exposed rock \citep{Abbot2012}. Although weathering also occurs on the seafloor \citep{Krissansen-Totton2017}, it is unclear whether this is efficient enough to sustain a stabilizing feedback loop. The location and surface area of exposed land exhibits complex far-field teleconnections that affect surface temperatures around the planet \citep{Sohl2017}. The impact of topography on climate has been considered for early Venus \citep{Way2016}. Major questions remain over climate regulation on rocky worlds lacking topographies: how robust would a silicate weathering feedback be? 

Another planetary system that may require land is biology. Uplifted land provides a means to concentrate minerals vital for prebiotic chemistry such as phosphate. Phosphorus is the limiting reagent in biochemical reactions (with respect to carbon and nitrogen), and surely part of any origins-of-life hypothesis. Further, the formation of aldehydes (precursor molecules of lipids, nucleic acids, and proteins) requires UV hydrolysis that cannot occur in the deep ocean.

\subsection{Statement of the problem}

Solar system studies have been extracting the parameters describing mantle convection for Mars and Venus through observations of the topography \citep[e.g.,][]{Wieczorek2015, Tosi2019}. We are concerned about the forward problem, examining what topography could look like on exoplanets with no solar system analogues, based on the interior. 

\begin{enumerate}
\item Can we extrapolate predictive models of topography to more massive rocky planets? This would ideally draw on the extant literature modelling the interior dynamics of such planets (although we do not touch on these details yet).
\item Can we develop simple scaling relationships to examine the nature of how topography changes with planet mass?
\item Is there a planet mass-limit to ``significant" topography? That is, would the topography be so insufficient that it does not participate in other planetary processes?
\end{enumerate}
