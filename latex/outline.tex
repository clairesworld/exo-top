\documentclass[10pt,a4paper]{article}
\usepackage[utf8]{inputenc}
\usepackage{amsmath}
\usepackage{amsfonts}
\usepackage{amssymb}
\usepackage{natbib}
\bibliographystyle{plainnat}

\author{Claire Marie Guimond, Oliver Shorttle, John Rudge}
\title{Claire PhD Research Outline}
\begin{document}
\maketitle

\section{Steps}
\begin{enumerate}
\item Dynamic topography (key parameters---$M_p$, $q_{\rm stellar}$, $H_0$, $t_{\rm age}$, core radius fraction) 
	\begin{enumerate}
	\item Steady state heat flux calculation as function of $a$, $M_p$, $X_{\rm K, U, Th}$
	\item Thermal model
	
		\begin{enumerate}
		
		\item Steady-state isoviscous mantle
		\item Temperature-dependent viscosity
		\item Age dependence
		\end{enumerate}
	\item Mapping temperatures to topography
	\end{enumerate}
		
\item Flexural topography
	\begin{enumerate}
	\item Lithospheric strength model
	\end{enumerate}
	
\item Things we wish we'd never thought of
	\begin{enumerate}
	\item Tidally-locked planets with non-uniform $T_s$, $T_m$
	\item K/U/Th ratios vary depending on condensation history in protoplanetary disc; this introduces uncertainty... maybe look at stellar catalogues?
	\end{enumerate}
	
\end{enumerate}
	
\section{Some simple calculations}

\subsection{Steady state heat flux}

We model a planet as a uniform sphere with mass $M_p$ and density $\rho$, and write energy balance equations for the surface and interior with fluxes in W. 

At the surface, $Q_{\rm sfc}^{\uparrow} = Q_*^{\downarrow} + Q^{\uparrow}_{\rm bl}$, where $Q_{\rm sfc}^{\uparrow}$ is the blackbody radiation lost to space, $Q_*^{\downarrow}$ is the incident stellar flux, and $Q_{bl}$ is the flux from the interior from conduction across the boundary layer. The left hand side comprises outgoing fluxes, and the right hand side comprises incoming fluxes. 

In the interior, the outgoing conductive flux across the boundary layer is balanced by internal radiogenic heating, $Q^{\uparrow}_{\rm bl} = Q_{\rm rad}$. If we don't know the boundary layer flux, which would depend on interior temperature, then
\begin{equation}
Q_{\rm sfc}^{\uparrow} = Q_*^{\downarrow} + Q_{\rm rad}.
\end{equation}

We can make order-of-magnitude estimations. First, the stellar contribution is parameterized as
\begin{equation}
Q_*^{\downarrow} = \frac{L_*(1-A)}{4a^2} \; \pi R_p^2,
\end{equation}
where $L_*$ is the stellar luminosity in W, $A$ is the planetary albedo, $a$ is the orbital semi-major axis in m, and $R_p = \left(\frac{3M_p}{4 \pi \rho}\right)^{1/3}$ is the radius of the planet. For a blackbody with 0 albedo, 
\begin{equation}
Q_*^{\downarrow} = \frac{10^{26} \; {\rm W}}{\left(10^{11}\right)^2 \; \rm{m}^2} \cdot \left(\frac{10^{24}\;{\rm kg}}{10^3 \; {\rm kg}\;{\rm m}^3}\right)^{2/3} = 10^{17} \; {\rm W}.
\end{equation}


Next,
\begin{equation}
Q_{\rm rad} = \sum^{\rm K, U, Th}_n [ H_{0, n} \; X_n \; n_0 \; e^{\ln 2 \frac{\tau}{\tau_{1/2 ,n}}} ] \; M_p,
\end{equation}
where we are summing over the heat-producing elements K, U, and Th, $H_{0, n}$ is the heat production of the $n^{th}$ isotope in W kg$^{-1}$, $X_n$ is the natural abundance of the $n^{th}$ isotope in terms of mass (compared to all isotopes of that element), $n_0$ is the concentration by mass of that element in the planet, $\tau$ is the age of the planet in seconds,  $\tau_{1/2 ,n}$ is the half-life of the $n^{th}$ isotope in seconds, and $M_p$ is the mass of the planet in kg. Using Earth values, and taking only the isotope with highest-order heat flux (40-K), we have
\begin{equation}
Q_{\rm rad} = \frac{10^{-5}\;{\rm W}}{{\rm kg}\;^{40}{\rm K}} \cdot \frac{10^{-3}\;\rm{kg\;K}}{\rm kg} \cdot \frac{10^{-4}\;{\rm kg}\;^{40}{\rm K}}{\rm{kg\;K}} \cdot 10^{24} \; {\rm kg} = 10^{12} \; {\rm W}.
\end{equation}

Therefore, $Q_{\rm sfc}^{\uparrow} = 10^{17} \; {\rm W} + 10^{12} \; {\rm W} = 10^{17} \; {\rm W}$.

To put this in terms of flux densities, we assume an isotropically-emitting sphere, so we have $q_{\rm sfc}^{\uparrow} = Q_{\rm sfc}^{\uparrow}/\rm{SA}_p$ where $\rm{SA}_p = 4\pi\;R_p^2$ is the of the planet's surface area, giving 
\begin{equation}
q_{\rm sfc}^{\uparrow} = \frac{10^{17} \; {\rm W}}{10^1 \cdot \left(10^{24}\;{\rm kg}/10^3 \; {\rm kg}\;{\rm m}^3\right)^{2/3}} = 10^2 \; \rm{W\;m}^{-2}.
\end{equation}

\subsection{Thermal evolution of mantle with uniform viscosity and temperature}

The thermal state of a planetary mantle is controlled by the balance of heat loss (by conduction through the lid), $Q_{\rm lid}$, with (radiogenic) heating, $Q_{\rm rad}$. Secular core cooling represents an additional source of heat to the mantle, here represented by an optional parameter. In this section we make the preliminary simplifying assumption that viscosity does not change with temperature.

The change in temperature due to this balance is governed by the O.D.E.,
\begin{equation}
M_m c_v \frac{{\rm d}T_m}{{\rm d}t} = -Q_{\rm lid} + Q_{\rm rad},
\end{equation}
where $M_m$ is the mantle mass, $c_v$ is the mantle specific heat capacity at constant volume, $Q$ is the integral of the heat flux over surface area or volume, respectively, and the sign of $Q$ indicates cooling or heating. We use {\tt scipy.integrate} to integrate this O.D.E. from $\tau_0$, taken to be the time of magma ocean cooling, until the age of the planet, $\tau_f$.  

Table \ref{tab:params} lists the free parameters in this model, being the planetary mass, $M_p$, semi-major axis, $a$, and core mass fraction, CMF; as well as the stellar luminosity, $L_*$, planetary albedo, Alb, mantle thermal conductivity, $k_m$, mantle thermal expansivity, $\alpha_m$, mantle density, $\rho_m$, mantle heat capacity at constant volume, $c_v$, critical Rayleigh number for convection, Ra$_{\rm crit}$, and mantle dynamic viscosity, $\eta_m$.

From this we also calculate a number of derived parameters:

the radius of the planet, $R_p$, based on \citet{Zeng2016} using PREM,
\begin{equation}
\frac{R_p}{R_E} = (1.07 - 0.21\; {\rm CMF})\left(\frac{M_p}{M_E}\right)^{1/3.7},
\end{equation}
which has surface area SA$_p$; the radius of the core, using the scaling relationship from \citet{Zeng2017},
\begin{equation}
R_c = R_p \; {\rm CMF}^{0.5},
\end{equation}
which has surface area SA$_c$; the surface gravity,
\begin{equation}
g_{\rm sfc} =\frac{6.674\times 10^{-11}M_p}{R_p^2};
\end{equation}
the surface temperature, $T_s$, assuming the surface is in blackbody equilibrium,
\begin{align}
T_s &= \left(\frac{q_* \pi R_p^2}{\sigma \; {\rm SA}_p}\right)^{1/4},\\
q_* &= \frac{L_*(1-{\rm Alb})}{4 \pi a^2};
\end{align}
the thermal diffusivity of the mantle,
\begin{equation}
\kappa_m = \frac{k_m}{\rho_m c_p};
\end{equation}

A vigorously-convecting medium is said to be isothermal except in its upper and lower boundary layers, which have steep temperature gradients.  Heat escapes the system via conduction through the boundary layers. For mantle thermal evolution, we only consider heat transport through the upper boundary layer. Hence mantle cooling depends on the boundary layer thickness, $d_{\rm lid}$, as well as the temperature difference between the mantle and surface, $\Delta T = T_m - T_s$, and the thermal conductivity of the mantle, $k_m$:
\begin{align}
Q_{\rm lid} &= \int_{\rm A} q_{\rm lid} \; {\rm d}A = q_{\rm lid} \; {\rm SA}_p,\\
q_{\rm lid} &= \frac{k_m \Delta T}{d_{\rm lid}}.
\end{align}
Here, $d_{\rm lid}$ is calculated by setting the Rayleigh number to its critical value for initiating convection:
\begin{equation}
d_{\rm lid} = \left(\frac{{\rm Ra}_c \eta_m \kappa_m}{\alpha_m \rho_m g_{\rm sfc} \Delta T}\right)^{\frac{1}{3}}
\end{equation}

Meanwhile, radiogenic heating is the same as in the previous section:
\begin{align}
Q_{\rm rad} &= \int_{\rm A} q_{\rm rad} \; {\rm d}M = q_{\rm rad} M_p,\\
q_{\rm rad} &= \sum^{\rm K, U, Th}_n \left[ H_{0, n} \; X_n \; n_0 \; e^{\ln 2 \frac{\tau}{\tau_{1/2 ,n}}} \right],
\end{align}

If CMF \textgreater\; 0, then the heat flux across the core-mantle boundary is obtained by analogy to Earth, assuming the same flux density, $q_{\rm CMB}$:
\begin{equation}
Q_{\rm CMB} = q_{\rm CMB}\;{\rm SA}_c
\end{equation}

%\begin{equation}
%{\frac{{\rm d}T_m}{{\rm d}t}}_{\rm rad} = \frac{h [{\rm W} \; {\rm kg}^{-1}] \times M_m [{\rm kg}]}{M_m [{\rm kg}] \times c_v [{\rm J}\;{\rm kg}^{-1}\;{\rm K}^{-1}]} = \frac{{\rm K}}{{\rm s}}
%\end{equation}

\bibliography{planetary-interiors}


\end{document}
