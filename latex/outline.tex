\documentclass[10pt,a4paper]{article}
\usepackage[utf8]{inputenc}
\usepackage{amsmath}
\usepackage{amsfonts}
\usepackage{amssymb}
\author{Claire Marie Guimond, Oliver Shorttle, John Rudge}
\title{Claire PhD Research Outline}
\begin{document}
\maketitle

\section{(Time)line}
\begin{enumerate}
\item Dynamic topography (key parameters---$M_p$, $q_{\rm stellar}$, $H_0$, $t_{\rm age}$, core radius fraction) 
	\begin{enumerate}
	\item Steady state heat flux calculation as function of $a$, $M_p$, $X_{\rm K, U, Th}$
	\item Thermal model
	
		\begin{enumerate}
		
		\item Steady-state isoviscous mantle
		\item Temperature-dependent viscosity
		\item Age dependence
		\end{enumerate}
	\item Mapping temperatures to topography
	\end{enumerate}
		
\item Flexural topography
	\begin{enumerate}
	\item Lithospheric strength model
	\end{enumerate}
	
\item Things we wish we'd never thought of
	\begin{enumerate}
	\item Tidally-locked planets with non-uniform $T_s$, $T_m$
	\item K/U/Th ratios vary depending on condensation history in protoplanetary disc; this introduces uncertainty... maybe look at stellar catalogues?
	\end{enumerate}
	
\end{enumerate}
	
\section{Some simple calculations}

\subsection{Steady state heat flux}

We model a planet as a uniform sphere with mass $M_p$ and density $\rho$, and write energy balance equations for the surface and interior with fluxes in W. 

At the surface, $Q_{\rm sfc}^{\uparrow} = Q_*^{\downarrow} + Q^{\uparrow}_{\rm bl}$, where $Q_{\rm sfc}^{\uparrow}$ is the blackbody radiation lost to space, $Q_*^{\downarrow}$ is the incident stellar flux, and $Q_{bl}$ is the flux from the interior from conduction across the boundary layer. The left hand side comprises outgoing fluxes, and the right hand side comprises incoming fluxes. 

In the interior, the outgoing conductive flux across the boundary layer is balanced by internal radiogenic heating, $Q^{\uparrow}_{\rm bl} = Q_{\rm rad}$. If we don't know the boundary layer flux, which would depend on interior temperature, then
\begin{equation}
Q_{\rm sfc}^{\uparrow} = Q_*^{\downarrow} + Q_{\rm rad}.
\end{equation}

We can make order-of-magnitude estimations. First, the stellar contribution is parameterized as
\begin{equation}
Q_*^{\downarrow} = \frac{L_*(1-A)}{4a^2} \; \pi R_p^2,
\end{equation}
where $L_*$ is the stellar luminosity in W, $A$ is the planetary albedo, $a$ is the orbital semi-major axis in m, and $R_p = \left(\frac{3M_p}{4 \pi \rho}\right)^{1/3}$ is the radius of the planet. For a blackbody with 0 albedo, 
\begin{equation}
Q_*^{\downarrow} = \frac{10^{26} \; {\rm W}}{\left(10^{11}\right)^2 \; \rm{m}^2} \cdot \left(\frac{10^{24}\;{\rm kg}}{10^3 \; {\rm kg}\;{\rm m}^3}\right)^{2/3} = 10^{17} \; {\rm W}.
\end{equation}


Next,
\begin{equation}
Q_{\rm rad} = \sum^{\rm K, U, Th}_n [ H_{0, n} \; X_n \; n_0 \; e^{\ln 2 \frac{\tau}{\tau_{1/2 ,n}}} ] \; M_p,
\end{equation}
where we are summing over the heat-producing elements K, U, and Th, $H_{0, n}$ is the heat production of the $n^{th}$ isotope in W kg$^{-1}$, $X_n$ is the natural abundance of the $n^{th}$ isotope in terms of mass (compared to all isotopes of that element), $n_0$ is the concentration by mass of that element in the planet, $\tau$ is the age of the planet in seconds,  $\tau_{1/2 ,n}$ is the half-life of the $n^{th}$ isotope in seconds, and $M_p$ is the mass of the planet in kg. Using Earth values, and taking only the isotope with highest-order heat flux (40-K), we have
\begin{equation}
Q_{\rm rad} = \frac{10^{-5}\;{\rm W}}{{\rm kg}\;^{40}{\rm K}} \cdot \frac{10^{-3}\;\rm{kg\;K}}{\rm kg} \cdot \frac{10^{-4}\;{\rm kg}\;^{40}{\rm K}}{\rm{kg\;K}} \cdot 10^{24} \; {\rm kg} = 10^{12} \; {\rm W}.
\end{equation}

Therefore, $Q_{\rm sfc}^{\uparrow} = 10^{17} \; {\rm W} + 10^{12} \; {\rm W} = 10^{17} \; {\rm W}$.

In terms of flux densities, we assume an isotropically-emitting sphere, so we have $q_{\rm sfc}^{\uparrow} = Q_{\rm sfc}^{\uparrow}/\rm{SA}_p$ where $\rm{SA}_p = 4\pi\;R_p^2$ is the of the planet's surface area, giving 
\begin{equation}
q_{\rm sfc}^{\uparrow} = \frac{10^{17} \; {\rm W}}{10^1 \cdot \left(10^{24}\;{\rm kg}/10^3 \; {\rm kg}\;{\rm m}^3\right)^{2/3}} = 10^2 \; \rm{W\;m}^{-2}.
\end{equation}


\end{document}
