\section{Model description}\label{sec:methods}









We start with a simple model of thermal history based on parameterized convection. From this we can estimate dynamic topography directly using scaling relationships in the literature. 

Table \ref{tab:params} lists the parameters used in this work. The key free parameters we are interested in tuning are the planet mass, $M_p$, and present-day radiogenic heating rate, $H_{4.5}$. The initial conditions contain information about the formation history of the planet, namely the leftover gravitational energy of accretion and core segregation. Although thermal expansivity $\alpha$, mantle density $\rho_m$, and thermal diffusivity $\kappa_m$ vary with depth \citep{Anderson1987}, we use values assumed for Earth's upper mantle. Although not explored fully in this work, the core mass fraction, CMF, is an input parameter that along with $M_p$ controls the maximum depth of the convecting region, and could also be tenable to astrophysical constraints \citep{Suissa2018}.

From the input parameters in Table \ref{tab:params} we can find the derived bulk properties: the radius of the planet, $R_p$, based on the mass-radius relation in \citet{Zeng2016},
\begin{equation}
\frac{R_p}{R_E} = (1.07 - 0.21\; {\rm CMF})\left(\frac{M_p}{M_E}\right)^{1/3.7},
\end{equation}
which has surface area $S_p$. Next, the radius of the core, using the scaling relationship from \citet{Zeng2017},
\begin{equation}
R_c = R_p \; {\rm CMF}^{0.5},
\end{equation}
which has surface area $S_c$; the surface gravity, $g_{\rm sfc} = 6.674\times 10^{-11}M_p/R_p^2$ in m s$^{-2}$; and the thermal diffusivity of the mantle, $\kappa_m = k_m/(\rho_m c_m)$ in ms$^{2}$ s$^{-1}$, where $k_m$ is the mantle thermal conductivity in W m$^{-1}$ K$^{-1}$, $\rho_m$ is the mantle density in kg m$^{-3}$, and $c_m$ is the mantle heat capacity at constant volume in J kg$^{-1}$ K$^{-1}$. Our Mars and Venus cases use the observed $R_p$ and $M_p$ for these planets which are known precisely, along with $R_c$ assumptions from the literature. 


We can also find the surface temperature, $T_s$, assuming the surface is in blackbody equilibrium,
\begin{align}
T_s &= \left(\frac{q_* \pi R_p^2}{\sigma_{\rm SB} \; S_p}\right)^{1/4},\\
q_* &= \frac{L_*(1-{\rm Al})}{4 \pi a^2};
\end{align}
where $q_*$ is the incident stellar radiation in W m$^{-2}$, Al is the geometric albedo, $L_*$ is the stellar luminosity in W, $\sigma_{\rm SB}$ is the Stefan–Boltzmann constant in W m$^{-2}$ K$^{-4}$, and $T_s$ is in K. For solar system analogue cases we use the measured $T_s$ (see Table \ref{tab:params}). $T_s$ has no effect on thermal evolution and topography.




\subsection{Temperature-dependent viscosity, stagnant lid convection}\label{sec:viscosity-model}

\subsubsection{Mantle rheology}

We use an Arrhenius law for the diffusion creep of dry olivine \citep{Karato1993}. Experiments show that viscosity has an Arrhenius dependence on temperature,
\begin{equation}
\eta(T) = \frac{\mu}{2 A_{rh}} \left(\frac{h_{rh}}{B}\right)^m \exp\left(\frac{E_a}{R_b T}\right),
\end{equation}
where $R_b$ is the gas constant in J mol$^{-1}$ K$^{-1}$, $E_a$ is the activation energy in J mol$^{-1}$, $\eta$ is the dynamic viscosity in Pa s, $\mu$ is the shear modulus in Pa, $A_{rh}$ is a preexponential factor in s$^{-1}$ (of order $10^{15}$ for diffusion creep), $B$ is the Burgers vector in m (quantifying the distortion in the crystal lattice due to dislocation), $h_{rh}$ is the assumed grain size of the rock in m, and $m$ is an exponent which depends on the type of creep ($m=2$ for diffusion creep). We choose $h_{rh}$ to match the linearized rheology law of \citet{Thiriet2019}.

\subsubsection{Heat fluxes}
We are solving equation (\ref{eq:T_ODE}) using the explicit Runge-Kutta method of order 5(4), where the heat fluxes are calculated using the common scalings from parameterized convection models. Throughout, heat fluxes per unit volume or area are given by lowercase $q$, and the planet-integrated value in uppercase $Q$. The radiogenic heat flux in W kg$^{-1}$ is:
\begin{align}
q_{\rm rad} &= H_{4.5}\sum^{\rm K, U, Th}_n \left[ h_{n}  e^{\lambda_n (t_f - t)} \right], \nonumber \\
h_n &= \frac{c_n p_n}{\sum c_n p_n}, \\
c_n &= \frac{n_nX_n}{n_{\rm U} X_{\rm U}} \nonumber ,
\end{align}
where we are summing over the heat-producing elements K, U, and Th, $H_{4.5}$ is the present-day heat production in W kg$^{-1}$, $h_n$ is the relative contribution of the $n^{th}$ isotope, $c_n$ is radiogenic element abundance relative to U, $n_n$ is the natural isotopic abundance in terms of mass, $X_n$ is the bulk planet abundance in terms of mass, $p_n$ is the heating power in W kg$^{-1}$, $t_f$ is the present-day age in s, and $\lambda_n$ is the decay constant in s$^{-1}$ \citep{Korenaga2006}. We use the canonical values for decay constants and isotopic natural abundances \citep{Jaupart2015}.

Heat travels by conduction through the upper and lower thermal boundary layers, so these fluxes depend on thermal conductivity, $k_m$, and the boundary layer thickness, $\delta$, with subscripts $u$ and $c$ referring to the upper and lower layers respectively: 
\begin{align}
q_{u, c} &= k_m \frac{\Delta T_{u, c}}{\delta_{u, c}} \label{eq:q_u}\\
\delta_{u, c} &= d \left(\frac{{\rm Ra}_{{\rm crit}, u, c}}{{\rm Ra}_{rh, u, c}}\right)^\beta,\label{eq:d_u}
\end{align}
where we equate Ra$_{rh}$ with the interior Rayleigh number from equation (\ref{eq:Ra}). We assume $\beta = 1/3$, so $q$ is independent of the layer depth $d$.\footnote{Numerical studies give $\beta$ around 0.30 \citep{Thiriet2019}.} The temperature contrasts are given by $\Delta T_u = T_m - T_l$, and $\Delta T_c = T_c - T_m$, where $T_l$ is the temperature at the top of the convecting region (\ref{eq:Tl}). The viscosity used to define Ra is taken at the isothermal interior mantle ($T_m$) for the upper boundary layer, and at $(T_c + T_m)/2$ for the lower, after \citet{Thiriet2019}. Gravity is evaluated at $R_p$ for $\delta_u$ and at $R_c$ for $\delta_c$. The critical Rayleigh numbers are Ra$_{{\rm crit}, u}$ = Ra$_{{\rm crit}}$, and Ra$_{{\rm crit}, c}$ = 0.28Ra$_{i}^{0.21}$, where Ra$_{i}$ is the interior Rayleigh number in (\ref{eq:Ra}) \citep{Thiriet2019}.




\subsubsection{Temperatures and heat flow in the lid}

We distinguish between the upper thermal boundary layer of the convecting region and the conducting stagnant lid. The lid spans the radius $R_l$ where $T = T_l$ and the surface of the planet at $R_p$. Fluid dynamics experiments show that the temperature drop between $T_m$ and $T_l$ is proportional to the viscous temperature scale $\Delta T_{rh}$, or the rate of viscosity change with temperature \citep{Davaille1993},
\begin{align}
\label{eq:Tl}
T_{l} &= T_m - a_{rh} \Delta T_{rh} \\
\Delta T_{rh} &= \frac{\eta(T_m)}{{\rm d}\eta/{\rm d}T_m\vert_{Tm}} = \frac{R^* T_m^2}{E_a},
\end{align}
where $R^*$ is the universal gas constant in J mol$^{-1}$ K$^{-1}$, and $a_{rh} = 2.44$ for $\beta = 1/3$ based on fits to 3D convection models \citep{Thiriet2019}. 



In the stagnant lid of thickness $D_l = R_p - R_l$, the temperature decreases from $T_l$ to $T_s$ by conduction with internal heating, assumed to be in steady state:
\begin{align}
T_{\rm lid}(r) &= \frac{-\rho_m H(t)}{6k_l} r^2 + \frac{c_1}{k_m r} + c_2 \\
    c_1 &= k_m \frac{T_l - T_s - \rho_m H(t)/(6 k_m) \left(R_p^2 - R_l^2\right)}{1/R_l - 1/R_p} \\
    c_2 &= T_s + \frac{\rho_m H(t)}{6 k_m} R_p^2 - \frac{c_1}{k_m R_p}
\end{align} 
where $H(t)$ is the lithospheric heat production in W kg$^{-1}$, assumed to be constant with $r$ and equal to the mantle value. In reality, we might anticipate more radiogenic heating in the lid because it overlaps with the low density crust extracted by melting.

Evaluating the associated conductive heat flux, $-k_l \d T/\d r$, at $r = R_p$ gives the surface heat flux:
\begin{align}
q_{\rm lid} &= -\frac{\rho_m H(t)}{3k_m}r - \frac{c_1}{k_m r^2},\\
q_{s} &= -\frac{\rho_m H(t)}{3k_m}R_p - \frac{c_1}{k_m R_p^2}. \label{eq:q_s}
\end{align}



\subsubsection{Lid thickness}


The lid does not instantly grow or shrink in response to a change in $q_{u}$. There is a lag in which $D_l$ adjusts such that the difference between the flux out of the top of the lid and the flux into the base of the lid is minimized: \citep{Thiriet2019}:
\begin{equation}\label{eq:D_l}
\frac{\d D_l}{\d t} = \frac{q_{\rm lid}\vert_{R_l} -q_u}{\rho_m c_{m} (T_m - T_l)} . 
\end{equation}
From this we can calculate $R_l = R_p - D_l$. We account for the mass of the convecting region $M_m$ changing with $R_l$. (\ref{eq:D_l}) ignores the time-dependence of heat conduction in the lid.




\subsection{Height of dynamic topography}

We consider three scaling laws to estimate the root-mean-square (RMS) amplitude of dynamic topography. The first can be written by balancing  $\tau_zz$ (\ref{eq:tau_param}) with $p_0 = \rho_m g \Delta h$:
\begin{equation}\label{eq:dyn_top_stress_A}
\Delta h_{\rm RMS} = C_1 \alpha_m \Delta T_{rh} \delta_u.
\end{equation}
where $C_1 = 2$ from fits to spherical geometry, temperature-dependent viscosity models \citep{Reese2005}.

The second is provided by the cartesian geometry, constant-viscosity case of \citet{Parsons1983} equation (33), with the exponent on the surface flux Rayleigh number taken to equal 0.5 (their no-internal-heating case):
\begin{equation}\label{eq:PD83_A}
\Delta h_{\rm RMS} = C_2 \left[\frac{\alpha_m q_{s} \eta_m(T) \kappa_m}{\rho_m g k_m}\right]^\frac{1}{2},
\end{equation}
where $C_2 = 5.4$ to match the predictions of RMS dynamic topography in \citet{Lees2019}, and $q_u$ is replaced by the surface flux $q_s$. This should be equivalent to equation (\ref{eq:dyn_top_stress}); the approximation expressed in the latter comes from the notion that most of the contribution to surface topography lies in the displacement of stress boundaries in the thermal boundary layer.

The third scaling is given in (\ref{eq:KH92}) and is based on fits to 2D cylindrical, isoviscous numerical experiments \citep{Kiefer1992}.

Although both (\ref{eq:PD83_A}) and (\ref{eq:KH92}) are scalings for ``peak topography" over a given rising plume, we naively treat them as the global RMS dynamic topography, $\Delta h_{\rm RMS}$, assuming that the values of $\tau_{zz}$ they are based on reflects the global RMS convective stress.

