\section{A note on terminology}
In the exoplanet literature, ``Earth-like" planets are usually classed as such according to their planetary radius (or mass) and orbital period (or equivalently, distance from star) \citep{Guimond2018}. This is because these are the first (and for now, only) observables we get for small planets. However, even a planet with mass and period precisely measured at near Earth-like values cannot be expected to remind us of home \citep{Tasker2017}. \citet{Moore2017} caution us against propagating the terms ``super-Earth" and ``habitable zone," associated with a planet's radius and period respectively, as they eschew scientific emotional distance with what is essentially clickbait:
\begin{quote}
We could just as well call a two-Earth-mass exoplanet a super-Venus as a super-Earth, but it is abundantly clear that those terms do not mean the same thing to most audiences and that they both imply vastly more than is known about any exoplanet ... Although metaphorically rich and easy to digest, such terms come with a lot of baggage that is dangerous to both the public perception of the work being done by the community and to the clarity of thought required to advance the field along several disciplinary fronts.
\end{quote}

With this in mind, we try to avoid the term ``super-Earth" and temporarily call them massive rocky planets. 